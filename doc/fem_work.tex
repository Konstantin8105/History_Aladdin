
% =============================================================== 
% Design of programming language for fem stack machine          *
% ===============================================================

\input epsf

% \documentstyle[singlespace,11pt]{report}
\documentstyle[11pt]{report}
\renewcommand{\baselinestretch}{1.5}
\textheight 9.0in
\textwidth 6.40in
\voffset -0.85in
\hoffset -0.60in
\def\R{{\bf R}}
\def\j{{\rm j}}
\def\C{{\bf C}}
%\renewcommand{\thefigure}{\thechapter.\arabic{figure}}
\begin{document}

\title{\bf A LANGUAGE FOR INTERACTIVE ENGINEERING MATRIX AND FINITE ELEMENT ANALYSIS}

\author{                               \\
                                       \\
                                       \\
        \bf Mark Austin, Xiaoguang Chen \\
        \bf Wane-Jang Lin and Lanheng Jin\\
                                         \\
            Institute for Systems Research\\
            and \\
            Department of Civil Engineering\\
            University of Maryland \\
            College Park, MD 20742}


\maketitle

\begin{abstract}

This report describes the design, architecture, implementation
and prototype testing of a computational toolkit for engineering matrix and finite element analysis.
The computational toolkit design deviates from the traditional design of finite
element analysis packages; indeed, the command language
in this package may be viewed as an combination of C language, matrix operation language,
and languages specially for engineering analysis such as structural dynamics and 
finite element analysis.
\vspace{0.15in}

The software consists several major parts:
(1) a language, designed with YACC grammar, for control flow, conditional
branching, matrix and engineering calculations;
(2) a matrix operation package, including linear algebraic equation solver,
eigen value problem solver and sub-space iterations;
(3) a finite element analysis package, with two- and three-dimensional
truss/beam, shell, plate, plane stress \& plane strain elements;
(4) a units package, including both SI and 
US units and their conversions for numbers, variables, and matrices.
The finite element package is capable of analyzing
linear/nonlinear, static/dynamic structural problems. 
\vspace{0.15 in}

The combination of above mentioned four parts makes this took kit is quite
powerful as demonstrated in this report. A engineer can write down his
numerical algorithm in the sheet and type the same algorithm as the inputfile.
There is no need for compile process even if the algorithm is changed.
A variety of problems has been solved by using this package including:
nonlinear optimization problem with QP method, root of nonlinear equations
with BFGS algorithm, linear dynamic problems with Newmark method, modal
analysis, finite element analysis of structures, made of linear and nonlinear
materials.
\end{abstract}

\tableofcontents

\part{\bf A LANGUAGE FOR ENGINEERING ANALYSIS}

\input chapter_shell.tex

\bigskip\bigskip 

\bibliography{/homes/xgchen/SRC/bibliography/references}   % specify bibliography in eosdis.bib
\bibliographystyle{unsrt}                      % specify plain.sty as style file

\addcontentsline{toc}{chapter}{\protect\numberline{}{\bf REFERENCE}}

\end{document}


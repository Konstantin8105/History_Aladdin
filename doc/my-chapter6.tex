\chapter{Input Files for Finite Element Analysis Problems}

\vspace{0.15 in}
\noindent\hspace{0.5 in}
In this chapter we exercise ALADDIN's finite element capabilities
by working through four problem in detail.
The finite element problems are:

\begin{description}
\item{[1]}
A linear static analysis of a five story moment resistant frame.
The frame supports gravity loads plus a moderate lateral load.
\item{[2]}
A linear time-history earthquake analysis of the five story frame described in Item [1].
The time-history response is generated by a 1979 El Centro ground motion.
\item{[3]}
An AASHTO working stress design of a composite highway bridge girder,
with rule checking built into the input file.
\item{[4]}
A linear elastic three dimensional analysis of 2-span highway bridge.
Again, the bridge is modeled with the four node shell element.
Moment envelopes are computed for a gravity loads plus a moving truck live load.
\end{description}

\vspace{0.15 in}\noindent
The analyses in problems [3] and [4] include moving live loads.
The application of ALADDIN to the solution of nonlinear static analyses 
with an 8-node shell element is reported in Chen and Austin ~\cite{chen95}.

\section{Linear Static Analyses}

\subsection{Analysis of Five Story Moment Resistant Frame}

\label{fig: five-story-building-elevation-view}

\vspace{0.15 in}\noindent
{\bf Description of Problems :}
We begin with a linear static analysis of a five story steel moment resistant frame.
A front elevation view of the frame with dimensions and preliminary
section sizes is shown in Figure \ref{fig: five-story-building-elevation-view}.
We will assume that this frame is one in a row of frames spaced at {\tt 20} {\tt ft} centers,
as shown in Figure \ref{fig: five-story-building-plan-view}.

\clearpage

\begin{figure}[ht]
\vspace{0.10 in}
\epsfxsize= 4.7truein
\centerline{\epsfbox{my-chapter6-fig2.ps}}
\caption{Elevation View of Five Story Planar Steel Frame}
\label{fig: five-story-building-elevation-view}
\end{figure}

\begin{figure}[ht]
\vspace{0.30 in}
\epsfxsize= 4.4truein
\centerline{\epsfbox{my-chapter6-fig4.ps}}
\caption{Plan View of Five Story Building}
\label{fig: five-story-building-plan-view}
\end{figure}

\clearpage
\vspace{0.15 in}
\noindent\hspace{0.50 in}
The frame is constructed with two steel section types, and one material type. 
The columns are of type UC $12 \times 12$,
with $I_{zz}$ = 1541.9 $in^4$, $I_{yy}$ = 486.3 $in^4$,
and cross section area 47.38 $in^2$.
The beams are of section type UB $21 \times 8 1/4$,
with $I_{zz}$ = 1600.3 $in^4$, $I_{yy}$ = 66.2 $in^4$,
and cross section area 21.46 $in^2$.
The beams and columns have Young's Modulus 29000 ksi, and yield stress 36.0 ksi.

\vspace{0.15 in}
\noindent\hspace{0.5 in}
The frame will be analyzed for gravity loads
plus a moderate lateral earthquake load.
Wind and other loads are omitted for simplicity.
The specified dead load is {\tt 80} ${\tt lbf/ft^2}$,
and the specified live load {\tt 40} ${\tt lbf/ft^2}$ for the four floors,
and {\tt 20} ${\tt lbf/ft^2}$ for the roof.
We will assume that the tributary area for gravity loads
is rectangular in shape, with a uniform load distribution
along each girder span of the frame.
With the frames spaced {\tt 20} {\tt ft} apart,
it follows that floor gravity loads are {\tt 0.200} kips/in,
and {\tt 0.1667} kips/in on the roof.
Total gravity loads are {\tt 663800} {\tt lbf}.

\vspace{0.15 in}
\noindent\hspace{0.5 in}
In this first release of ALADDIN, external loads may only be applied to the nodes.
Consequently, formulae for the fixed-end moments due
to gravity loads must be explicitly included in the data file.
If the loading per unit length is $w$, and $L$ is the span length
of the frame bay, then the fixed end shear force is $wL/2$,
and the fixed end moment is $wL^2/12$.

\vspace{0.15 in}
\noindent\hspace{0.5 in}
Moderate earthquake ground motions are modeled with a psuedo-static
lateral force equal to 10\% of total gravity loads (i.e. 63800 lbf).
The lateral forces are distributed over the height of the frame as
shown in Figure \ref{fig: five-story-building-load-distribution}.

\vspace{0.15 in}
\noindent\hspace{0.5 in}
We will assume full fixity for the bases of the columns.
Each node of the building frame will be modeled with
two translational and one rotational degree of freedom.
Axial forces in the beam elements are removed by
lumping the horizontal degrees of freedom at each floor level.
Together these assumptions imply 5 displacement and 4 rotational
degrees of freedom per floor level,
leading to $(45 \times 45)$ global stiffness matrix.

\begin{figure}[ht]
\vspace{0.15 in}
\epsfxsize= 5.4truein
\centerline{\epsfbox{my-chapter6-fig14.ps}}
\caption{Distribution of Lateral Loads for Moderate Earthquake Ground Motions}
\label{fig: five-story-building-load-distribution}
\end{figure}

\vspace{0.15 in}\noindent
{\bf Input File :}
The finite element model for the five story moment-resistant building
frame, with external loads, is defined and solved in
the following {\tt [a]} to {\tt [h]} part input file.

\vskip 0.1truein
\begin{footnotesize}
\vspace{0.10 in}
\noindent
{\rule{2.3 in}{0.035 in} START OF INPUT FILE \rule{2.3 in}{0.035 in} }
\begin{verbatim}
/* 
 *  =====================================================================
 *  Analysis of Five Story Steel Moment Resistant Frame 
 * 
 *  Written By: Mark Austin                                 October, 1994
 *  =====================================================================
 */ 

/* [a] : Setup problem specific parameters */

   NDimension         = 2;
   NDofPerNode        = 3;
   MaxNodesPerElement = 2;

   StartMesh();

/* [b] : Generate two-dimensional grid of nodes */

   node = 0;
   for( y = 0 ft; y <= 50 ft; y = y + 10 ft ) {
      for( x = 0 ft; x <= 55 ft; x = x + 20 ft ) {

        /* [b.1] : adjust column spacing for central bay */

           if(x == 40 ft) {
              x = x - 5 ft;
           }

        /* [b.2] : add new node to finite element mesh */

           node = node + 1;
           AddNode(node, [ x, y ] );
      }
   }

/* [c] : Attach column elements to nodes */

   elmtno = 0;
   for (  colno = 1;   colno <= 4;   colno = colno + 1) {
   for (floorno = 1; floorno <= 5; floorno = floorno + 1) {
        elmtno = elmtno + 1;

        end1 = 4*(floorno - 1) + colno;
        end2 = end1 + 4;

        AddElmt( elmtno, [ end1 , end2 ], "mycolumn");
   }
   }

/* [d] : Attach beam elements to nodes */

   for (floorno = 1; floorno <= 5; floorno = floorno + 1) {
   for (  bayno = 1;   bayno <= 3;     bayno = bayno + 1) {

        end1 = 4*floorno + bayno;
        end2 = end1 + 1;

        elmtno = elmtno + 1;
        AddElmt( elmtno, [ end1 , end2 ], "mybeam");
   }
   }

/* [e] : Define section and material properties */

   ElementAttr("mycolumn") { type     = "FRAME_2D";
                             section  = "mysection1";
                             material = "mymaterial";
                           }

   ElementAttr("mybeam") { type     = "FRAME_2D";
                           section  = "mysection2";
                           material = "mymaterial";
                           }

   SectionAttr("mysection1") { Izz       = 1541.9 in^4;
                               Iyy       =  486.3 in^4;
                               depth     =   12.0 in;
                               width     =   12.0 in;
                               area      =   47.4 in^2;
                             }

   SectionAttr("mysection2") { Izz       = 1600.3 in^4;
                               Iyy       =   66.2 in^4;
                               depth     =   21.0 in;
                               width     =   8.25 in;
                               area      =  21.46 in^2;
                             }

   MaterialAttr("mymaterial") { density = 0.1024E-5 lb/in^3;
                                poisson = 0.25;
                                yield   = 36.0   ksi;
                                E       = 29000  ksi;
                              }

/* [f] : Apply full-fixity to columns at foundation level */

   for(nodeno = 1; nodeno <= 4; nodeno = nodeno + 1) {
       FixNode( nodeno, [ 1, 1, 1 ]);
   }

   LinkNode([  5,  6,  7,  8 ], [ 1, 0, 0] );
   LinkNode([  9, 10, 11, 12 ], [ 1, 0, 0] );
   LinkNode([ 13, 14, 15, 16 ], [ 1, 0, 0] );
   LinkNode([ 17, 18, 19, 20 ], [ 1, 0, 0] );
   LinkNode([ 21, 22, 23, 24 ], [ 1, 0, 0] );

/* [g] : Compute equivalent nodal loads for distributed "dead + live" loads plus  */
/*       lateral wind loads                                                       */

   dead_load       = 80 lbf/ft^2;
   floor_live_load = 40 lbf/ft^2;
   roof_live_load  = 20 lbf/ft^2;
   frame_spacing   = 20 ft;

   for (floorno = 1; floorno <= 5; floorno = floorno + 1) {

   /* [g.1] : compute floor-level (and roof-level) uniform loads */

      live_load = floor_live_load;
      if( floorno == 5) {
          live_load = roof_live_load;
      }
      uniform_load = (dead_load + live_load)*(frame_spacing);

      for (colno = 1;   colno <= 4;     colno = colno + 1) {

         Fx = 0.0 lbf; Fy = 0.0 lbf; Mz = 0.0  lb*in;

      /* [g.2] : compute fixed end shear force for dead/live loads */

         if( colno == 1 || colno == 4) {
             Fy = -(uniform_load)*(20 ft)/2;
         }

         if( colno == 2 || colno == 3) {
             Fy = -(uniform_load)*(35 ft)/2;
         }

      /* [g.3] : compute fixed end moments for dead/live loads */

         if( colno == 1 ) {
             Mz = -(uniform_load)*(20 ft)*(20 ft)/12;
         }

         if( colno == 2 ) {
             Mz =  (uniform_load)*((20 ft)^2 - (15 ft)^2)/12;
         }

         if( colno == 3 ) {
             Mz = -(uniform_load)*((20 ft)^2 - (15 ft)^2)/12;
         }

         if( colno == 4 ) {
             Mz =  (uniform_load)*(20 ft)*(20 ft)/12;
         }

      /* [g.4] : compute horizontal force due to lateral loads */

         if(colno == 1 ) {
            Fx = 63800*(floorno/15)*(1 lbf);
         }

         nodeno = 4*floorno + colno;
         NodeLoad( nodeno, [ Fx, Fy, Mz ]); 
      }
   }

/* [h] : Compile and Print Finite Element Mesh */

   EndMesh();
   PrintMesh();

/* [i] : Compute "stiffness" and "external load" matrices */

   eload = ExternalLoad();
   stiff = Stiff();

   displ = Solve(stiff, eload);

   SetUnitsType("US");
   PrintDispl(displ);
   PrintStress(displ); 
   quit;
\end{verbatim}
\rule{6.25 in}{0.035 in}
\end{footnotesize}

\vspace{0.15 in}\noindent
Points to note are:

\vspace{0.10 in}
\begin{description}
\item{[1]}
In part {\tt [a]} we specify that this will be a two-dimensional analysis.
The maximum number of degrees of freedom per node will be three,
and the maximum number of nodes per element will be two.
The parameters {\tt NDimension}, {\tt NDofPerNode}, and
{\tt MaxNodesPerElement} are used by ALADDIN to assess memory
requirements for the problem storage and solution.

\item{[2]}
We use a nested {\tt for()} loop and a single {\tt if()} statement
to generate the planar layout of 24 finite element nodes.
Before the boundary conditions are applied, the structure
has 72 degrees of freedom.
In Section {\tt [f]} we apply full-fixity to each column at
the foundation level -- this reduces degrees
of freedom from 72 to 60.

\end{description}

\vspace{0.15 in}\noindent
{\bf Abbreviated Output File :}
The output file contains summaries of the mass and stiffness matrices for the shear building,
and abbreviated details of the external loading, {\tt "myload"},
and the response matrix {\tt "response"}.

\vspace{0.10 truein}
\begin{footnotesize}
\noindent
{\rule{1.7 in}{0.035 in} START OF ABBREVIATED OUTPUT FILE \rule{1.7 in}{0.035 in} }
\begin{verbatim}

==========================================
Title : DESCRIPTION OF FINITE ELEMENT MESH                                 
==========================================

 Problem_Type:  Static Analysis

=======================
Profile of Problem Size
=======================

   Dimension of Problem        =      2

   Number Nodes                =     24
   Degrees of Freedom per node =      3
   Max No Nodes Per Element    =      2

   Number Elements             =     35
   Number Element Attributes   =      2
   Number Loaded Nodes         =     20
   Number Loaded Elements      =      0

------------------------------------------------------------
Node#      X_coord           Y_coord          Tx    Ty    Rz  
------------------------------------------------------------

    1    0.00000e+00 ft    0.00000E+00 ft    -1    -2    -3 
    2    2.00000e+01 ft    0.00000E+00 ft    -4    -5    -6 
    3    3.50000e+01 ft    0.00000E+00 ft    -7    -8    -9 
    4    5.50000e+01 ft    0.00000E+00 ft   -10   -11   -12 
    5    0.00000e+00 ft    1.00000E+01 ft     1     6     7 

   ...... details of nodal coordinates and modeling dof removed ....

   19    3.50000e+01 ft    4.00000E+01 ft     4    34    35 
   20    5.50000e+01 ft    4.00000E+01 ft     4    36    37 
   21    0.00000e+00 ft    5.00000E+01 ft     5    38    39 
   22    2.00000e+01 ft    5.00000E+01 ft     5    40    41 
   23    3.50000e+01 ft    5.00000E+01 ft     5    42    43 
   24    5.50000e+01 ft    5.00000E+01 ft     5    44    45 

--------------------------------------------------------------------
Element#       Type       node[1]       node[2]    Element_Attr_Name
--------------------------------------------------------------------

       1     FRAME_2D          1          5            mycolumn
       2     FRAME_2D          5          9            mycolumn
       3     FRAME_2D          9         13            mycolumn
       4     FRAME_2D         13         17            mycolumn

      ...... details of element connectivity removed ....

      33     FRAME_2D         21         22              mybeam
      34     FRAME_2D         22         23              mybeam
      35     FRAME_2D         23         24              mybeam

--------------------- 
Element Attribute Data :        
--------------------- 

ELEMENT_ATTR No.   1  : name = "mycolumn" 
                      : section = "mysection1" 
                      : material = "mymaterial" 
                      : type = FRAME_2D
                      : gdof [0] =    1 : gdof[1] =    2 : gdof[2] =    3
                      : Young's Modulus =  E =        2.900e+04 ksi
                      : Yielding Stress = fy =        3.600e+01 ksi
                      : Poisson's ratio = nu =        2.500e-01   
                      : Density         =        1.024e-06 lb/in^3
                      : Inertia Izz     =        1.542e+03 in^4
                      : Area            =        4.740e+01 in^2

ELEMENT_ATTR No.   2  : name = "mybeam" 
                      : section = "mysection2" 
                      : material = "mymaterial" 
                      : type = FRAME_2D
                      : gdof [0] =    1 : gdof[1] =    2 : gdof[2] =    3
                      : Young's Modulus =  E =        2.900e+04 ksi
                      : Yielding Stress = fy =        3.600e+01 ksi
                      : Poisson's ratio = nu =        2.500e-01   
                      : Density         =        1.024e-06 lb/in^3
                      : Inertia Izz     =        1.600e+03 in^4
                      : Area            =        2.146e+01 in^2
                        
EXTERNAL NODAL LOADINGS 
------------------------------------------------
Node#      Fx (lbf)      Fy (lbf)    Mz (lbf.in)
------------------------------------------------
    5       4253.33     -24000.00    -960000.00
    6          0.00     -42000.00     420000.00
    7          0.00     -42000.00    -420000.00
    8          0.00     -24000.00     960000.00
    9       8506.67     -24000.00    -960000.00
   10          0.00     -42000.00     420000.00
   11          0.00     -42000.00    -420000.00
   12          0.00     -24000.00     960000.00
   13      12760.00     -24000.00    -960000.00
   14          0.00     -42000.00     420000.00
   15          0.00     -42000.00    -420000.00
   16          0.00     -24000.00     960000.00
   17      17013.33     -24000.00    -960000.00
   18          0.00     -42000.00     420000.00
   19          0.00     -42000.00    -420000.00
   20          0.00     -24000.00     960000.00
   21      21266.67     -20000.00    -800000.00
   22          0.00     -35000.00     350000.00
   23          0.00     -35000.00    -350000.00
   24          0.00     -20000.00     800000.00

============= End of Finite Element Mesh Description ==============

------------------------------------------------------------
 Node                           Displacement
  No             displ-x           displ-y             rot-z
------------------------------------------------------------
 units               in                in               rad 
   1         0.00000e+00       0.00000e+00       0.00000e+00
   2         0.00000e+00       0.00000e+00       0.00000e+00
   3         0.00000e+00       0.00000e+00       0.00000e+00
   4         0.00000e+00       0.00000e+00       0.00000e+00
   5         1.02973e-01      -7.40873e-03      -1.23470e-03
   6         1.02973e-01      -1.64731e-02      -6.24404e-04
   7         1.02973e-01      -1.90281e-02      -8.22556e-04
   8         1.02973e-01      -1.27863e-02      -7.58887e-04
   9         2.55749e-01      -1.34750e-02      -1.19639e-03

   ....... details of displacements removed .....

  21         5.62354e-01      -2.29345e-02      -6.66741e-04
  22         5.62354e-01      -4.88997e-02      -8.48912e-05
  23         5.62354e-01      -5.50729e-02      -3.30685e-04
  24         5.62354e-01      -3.63403e-02       6.32278e-05

MEMBER FORCES 
------------------------------------------------------------------------------

Elmt No   1 : 
Coords (X,Y) = (     0.000 in,     60.000 in)
exx =  -6.17394e-05 , curva =   -0.00040508 , gamma =  -3.10985e-04

 Fx1 =   8.48669e+04 lbf  Fy1 =   8.97143e+03 lbf  Mz1 =   9.98366e+05 lbf.in
 Fx2 =  -8.48669e+04 lbf  Fy2 =  -8.97143e+03 lbf  Mz2 =   7.82053e+04 lbf.in

 Axial Force : x-direction =  -8.48669e+04 lbf
 Shear Force : y-direction =   8.97143e+03 lbf

Elmt No   6 : 
Coords (X,Y) = (   240.000 in,     60.000 in)
exx =  -1.37276e-04 , curva =   -0.00020486 , gamma =  -7.05134e-04

 Fx1 =   1.88700e+05 lbf  Fy1 =   2.03420e+04 lbf  Mz1 =   1.45319e+06 lbf.in
 Fx2 =  -1.88700e+05 lbf  Fy2 =  -2.03420e+04 lbf  Mz2 =   9.87850e+05 lbf.in

 Axial Force : x-direction =  -1.88700e+05 lbf
 Shear Force : y-direction =   2.03420e+04 lbf

Elmt No  11 : 
Coords (X,Y) = (   420.000 in,     60.000 in)
exx =  -1.58567e-04 , curva =   -0.00026987 , gamma =  -5.77161e-04

 Fx1 =   2.17967e+05 lbf  Fy1 =   1.66502e+04 lbf  Mz1 =   1.30552e+06 lbf.in
 Fx2 =  -2.17967e+05 lbf  Fy2 =  -1.66502e+04 lbf  Mz2 =   6.92505e+05 lbf.in

 Axial Force : x-direction =  -2.17967e+05 lbf
 Shear Force : y-direction =   1.66502e+04 lbf

Elmt No  16 : 
Coords (X,Y) = (   660.000 in,     60.000 in)
exx =  -1.06552e-04 , curva =   -0.00024898 , gamma =  -6.18281e-04

 Fx1 =   1.46467e+05 lbf  Fy1 =   1.78364e+04 lbf  Mz1 =   1.35297e+06 lbf.in
 Fx2 =  -1.46467e+05 lbf  Fy2 =  -1.78364e+04 lbf  Mz2 =   7.87403e+05 lbf.in

 Axial Force : x-direction =  -1.46467e+05 lbf
 Shear Force : y-direction =   1.78364e+04 lbf
\end{verbatim}
\rule{6.25 in}{0.035 in}
\end{footnotesize}

\vspace{0.15 in}\noindent
{\bf Remark 1 :} We have printed the internal element forces for the
columns between the foundation and the first floor level. The accuracy of
the analysis can be checked by making sure the sum of column actions 
is balanced by the sum of external loads.
In the horizontal direction we have:

\begin{footnotesize}
\begin{verbatim}
====================================================================
External Forces                           Shear Forces in Columns
====================================================================
Roof           21,266.27 lbf                
Floor 4        17,013.33 lbf              Element  1    8,971.43 lbf
Floor 3        12,760.00 lbf              Element  6   20,342.00 lbf
Floor 2         8,506.67 lbf              Element 11   16,650.20 lbf
Floor 1         4,253.33 lbf              Element 16   17,836.40 lbf
====================================================================
Total          63,800.00 lbf              Total        63,800.03 lbf
====================================================================
\end{verbatim}
\end{footnotesize}

\vspace{0.15 in}\noindent
And in the vertical direction we have:


\begin{footnotesize}
\begin{verbatim}
====================================================================
Gravity Loads                             Axial Forces in Columns
====================================================================
Roof           -110,000 lbf                
Floor 4        -132,000 lbf               Element  1   -84,866.9 lbf
Floor 3        -132,000 lbf               Element  6  -188,700.0 lbf
Floor 2        -132,000 lbf               Element 11  -217,967.0 lbf
Floor 1        -132,000 lbf               Element 16  -146,467.0 lbf
====================================================================
Total          -638,000 lbf               Total       -638,000.9 lbf
====================================================================
\end{verbatim}
\end{footnotesize}

\vspace{0.15 in}\noindent
{\bf Remark 2 :} The five story moment resistant frame
may be modeled as a shear structure by fixing the nodal
rotations and vertical displacements at floors 1 to 5,
and lumping the horizontal displacements at each floor level.
These tasks are accomplished by modifying
Section [f] of the input file to

\begin{footnotesize}
\begin{verbatim}
/* [f] : Apply full-fixity to columns at foundation level */

   for(nodeno = 1; nodeno <= 4; nodeno = nodeno + 1) {
       FixNode( nodeno, [ 1, 1, 1 ]);
   }

   for(nodeno = 5; nodeno <= 24; nodeno = nodeno + 1) {
       FixNode( nodeno, [ 0, 1, 1 ]);
   }

   LinkNode([  5,  6,  7,  8 ], [ 1, 0, 0] );
   LinkNode([  9, 10, 11, 12 ], [ 1, 0, 0] );
   LinkNode([ 13, 14, 15, 16 ], [ 1, 0, 0] );
   LinkNode([ 17, 18, 19, 20 ], [ 1, 0, 0] );
   LinkNode([ 21, 22, 23, 24 ], [ 1, 0, 0] );
\end{verbatim}
\end{footnotesize}

\vspace{0.15 in}\noindent
Overall displacements in the frame will now be
represented with five global degrees of freedom.
Details of the stiffness and load vector are:

\begin{footnotesize}
\begin{verbatim}
MATRIX : "eload"

row/col                  1   
        units                
   1      lbf   4.25333e+03
   2      lbf   8.50667e+03
   3      lbf   1.27600e+04
   4      lbf   1.70133e+04
   5      lbf   2.12667e+04

SKYLINE MATRIX : "stiff"

row/col                  1             2             3             4             5   
        units          N/m           N/m           N/m           N/m           N/m   
   1            4.35045e+08  -2.17523e+08   0.00000e+00   0.00000e+00   0.00000e+00 
   2           -2.17523e+08   4.35045e+08  -2.17523e+08   0.00000e+00   0.00000e+00 
   3            0.00000e+00  -2.17523e+08   4.35045e+08  -2.17523e+08   0.00000e+00 
   4            0.00000e+00   0.00000e+00  -2.17523e+08   4.35045e+08  -2.17523e+08 
   5            0.00000e+00   0.00000e+00   0.00000e+00  -2.17523e+08   2.17523e+08 
\end{verbatim}
\end{footnotesize}

\vspace{0.15 in}\noindent
A summary of output is:

\begin{footnotesize}
\begin{verbatim}
------------------------------------------------------------
 Node                           Displacement
  No             displ-x           displ-y             rot-z
------------------------------------------------------------
 units               in                in               rad 
   1         0.00000e+00       0.00000e+00       0.00000e+00
   2         0.00000e+00       0.00000e+00       0.00000e+00
   3         0.00000e+00       0.00000e+00       0.00000e+00
   4         0.00000e+00       0.00000e+00       0.00000e+00
   5         5.13652e-02       0.00000e+00       0.00000e+00
   6         5.13652e-02       0.00000e+00       0.00000e+00
   7         5.13652e-02       0.00000e+00       0.00000e+00
   8         5.13652e-02       0.00000e+00       0.00000e+00
   9         9.93061e-02       0.00000e+00       0.00000e+00
  10         9.93061e-02       0.00000e+00       0.00000e+00

  ...... details of displacements removed ......

  21         1.88339e-01       0.00000e+00       0.00000e+00
  22         1.88339e-01       0.00000e+00       0.00000e+00
  23         1.88339e-01       0.00000e+00       0.00000e+00
  24         1.88339e-01       0.00000e+00       0.00000e+00


MEMBER FORCES 
------------------------------------------------------------------------------

Elmt No   1 : 

 Axial Force : x-direction =   0.00000e+00 lbf
 Shear Force : y-direction =   1.59500e+04 lbf

Elmt No   6 : 

 Axial Force : x-direction =   0.00000e+00 lbf
 Shear Force : y-direction =   1.59500e+04 lbf

Elmt No  11 : 

 Axial Force : x-direction =   0.00000e+00 lbf
 Shear Force : y-direction =   1.59500e+04 lbf

Elmt No  16 : 

 Axial Force : x-direction =   0.00000e+00 lbf
 Shear Force : y-direction =   1.59500e+04 lbf
\end{verbatim}
\end{footnotesize}

\vspace{0.15 in}\noindent
Fixing nodal rotations increases the overall stiffness of
the structure -- the result is lateral displacements 
in the shear structure which are smaller than for the moment
resistant frame (i.e. 1.8834e-01 inches versus 5.6235e-01 inches).
Again, notice that shear forces across the base of the structure are
balanced by the external loads (i.e 4 times 1.59500e+04 lbf = 6.3800e+04 lbf).

\clearpage
\subsection{Working Stress Design (WSD) of Simplified Bridge}

\vspace{0.15 in}\noindent
{\bf Description of WSD Rule Checking Problem :}
This example illustrates the application of ALADDIN to
AASHTO Working Stress Design (WSD) code checking.
We will conduct a finite element analysis of a one span simply
supported composite bridge with cover plate under the girders,
and then apply the rule checking.
The analysis will be simplified by considering only an internal
girder of the bridge system -- a plan and cross sectional view of
a typical bridge system is shown in Figures \ref{fig: wsd-cross-section}
and \ref{fig: wsd-cross-section}.

\begin{figure}[ht]
\epsfxsize= 5.7truein
\centerline{\epsfbox{my-chapter6-fig27.eps}}
\vspace{0.20 in}
\caption{Plan of Highway Bridge}
\label{fig: wsd-plan-of-bridge}
\end{figure}

\begin{figure}[ht]
\epsfxsize= 5.7truein
\centerline{\epsfbox{my-chapter6-fig22.eps}}
\vspace{0.20 in}
\caption{Typical Cross Section}
\label{fig: wsd-cross-section}
\end{figure}

\vspace{0.15 in}\noindent
The bridge girders are made of rolled beam W33x130 with a $14" \times 3/4"$ steel cover plate.
An elevation view of the bridge and the position of the steel cover plate
is shown in Figure \ref{fig: wsd-elevation-of-beam}.
The material properties are $F_{y}$ = 50 ksi, and $E_{s}$ = 29000 ksi.
The effective cross sectional properties of the composite steel/concrete
girder (with and without the cover plate) are computed with $n$ = $E_{s}/E_{c}$ = 10.
The section properties are shown in Figure \ref{fig: wsd-section-properties}.

\begin{figure}[h]
\epsfxsize= 6.0truein
\centerline{\epsfbox{my-chapter6-fig23.eps}}
\caption{Elevation of Beam}
\label{fig: wsd-elevation-of-beam}
\end{figure}

\begin{figure}[h]
\epsfxsize= 6.0truein
\centerline{\epsfbox{my-chapter6-fig21.eps}}
\caption{Section Properties (n = 10)}
\label{fig: wsd-section-properties}
\end{figure}

\vspace{0.15 in}\noindent
The bridge is subjected to dead and live external loadings.
The design dead load includes 7 inches concrete slab, steel girder, and superimposed load.
The design live load consists of a 72 kips HS-20 truck,
which will be modeled as a single concentrated load moving load along the girder nodes.

\vspace{0.15 in}\noindent
{\bf Input File :} The single-span bridge girder is modeled with 10 two-dimensional
beam/column finite elements.

\vskip 0.1truein
\begin{footnotesize}
\vspace{0.10 in}
\noindent
{\rule{2.3 in}{0.035 in} START OF INPUT FILE \rule{2.3 in}{0.035 in} }
\begin{verbatim}
/* 
 *  ==================================================================
 *  Test input file for bridge design rule checking wsing 2D-beam
 *  element and static analysis      
 * 
 *  Written by:  Wane-Jang Lin                               June 1995
 *  ==================================================================
 */ 

print "*** DEFINE PROBLEM SPECIFIC PARAMETERS \n\n";

NDimension         = 2;
NDofPerNode        = 3;
MaxNodesPerElement = 2;

StartMesh();
div_no = 10;

   print "*** GENERATE GRID OF NODES FOR FE MODEL \n\n";
      
   length = 56.70 ft;
   cov_dis = 10.35 ft;

   dL = length/div_no;
   x = 0 ft;  y = 0 ft;
   for( i=1; x<=length ; i=i+1 ) {
       AddNode(i, [x, y]);
       x = x+dL;
   }

   print "*** ATTACH ELEMENTS TO GRID OF NODES \n\n";

   elmt = 1;
   x = 0 ft;
   for( i=1; i<=div_no ; i=i+1 ) then {
      if( x<cov_dis || (x+dL)>(length-cov_dis) ) {
          AddElmt( elmt, [i, i+1], "girder1");
      } else {
          AddElmt( elmt, [i, i+1], "girder2");
      }
      elmt = elmt+1;
      x = x+dL;
   }

   print "*** DEFINE ELEMENT, SECTION AND MATERIAL PROPERTIES \n\n";

   I1 = 17268.4 in^4;  y1 = 29.40 in;  area1 = 38.3 in^2;
   I2 = 25700.1 in^4;  y2 = 27.31 in;  area1 = 48.8 in^2;

   ElementAttr("girder1")  { type     = "FRAME_2D";
                             section  = "no_cover";
                             material = "STEEL3";
                           }
   ElementAttr("girder2")  { type     = "FRAME_2D";
                             section  = "cover";
                             material = "STEEL3";
                           }

   SectionAttr("no_cover") { Izz     =   I1;
                             area    =   area1;
                             depth   =   33.09 in;
                             width   =   11.51 in;
                           }
   SectionAttr("cover")    { Izz     =   I2;
                             area    =   area2;
                             depth   =   33.09 in;
                             width   =   11.51 in;
                           }

/* 
 *  ======================================
 *  Setup Boundary Conditions [dx, dy, rz]
 *  ======================================
 */ 

   FixNode(1, [1, 1, 0]);
   FixNode(div_no+1, [0, 1, 0]);

/* 
 *  =====================================
 *  Compile and Print Finite Element Mesh 
 *  =====================================
 */ 

   EndMesh();
   PrintMesh();
   SetUnitsType("US");
   stiff = Stiff();
   lu = Decompose(stiff);

/* 
 *  =====================================
 *  Add Dead Nodal Loads [Fx, Fy, Mz]
 *  =====================================
 */ 

   weight = 0.199 kips/ft + 0.868 kips/ft;
   load = weight*dL;
      
   for( i=2; i<=div_no; i=i+1 ) {
        NodeLoad( i, [0 kips, -load, 0 kips*ft]);
   }
   NodeLoad(        1, [0 kips, -load/2, 0 kips*ft]);
   NodeLoad( div_no+1, [0 kips, -load/2, 0 kips*ft]);

   eload  = ExternalLoad();
   displ  = Substitution(lu, eload);
   max_displ_dead = GetDispl([div_no/2+1],displ);
   max_mom_dead   = GetStress([div_no/2],displ);
   max_sh_dead    = GetStress([1],displ);
   cover_mom_dead = GetStress([2],displ);
   PrintMatrix(max_displ_dead,max_mom_dead,max_sh_dead,cover_mom_dead);

/* Get Moment Diagram */

   moment_dead = Zero([ div_no+1 , 1 ]);
   for( i=1 ; i<=div_no ; i=i+1 ) {
        temp = GetStress([i],displ);
        if( i == 1 ) {  moment_dead[1][1] = temp[1][3];  }
        moment_dead[i+1][1] = temp[2][3];
   }
   PrintMatrix(moment_dead);

/* Zero-out dead loadings */

   for( i=2; i<=div_no; i=i+1 ) {
        NodeLoad( i, [0 kips, load, 0 kips*ft]);
   }
   NodeLoad(        1, [0 kips, load/2, 0 kips*ft]);
   NodeLoad( div_no+1, [0 kips, load/2, 0 kips*ft]);

/* 
 *  =====================================
 *  Add Live Truck Load [Fx, Fy, Mz] 
 *  =====================================
 */ 

   step   = div_no+1;
   influ_mid_mom = Zero([ step , 1 ]);
   influ_end_sh  = Zero([ step , 1 ]);
   envelop_mom_live  = Zero([ step , 1 ]);
   HS20   = 72 kips;
   mom_elmt = div_no/2;
   sh_elmt  = 1;

   for( i=1 ; i<=step ; i=i+1 ) {
       NodeLoad( i, [0 kips, -HS20, 0 kips*ft] );
       eload = ExternalLoad();
       displ  = Substitution( lu, eload );
       mid_mom = GetStress( [mom_elmt], displ );
       end_sh  = GetStress( [sh_elmt], displ );
       if( i == 1 ) then {
           mom_rang  = GetStress( [i], displ );
       } else {
           mom_rang  = GetStress( [i-1], displ );
       }

       /* Get moment influence line at the middle of span */

       influ_mid_mom[i][1] = mid_mom[2][3];

       /* Get Shear Influence line at the end support */

       influ_end_sh[i][1]  = end_sh[1][2];

       /* Create envelope of moment influence line at the middle of span */

       if( i == 1 ) then { 
           envelop_mom_live[1][1]  = mom_rang[1][3];
       } else {
          envelop_mom_live[i][1]  = mom_rang[2][3];
       }

       if( i == mom_elmt+1 ) {
           max_displ_live = GetDispl([mom_elmt+1],displ);
           max_mom_live = GetStress([mom_elmt],displ);
       }
       if( i == sh_elmt ) {
           max_sh_live = GetStress([sh_elmt],displ);
           max_sh_live[1][2]  = max_sh_live[1][2] + HS20;
           influ_end_sh[1][1] = HS20;
       }
       if( i == 3 ) {
           cover_mom_live = GetStress([2],displ);
       }

       NodeLoad( i, [0 kips, HS20, 0 kips*ft] );
   }

   PrintMatrix(max_displ_live,max_mom_live,max_sh_live,cover_mom_live);
   PrintMatrix(influ_mid_mom, influ_end_sh, envelop_mom_live);

/* 
 *  ============================
 *  Results for DL + (LL+Impact)
 *  ============================
 */ 

   impact = 1 + 50/(56.7+125);
   print "\nimpact factor = ",impact,"\n";
   max_displ  = max_displ_dead + impact*max_displ_live;
   max_mom    = max_mom_dead + impact*max_mom_live;
   max_sh     = max_sh_dead + impact*max_sh_live;
   cover_mom  = cover_mom_dead + impact*cover_mom_live;

   PrintMatrix(max_displ,max_mom,max_sh,cover_mom);

/* 
 *  =========================================================
 *  WSD Code Checking for Deflections and Stress Requirements  
 *  =========================================================
 */ 

   print "\n\nSTART ASD CODE CHECKING::\n";

/* Deflection Checking */

   if( -impact*max_displ_live[1][2] > (1/800)*length ) then {
       print "\n\tWarning: (LL+I) deflection exceeds 1/800 span\n";
   } else {
       print "\n\tOK : (LL+I) deflection less than 1/800 span\n";
   }

/* Moment Stress Checking */

   my_material = GetMaterial([1]);
   my_section  = GetSection([1]);

   /* max stress without cover plate */
   stress1 = cover_mom[2][3]*y1/I1;

   /* max stress with cover plate (in the middle of span) */
   stress2 = max_mom[2][3]*y2/I2;

   if( stress1 > 0.55*my_material[3][1] ) then{
       print "\n\tWarning : moment stress without cover plate larger than 0.55*Fy\n";
   } else {
       if( stress2 > 0.55*my_material[3][1] ) then{
           print "\n\tWarning : moment stress with cover plate larger than 0.55*Fy\n";
       } else {
           print "\n\tOK : moment stress less than 0.55*Fy\n";
       }
   }

/* Shear Stress Checking */

   shear = max_sh[1][2]/my_section[11][1];
   if( shear > 0.33*my_material[3][1] ) then{
       print "\n\tWarning : shear stress larger than 0.33*Fy\n";
   } else {
       print "\n\tOK : shear stress less than 0.33*Fy\n";
   }

   quit;
\end{verbatim}
\rule{6.25 in}{0.035 in}
\end{footnotesize}

\vspace{0.15 in}\noindent
Points to note are:

\vspace{0.10 in}
\begin{description}
\item{[1]}
In this example, we check the analysis result with AASHTO WSD specification.
The impact factor for live load is based on the formula AASHTO Eq. (3-1)
\begin{displaymath}
I = \frac{50}{L+125} 
\end{displaymath}
in which\\
$I$ = impact fraction (maximum 30 percent);\\
$L$ = length in feet of the portion of the span;
\item{[2]}
The deflection checking is based on AASHTO Art.10.6.2, the deflection due to
service live load plus impact shall not exceed $1/800$ of the span.
\item{[3]}
The allowable stress is $0.55 \times F_{y}$ for tension and
compression member, $0.33 \times F_{y}$ for shear in web.
\end{description}

\vspace{0.15 in}\noindent
{\bf Abbreviated Output File :}
The opening sections of the following output file 
contain details of the 11 node, 10 element mesh.
The left-hand side of the girder is modeled as a pin fixed against translation.
The right-hand side of the girder is supported on a roller.

\vspace{0.15 in}
\noindent\hspace{0.51 in}
The latter components of the output file contain summaries
of the shear and moment envelopes,
together with maximum values of response quantities.

\vspace{0.20 truein}
\begin{footnotesize}
\noindent
{\rule{1.7 in}{0.035 in} START OF ABBREVIATED OUTPUT FILE \rule{1.7 in}{0.035 in} }
\begin{verbatim}

===========================================================================
Title : DESCRIPTION OF FINITE ELEMENT MESH                                 
===========================================================================

 Problem_Type:  Static Analysis

=======================
Profile of Problem Size
=======================

   Dimension of Problem        =      2

   Number Nodes                =     11
   Degrees of Freedom per node =      3
   Max No Nodes Per Element    =      2

   Number Elements             =     10
   Number Element Attributes   =      2
   Number Loaded Nodes         =      0
   Number Loaded Elements      =      0


------------------------------------------------------------------------------
Node#      X_coord           Y_coord          Tx    Ty    Rz  
------------------------------------------------------------------------------

    1    0.0000e+00 ft    0.0000e+00 ft    -1    -2     1 
    2    5.6700e+00 ft    0.0000e+00 ft     2     3     4 
    3    1.1340e+01 ft    0.0000e+00 ft     5     6     7 
    4    1.7010e+01 ft    0.0000e+00 ft     8     9    10 
    5    2.2680e+01 ft    0.0000e+00 ft    11    12    13 
    6    2.8350e+01 ft    0.0000e+00 ft    14    15    16 
    7    3.4020e+01 ft    0.0000e+00 ft    17    18    19 
    8    3.9690e+01 ft    0.0000e+00 ft    20    21    22 
    9    4.5360e+01 ft    0.0000e+00 ft    23    24    25 
   10    5.1030e+01 ft    0.0000e+00 ft    26    27    28 
   11    5.6700e+01 ft    0.0000e+00 ft    29    -3    30 


--------------------------------------------------------------------
Element#     Type         node[1]   node[2]      Element_Attr_Name
----------------------------------------------------------------------

       1   FRAME_2D             1         2      girder1
       2   FRAME_2D             2         3      girder1
       3   FRAME_2D             3         4      girder2
       4   FRAME_2D             4         5      girder2
       5   FRAME_2D             5         6      girder2
       6   FRAME_2D             6         7      girder2
       7   FRAME_2D             7         8      girder2
       8   FRAME_2D             8         9      girder2
       9   FRAME_2D             9        10      girder1
      10   FRAME_2D            10        11      girder1

--------------------- 
Element Attribute Data :        
--------------------- 


ELEMENT_ATTR No.   1  : name = "girder1" 
                      : section = "no_cover" 
                      : material = "steel" 
                      : type = FRAME_2D
                      : gdof [0] =    1 : gdof[1] =    2 : gdof[2] =    3
                      : Young's Modulus =  E =        2.900e+04 ksi
                      : Yielding Stress = fy =        5.000e+01 ksi
                      : Poisson's ratio = nu =        3.000e-01   
                      : Density         =        0.000e+00 (null)
                      : Inertia Izz     =        1.727e+04 in^4
                      : Area            =        3.809e+02 in^2


ELEMENT_ATTR No.   2  : name = "girder2" 
                      : section = "cover" 
                      : material = "steel" 
                      : type = FRAME_2D
                      : gdof [0] =    1 : gdof[1] =    2 : gdof[2] =    3
                      : Young's Modulus =  E =        2.900e+04 ksi
                      : Yielding Stress = fy =        5.000e+01 ksi
                      : Poisson's ratio = nu =        3.000e-01   
                      : Density         =        0.000e+00 (null)
                      : Inertia Izz     =        2.570e+04 in^4
                      : Area            =        3.809e+02 in^2

============= End of Finite Element Mesh Description ==============


MATRIX : "max_displ_dead"

row/col                  1            2            3   
        units           in           in          rad   
   1            0.00000e+00 -3.44203e-01  5.69206e-18

MATRIX : "max_mom_dead"

row/col                  1            2            3   
        units          lbf          lbf       lbf.in   
   1           -0.00000e+00  3.02494e+03 -4.93961e+06
   2            0.00000e+00 -3.02494e+03  5.14543e+06

MATRIX : "max_sh_dead"

row/col                  1            2            3   
        units          lbf          lbf       lbf.in   
   1           -0.00000e+00  2.72245e+04 -8.24290e-09
   2            0.00000e+00 -2.72245e+04  1.85236e+06

MATRIX : "cover_mom_dead"

row/col                  1            2            3   
        units          lbf          lbf       lbf.in   
   1           -0.00000e+00  2.11746e+04 -1.85236e+06
   2            0.00000e+00 -2.11746e+04  3.29308e+06

MATRIX : "moment_dead"

row/col                  1   
        units                
   1   lbf.in  -8.24290e-09
   2   lbf.in   1.85236e+06
   3   lbf.in   3.29308e+06
   4   lbf.in   4.32216e+06
   5   lbf.in   4.93961e+06
   6   lbf.in   5.14543e+06
   7   lbf.in   4.93961e+06
   8   lbf.in   4.32216e+06
   9   lbf.in   3.29308e+06
  10   lbf.in   1.85236e+06
  11   lbf.in   8.24290e-09

MATRIX : "max_displ_live"

row/col                  1            2            3   
        units           in           in          rad   
   1            0.00000e+00 -6.53755e-01  1.06252e-17

MATRIX : "max_mom_live"

row/col                  1            2            3   
        units          lbf          lbf       lbf.in   
   1           -0.00000e+00  3.60000e+04 -9.79776e+06
   2            0.00000e+00 -3.60000e+04  1.22472e+07

MATRIX : "max_sh_live"

row/col                  1            2            3   
        units          lbf          lbf       lbf.in   
   1           -0.00000e+00  7.20000e+04  0.00000e+00
   2            0.00000e+00 -0.00000e+00  0.00000e+00

MATRIX : "cover_mom_live"

row/col                  1            2            3   
        units          lbf          lbf       lbf.in   
   1           -0.00000e+00  5.76000e+04 -3.91910e+06
   2            0.00000e+00 -5.76000e+04  7.83821e+06

MATRIX : "influ_mid_mom"

row/col                  1   
        units                
   1   lbf.in   0.00000e+00
   2   lbf.in   2.44944e+06
   3   lbf.in   4.89888e+06
   4   lbf.in   7.34832e+06
   5   lbf.in   9.79776e+06
   6   lbf.in   1.22472e+07
   7   lbf.in   9.79776e+06
   8   lbf.in   7.34832e+06
   9   lbf.in   4.89888e+06
  10   lbf.in   2.44944e+06
  11   lbf.in   0.00000e+00

MATRIX : "influ_end_sh"

row/col                  1   
        units                
   1      lbf   7.20000e+04
   2      lbf   6.48000e+04
   3      lbf   5.76000e+04
   4      lbf   5.04000e+04
   5      lbf   4.32000e+04
   6      lbf   3.60000e+04
   7      lbf   2.88000e+04
   8      lbf   2.16000e+04
   9      lbf   1.44000e+04
  10      lbf   7.20000e+03
  11      lbf   0.00000e+00

MATRIX : "envelop_mom_live"

row/col                  1   
        units                
   1   lbf.in   0.00000e+00
   2   lbf.in   4.40899e+06
   3   lbf.in   7.83821e+06
   4   lbf.in   1.02876e+07
   5   lbf.in   1.17573e+07
   6   lbf.in   1.22472e+07
   7   lbf.in   1.17573e+07
   8   lbf.in   1.02876e+07
   9   lbf.in   7.83821e+06
  10   lbf.in   4.40899e+06
  11   lbf.in   0.00000e+00

impact factor =      1.275 

MATRIX : "max_displ"

row/col                  1            2            3   
        units           in           in          rad   
   1            0.00000e+00 -1.17786e+00  1.92411e-17

MATRIX : "max_mom"

row/col                  1            2            3   
        units          lbf          lbf       lbf.in   
   1           -0.00000e+00  4.89314e+04 -1.74335e+07
   2            0.00000e+00 -4.89314e+04  2.07628e+07

MATRIX : "max_sh"

row/col                  1            2            3   
        units          lbf          lbf       lbf.in   
   1           -0.00000e+00  1.19037e+05 -8.24290e-09
   2            0.00000e+00 -2.72245e+04  1.85236e+06

MATRIX : "cover_mom"

row/col                  1            2            3   
        units          lbf          lbf       lbf.in   
   1           -0.00000e+00  9.46249e+04 -6.84991e+06
   2            0.00000e+00 -9.46249e+04  1.32882e+07


START ASD CODE CHECKING::

   OK : (LL+I) deflection less than 1/800 span

   OK : moment stress less than 0.55*Fy
 
   OK : shear stress less than 0.33*Fy
\end{verbatim}
\rule{6.25 in}{0.035 in}
\end{footnotesize}

\vspace{0.15 in}\noindent
A summary of the bridge response is contained
in Figures \ref{fig: mom_diag} to \ref{fig: influ_s}.
The following points are noted:

\begin{description}
\item{[1]}
Since this is a simple-supported bridge, the maximum displacement and
maximum bending moment will occur at the middle of the span.
The maximum shear force will occur at the end support.

\item{[2]}
The final results of moment, shear and displacement are calculated
according to AASHTO ASD request: Total = DL + impact * LL.

\item{[3]}
The stress due to bending is given by:

\begin{equation}
\hbox{Moment stress} = {{M \cdot y} \over I}.
\label{eq: working-check-part1}
\end{equation}

Because there are two different section properties, we have to calculate
not only the maximum moment stress in the middle of the span (stress2),
but also the moment stress of where the section changed (stress1).

\begin{equation}
\hbox{Shear stress}  = {V \over {tw \cdot d}}.
\label{eq: working-check-part2}
\end{equation}

We assume that the shear force is carried by the girder web alone,
and therefore, we only check for maximum shear at the end support.

\item{[4]}
The influence line for the bending moment at the middle
of the span -- details are shown in Figure: \ref{fig: influ_m} -- is obtained
by iteratively positioning one truck load at a finite element node,
then solving for the reaction forces.
A similar procedure is employed to compute
the influence line of shear force at the end support -- 
see Figure \ref{fig: influ_s}.

\item{[5]}
Figure \ref{fig: mom_diag}) shows the distribution of
bending moments due to truck loadings (we note in passing that
the bending momend diagram corresponds to
an envelope of the moment influence lines of the truck load).

\item{[6]}
The follow array elements are used in the generation of
program output:

\begin{footnotesize}
\begin{verbatim}
    max_displ[1][2] = the maximum displacement of the beam.
    max_mom[2][3]   = the maximum moment.
    max_sh[1][2]    = the maximum shear force.
    cover_mom[2][3] = the moment at where the bridge section changed.
\end{verbatim}
\end{footnotesize}
\end{description}

\begin{figure}[h]
\epsfxsize= 5.4truein
\centerline{\epsfbox{my-chapter6-fig26.eps}}
\caption{Moment Diagram of Dead Load and Truck Load}
\label{fig: mom_diag}
\end{figure}

\clearpage
\begin{figure}[th]
\vspace{0.10 in}
\epsfxsize= 4.2truein
\centerline{\epsfbox{my-chapter6-fig24.eps}}
\caption{Influence Line of Moment at Mid-Span}
\label{fig: influ_m}
\end{figure}

\begin{figure}[h]
\vspace{0.30 in}
\epsfxsize= 4.2truein
\centerline{\epsfbox{my-chapter6-fig25.eps}}
\caption{Influence Line of Shear at End Support}
\label{fig: influ_s}
\end{figure}

\clearpage
\subsection{Three-Dimensional Analysis of Highway Bridge}

\vspace{0.15 in}\noindent
{\bf Description of Bridge Problem :}
In this section we will conduct a two-part analysis of a three-dimensional
two-span highway bridge structure.

\begin{figure}[ht]
\epsfxsize= 6.0truein
\centerline{\epsfbox{my-chapter6-fig10.ps}}
\vspace{0.20 in}
\caption{Highway Bridge : Plan and Front Elevation of Two-Span Continuous Highway Bridge}
\label{fig: bridge-plan-and-elevation}
\end{figure}

\begin{figure}[ht]
\epsfxsize= 6.0truein
\centerline{\epsfbox{my-chapter6-fig11.ps}}
\vspace{0.20 in}
\caption{Highway Bridge : Cross Section of Two-Span Continuous Highway Bridge}
\label{fig: bridge-cross-section}
\end{figure}

\vspace{0.15 in}\noindent
A plan and front elevation view of the highway
bridge is shown in Figure \ref{fig: bridge-plan-and-elevation}.
The bridge has two spans, each 100 ft long.
The width of the bridge is 16 ft - 8 inches.
A cross section view of the bridge is shown in
Figure \ref{fig: bridge-cross-section}.

\begin{figure}[t]
\epsfxsize= 6.0truein
\centerline{\epsfbox{my-chapter6-fig12.ps}}
\vspace{0.20 in}
\caption{Highway Bridge : Plan of Finite Element Mesh for Two-Span Continuous Highway Bridge}
\label{fig: bridge-plan-of-mesh}
\end{figure}

\begin{figure}[t]
\epsfxsize= 6.0truein
\centerline{\epsfbox{my-chapter6-fig13.ps}}
\vspace{0.20 in}
\caption{Highway Bridge : Cross Section of Finite Element Mesh for Two-Span Continuous Highway Bridge}
\label{fig: bridge-cross-section-of-mesh}
\end{figure}

\vspace{0.15 in}
\noindent\hspace{0.5 in}
The bridge is constructed from one steel and one concrete material type.
The structural steel has $F_y$ = 50 ksi,
Poisson's ratio $v$ = 0.3, and E = 29,000 ksi.
For the concrete slab, $f_c'$ = 4000 psi,
Poisson's ratio $v$ = 0.3, $E_c$ = 3625 ksi, 
and unit weight = 150 pcf.
The W36x170 steel section has the following properties;
W36x170 : d = 36.17 in,
$b_f$ = 12.03 in, $t_f$ = 1.1 in, $t_w$ = 0.68 in,
and weight = 170 lbf/ft.
The concrete slab has thickness 7 in.

\vspace{0.15 in}
\noindent\hspace{0.5 in}
The highway bridge will be analyzed for two loading conditions.
First, we compute the deflections of the bridge due to gravity loads alone.
The concrete slab weighs $150 pcf \cdot 7 in$ = $87.5 lbf/ft^2$.
The girder weight is $170 lbf/ft$.
In part two of the bridge analysis, a 1000 kip
concentrated live load moves along one of the outer bridge girders.
We compute and plot the influence line for the moving load.

\vspace{0.15 in}
\noindent\hspace{0.5 in}
The boundary conditions for the left-hand sidee of the bridge are a hinged support.
The right-hand side of the bridge is supported on a roller.
The finite element model has 399 nodes and 440 shell elements.
After the boundary conditions are applied, the model has 2374 d.o.f.
The bridge is subject to a 1000 kip concentrated moving live load.


\vspace{0.15 in}\noindent
{\bf Input File :} The following input file defines the bridge
geometry, boundary conditions, finite element, section and material types,
and the external loads due to the dead weight of the bridge.

\vskip 0.1truein
\begin{footnotesize}
\vspace{0.10 in}
\noindent
{\rule{2.3 in}{0.035 in} START OF INPUT FILE \rule{2.3 in}{0.035 in} }
\begin{verbatim}
/* 
 *  =====================================================================
 *  Analysis of Two Span Continuous Composite Beam (100-100ft) assuming
 *  neutral axis of concrete deck coincide with top girder flange 
 * 
 *  Written By: Wane-Jang Lin                              December, 1994
 *  =====================================================================
 */ 

/* [a] : Setup problem specific parameters */

   print "*** DEFINE PROBLEM SPECIFIC PARAMETERS \n\n";

   NDimension         = 3;
   NDofPerNode        = 6;
   MaxNodesPerElement = 4;

   StartMesh();

   print "*** GENERATE GRID OF NODES FOR FE MODEL \n\n";

   girder   = 2;
   span     = 100 ft;
   spacing  = 8 ft + 4 in;
   slab     = 7 in;
   height   = 36.17 in - 1.1 in;
   flange   = 12.03 in;

   div_L    = 10;
   div_S    = 4;
   delta_L  = span/div_L;
   delta_S  = spacing/div_S;
   delta_f  = flange/2;
   delta_h  = height/2;

   section_no = 2*div_L + 1;
   nodes_per_girder = 6;
   nodes_per_section = nodes_per_girder*girder + (2*div_S+1) - girder;

   node = 0;
   x = 0 in;
   for(i=1 ; i<=section_no ; i=i+1 ) {
       x  =  delta_L*(i-1);
       for( j=1 ; j<=girder ; j=j+1 ) {
            y    = -spacing/2 + spacing*(j-1);
            AddNode( node+1, [ x, (y-delta_f),  delta_h ] );
            AddNode( node+2, [ x,          y ,  delta_h ] );
            AddNode( node+3, [ x, (y+delta_f),  delta_h ] );
            AddNode( node+4, [ x, (y-delta_f), -delta_h ] );
            AddNode( node+5, [ x,          y , -delta_h ] );
            AddNode( node+6, [ x, (y+delta_f), -delta_h ] );
            node = node + nodes_per_girder;
       }

       yy = spacing/2;
       for( j=1 ; j<=(2*div_S+1) ; j=j+1 ) {
            y = -spacing + delta_S*(j-1);
            if( (y!=yy) && (y!=-yy) ) {
                 node = node + 1;
                 AddNode( node, [ x, y, delta_h ] );
            }
       }
   }

   print "*** ATTACH ELEMENTS TO GRID OF NODES \n\n";

   elmtno = 0;
   a = 0;
   b = 0 + nodes_per_section;

   for( i=1 ; i<section_no ; i=i+1 ) {
   for( j=1 ; j<=girder ; j=j+1 ) {
       AddElmt( elmtno+1, [ a+1, b+1, b+2, a+2 ],"girder_flange_attr" );
       AddElmt( elmtno+2, [ a+2, b+2, b+3, a+3 ],"girder_flange_attr" );
       AddElmt( elmtno+3, [ a+2, b+2, b+5, a+5 ],"girder_web_attr" );
       AddElmt( elmtno+4, [ a+4, b+4, b+5, a+5 ],"girder_flange_attr" );
       AddElmt( elmtno+5, [ a+5, b+5, b+6, a+6 ],"girder_flange_attr" );
       elmtno = elmtno + nodes_per_girder - 1;
       a = a + nodes_per_girder;
       b = b + nodes_per_girder;
   }
   a = a + 1;
   b = b + 1;
   for( j=1 ; j<=girder ; j=j+1 ) {
       AddElmt( elmtno+1, [   a,   b, b+1, a+1 ], "deck_attr" );
       c = b - (girder+1-j)*nodes_per_girder - 3*(j-1);
       d = a - (girder+1-j)*nodes_per_girder - 3*(j-1);
       AddElmt( elmtno+2, [ a+1, b+1,   c,   d ], "deck_attr" );
       a = d;
       b = c;
       AddElmt( elmtno+3, [   a,   b, b+1, a+1 ], "deck_attr" );
       AddElmt( elmtno+4, [ a+1, b+1, b+2, a+2 ], "deck_attr" );
       a = a + 2;
       b = b + 2;
       c = c + (girder+1-j)*nodes_per_girder + (3*j-2) + 1;
       d = d + (girder+1-j)*nodes_per_girder + (3*j-2) + 1;
       AddElmt( elmtno+5, [   a,   b,   c,   d ], "deck_attr" );
       AddElmt( elmtno+6, [   d,   c, c+1, d+1 ], "deck_attr" );
       elmtno = elmtno + 6;
       a = d + 1;
       b = c + 1;
   }
   }

   print "*** DEFINE ELEMENT, SECTION AND MATERIAL PROPERTIES \n\n";

   ElementAttr("girder_flange_attr") { type  = "SHELL_4NQ";
                                    section  = "girder_flange";
                                    material = "STEEL3";
                                  }

   ElementAttr("girder_web_attr") { type     = "SHELL_4NQ";
                                    section  = "girder_web";
                                    material = "STEEL3";
                                  }

   ElementAttr("deck_attr") {    type = "SHELL_4NQ";
                              section = "deck";
                             material = "concrete";
                            }

   SectionAttr("girder_flange")      { thickness = 1.100 in; }
   SectionAttr("girder_web")         { thickness = 0.680 in; }
   SectionAttr("deck")        { thickness = 7 in; }
   MaterialAttr("concrete")   { poisson   = 0.3;
                                yield     = 0.85*(4000 psi);
                                E         = (29000 ksi)/8;
                              }

   print "*** SET UP BOUNDARY CONDITIONS \n\n";

   bc_hs = [ 1, 1, 1, 1, 0, 0 ]; /* hinged support  */
   bc_rs = [ 0, 1, 1, 1, 0, 0 ]; /* roller support  */

   for( i=1 ; i<=girder ; i=i+1 ) {
        node = nodes_per_girder*(i-1) + 5;
        FixNode(node, bc_hs);
        node = node + div_L*nodes_per_section;
        FixNode(node, bc_rs);
        node = node + div_L*nodes_per_section;
        FixNode(node, bc_rs);
   }

/* [b] : Apply Point Nodal Loads */

   print "*** SET UP LOADS \n\n";

   slab_load     = (150 lbf/ft^3)*slab;
   girder_weight = 170 lbf/ft;

   Fx = 0 lbf;    Fy = 0 lbf;
   Mx = 0 lbf*in; My = 0 lbf*in; Mz = 0 lbf*in;

   /* [b.1] : load for corner nodes of deck  */

      Fz = -slab_load*delta_L/2*delta_S/2;
      nodal_load = [Fx, Fy, Fz, Mx, My, Mz];
      node = nodes_per_girder*girder + 1;
      NodeLoad(node, nodal_load);
      node = node + nodes_per_section*(section_no-1);
      NodeLoad(node, nodal_load);
      node = node + 2*div_S - girder;
      NodeLoad(node, nodal_load);
      node = node - nodes_per_section*(section_no-1);
      NodeLoad(node, nodal_load);

   /* [b.2] load for edge nodes along x-direction  */

      Fz = -slab_load*delta_L*delta_S/2;
      nodal_load = [Fx, Fy, Fz, Mx, My, Mz];
      node = nodes_per_section + nodes_per_girder*girder + 1;
      for( i=2 ; i<section_no ; i=i+1 ) {
           NodeLoad(node, nodal_load);
           NodeLoad((node+2*div_S-girder), nodal_load);
           node = node + nodes_per_section;
      }

   /* [b.3] load for edge nodes along y-direction  */

   /* node 16 */
      Fz = -slab_load*delta_L/2*delta_S;
      nodal_load = [Fx, Fy, Fz, Mx, My, Mz];
      node = nodes_per_girder*girder + div_S;
      NodeLoad(node, nodal_load);
      node = node + nodes_per_section*(section_no-1);
      NodeLoad(node, nodal_load);

   /* node 14, 15, 17, 18 */
      Fz = -slab_load*delta_L/2*(delta_S-delta_f/2);
      nodal_load = [Fx, Fy, Fz, Mx, My, Mz];
      node = nodes_per_girder*girder + 1;
      for( i=1 ; i<=2 ; i=i+1 ) {
           NodeLoad(node+1, nodal_load);
           NodeLoad(node+2, nodal_load);
           NodeLoad(node+4, nodal_load);
           NodeLoad(node+5, nodal_load);
           node = node + nodes_per_section*(section_no-1);
      }

   /* node 1, 3, 7, 9 */
      Fz = -slab_load*delta_L/2*delta_S/2;
      nodal_load = [Fx, Fy, Fz, Mx, My, Mz];
      node = 0;
      for( i=1 ; i<=2 ; i=i+1 ) {
           NodeLoad(node+1, nodal_load);
           NodeLoad(node+3, nodal_load);
           NodeLoad(node+7, nodal_load);
           NodeLoad(node+9, nodal_load);
           node = node + nodes_per_section*(section_no-1);
      }

   /* node 2, 8 */
      Fz = -slab_load*delta_L/2*delta_f;
      nodal_load = [Fx, Fy, Fz, Mx, My, Mz];
      node = 0;
      for( i=1 ; i<=2 ; i=i+1 ) {
           NodeLoad(node+2, nodal_load);
           NodeLoad(node+8, nodal_load);
           node = node + nodes_per_section*(section_no-1);
      }

   /* [b.4] load for middle nodes */

   /* node 35... */
      Fz = -slab_load*delta_L*delta_S;
      nodal_load = [Fx, Fy, Fz, Mx, My, Mz];
      node = nodes_per_girder*girder + div_S;
      for( i=2 ; i<section_no ; i=i+1 ) {
           node = node + nodes_per_section;
           NodeLoad(node, nodal_load);
      }

   /* node 33, 34, 36, 37... */
      Fz = -slab_load*delta_L*(delta_S-delta_f/2);
      nodal_load = [Fx, Fy, Fz, Mx, My, Mz];
      node = nodes_per_girder*girder + 1;
      for( i=2 ; i<section_no ; i=i+1 ) {
           node = node + nodes_per_section;
           NodeLoad(node+1, nodal_load);
           NodeLoad(node+2, nodal_load);
           NodeLoad(node+4, nodal_load);
           NodeLoad(node+5, nodal_load);
      }

   /* node 20, 22, 26, 28... */
      Fz = -slab_load*delta_L*delta_S/2;
      nodal_load = [Fx, Fy, Fz, Mx, My, Mz];
      node = 0;
      for( i=2 ; i<section_no ; i=i+1 ) {
           node = node + nodes_per_section;
           NodeLoad(node+1, nodal_load);
           NodeLoad(node+3, nodal_load);
           NodeLoad(node+7, nodal_load);
           NodeLoad(node+9, nodal_load);
      }

   /* node 21, 27... */
      Fz = -slab_load*delta_L*delta_f;
      nodal_load = [Fx, Fy, Fz, Mx, My, Mz];
      node = 0;
      for( i=2 ; i<section_no ; i=i+1 ) {
           node = node + nodes_per_section;
           NodeLoad(node+2, nodal_load);
           NodeLoad(node+8, nodal_load);
      }

   /* [b.5] load for girder weight */

   /* end node 2, 8 */
      Fz = -girder_weight*delta_L/2;
      nodal_load = [Fx, Fy, Fz, Mx, My, Mz];
      node = 0;
      for( i=1 ; i<=2 ; i=i+1 ) {
           NodeLoad(node+2, nodal_load);
           NodeLoad(node+8, nodal_load);
           node = node + nodes_per_section*(section_no-1);
      }

   /* middle node 21, 27... */
      Fz = -girder_weight*delta_L;
      nodal_load = [Fx, Fy, Fz, Mx, My, Mz];
      node = 0;
      for( i=2 ; i<section_no ; i=i+1 ) {
           node = node + nodes_per_section;
           NodeLoad(node+2, nodal_load);
           NodeLoad(node+8, nodal_load);
      }

/* [c] : Compile and Print Finite Element Mesh */

   EndMesh();
   PrintMesh();

/* [d] : Compute Stiffness Matrix and External Load Vector */

   print "\n";
   print "*** COMPUTE AND PRINT STIFFNESS AND EXTERNAL LOAD MATRICES \n\n";

   SetUnitsType("US");

   eload = ExternalLoad();
   stiff = Stiff();

/* [e] : Compute and print static displacements */

   print "\n*** STATIC ANALYSIS PROBLEM \n\n";

   displ  = Solve( stiff, eload);
   PrintDispl(displ);
   PrintStress(displ);

   quit;
\end{verbatim}
\rule{6.25 in}{0.035 in}
\end{footnotesize}

\vspace{0.25 in}\noindent
{\bf Abbreviated Output File :} Several thousand lines of output are generated
by ALADDIN and the abovementioned input file defined.
Hence, we present an abbreviated summary of the output together with
contour and three-dimensional views of the bridge deck deflections.

\vspace{0.25 truein}
\begin{footnotesize}
\noindent
{\rule{1.7 in}{0.035 in} START OF ABBREVIATED OUTPUT FILE \rule{1.7 in}{0.035 in} }
\begin{verbatim}
===========================================================================
Title : DESCRIPTION OF FINITE ELEMENT MESH                                 
===========================================================================

 Problem_Type:  Static Analysis

=======================
Profile of Problem Size
=======================

   Dimension of Problem        =      3

   Number Nodes                =    399
   Degrees of Freedom per node =      6
   Max No Nodes Per Element    =      4

   Number Elements             =    440
   Number Element Attributes   =      3
   Number Loaded Nodes         =    315
   Number Loaded Elements      =      0

-----------------------------------------------------------------------------------
Node#       X_coord       Y_coord       Z_coord    Dx    Dy    Dz    Rx    Ry    Rz
-----------------------------------------------------------------------------------

    1  0.000e+00 ft -4.667e+00 ft  1.753e+01 in     1     2     3     4     5     6 
    2  0.000e+00 ft -4.166e+00 ft  1.753e+01 in     7     8     9    10    11    12 
    3  0.000e+00 ft -3.665e+00 ft  1.753e+01 in    13    14    15    16    17    18 
    4  0.000e+00 ft -4.667e+00 ft -1.753e+01 in    19    20    21    22    23    24 

    ....... details of nodal coordinates deleted .......

  398  2.000e+02 ft  6.250e+00 ft  1.753e+01 in  2363  2364  2365  2366  2367  2368 
  399  2.000e+02 ft  8.333e+00 ft  1.753e+01 in  2369  2370  2371  2372  2373  2374 

----------------------------------------------------------------------------------
Element#     Type      node[1]   node[2]   node[3]   node[4]    Element_Attr_Name
----------------------------------------------------------------------------------

       1   SHELL_4NQ         1        20        21         2    girder_flange_attr
       2   SHELL_4NQ         2        21        22         3    girder_flange_attr
       3   SHELL_4NQ         2        21        24         5    girder_web_attr
       4   SHELL_4NQ         4        23        24         5    girder_flange_attr

       ....... details of shell finite elements removed .......

     437   SHELL_4NQ          368       387       388       369      deck_attr
     438   SHELL_4NQ          369       388       389       370      deck_attr
     439   SHELL_4NQ          370       389       398       379      deck_attr
     440   SHELL_4NQ          379       398       399       380      deck_attr

--------------------- 
Element Attribute Data :        
--------------------- 

ELEMENT_ATTR No.   1  : name = "girder_flange_attr" 
                      : section = "girder_flange" 
                      : material = "STEEL3" 
                      : type = SHELL_4NQ
                      : dof-mapping : gdof[0] =  1 : gdof[1] =    2 : gdof[2] =  3
                                      gdof[3] =  4 : gdof[4] =    5 : gdof[5] =  6
                      : Young's Modulus =  E =        2.900e+04 ksi
                      : Yielding Stress = fy =        5.000e+01 ksi
                      : Poisson's ratio = nu =        3.000e-01   
                      : Density         =        0.000e+00 (null)
                      : Inertia Izz     =        0.000e+00 (null)
                      : Area            =        0.000e+00 (null)

ELEMENT_ATTR No.   2  : name = "girder_web_attr" 
                      : section = "girder_flange" 
                      : material = "STEEL3" 
                      : type = SHELL_4NQ
                      : dof-mapping : gdof[0] =  1 : gdof[1] =  2 : gdof[2] =  3
                                      gdof[3] =  4 : gdof[4] =  5 : gdof[5] =  6
                      : Young's Modulus =  E =        2.900e+04 ksi
                      : Yielding Stress = fy =        5.000e+01 ksi
                      : Poisson's ratio = nu =        3.000e-01   
                      : Density         =        0.000e+00 (null)
                      : Inertia Izz     =        0.000e+00 (null)
                      : Area            =        0.000e+00 (null)


ELEMENT_ATTR No.   3  : name = "deck_attr" 
                      : section = "deck" 
                      : material = "concrete" 
                      : type = SHELL_4NQ
                      : dof-mapping : gdof[0] =  1 : gdof[1] =  2 : gdof[2] =  3
                                      gdof[3] =  4 : gdof[4] =  5 : gdof[5] =  6
                      : Young's Modulus =  E =        3.625e+03 ksi
                      : Yielding Stress = fy =        3.400e+03 psi
                      : Poisson's ratio = nu =        3.000e-01   
                      : Density         =        0.000e+00 (null)
                      : Inertia Izz     =        0.000e+00 (null)
                      : Area            =        0.000e+00 (null)
                        
EXTERNAL NODAL LOADINGS 
-------------------------------------------------------------------------------
Node#    Fx (lbf)   Fy (lbf)  Fz (lbf)    Mx (lbf.in)  My (lbf.in)  Mz (lbf.in)
-------------------------------------------------------------------------------
   13    0.000      0.000     -455.729    0.000        0.000        0.000
  393    0.000      0.000     -455.729    0.000        0.000        0.000
  399    0.000      0.000     -455.729    0.000        0.000        0.000
   19    0.000      0.000     -455.729    0.000        0.000        0.000
   32    0.000      0.000     -911.458    0.000        0.000        0.000
   38    0.000      0.000     -911.458    0.000        0.000        0.000

   ..... details of nodal loads removed ....

  344    0.000      0.000    -1700.000    0.000        0.000        0.000
  350    0.000      0.000    -1700.000    0.000        0.000        0.000
  363    0.000      0.000    -1700.000    0.000        0.000        0.000
  369    0.000      0.000    -1700.000    0.000        0.000        0.000

============= End of Finite Element Mesh Description ==============

*** COMPUTE AND PRINT STIFFNESS AND EXTERNAL LOAD MATRICES 

*** STATIC ANALYSIS PROBLEM 

------------------------------------------------------------------------------
 Node                                             Displacement
  No             displ-x           displ-y           displ-z             rot-x 
------------------------------------------------------------------------------
 units               in                in                in               rad 
   1         1.33978e-01      -1.89314e-05      -1.80219e-03       2.02410e-04
   2         1.34131e-01      -2.29040e-05      -7.12851e-04       1.58344e-04
   3         1.34036e-01      -2.64397e-05       1.11846e-04       1.17222e-04
   4         4.80519e-04       7.93065e-05      -3.01849e-06       7.53268e-07
   5         0.00000e+00       0.00000e+00       0.00000e+00       0.00000e+00
  13         1.32863e-01      -3.34551e-05      -1.44068e-02       3.23153e-04

  ....... details of nodal displacements removed ......

 395         7.72668e-02      -5.87620e-05       1.43256e-03       3.45080e-05
 396         7.74839e-02      -9.59758e-13       1.76084e-03       5.23559e-16
 397         7.72668e-02       5.87620e-05       1.43256e-03      -3.45080e-05
 398         7.74887e-02      -1.71932e-05      -6.61126e-03      -2.90572e-04
 399         7.79492e-02       3.34551e-05      -1.44068e-02      -3.23153e-04
\end{verbatim}
\rule{6.25 in}{0.035 in}
\end{footnotesize}

\vspace{0.24 in}\noindent
Figures \ref{fig: bridge-deck-deflection-part1} and \ref{fig: bridge-deck-deflection-part2}
are contour and three-dimensional views of the bridge deck deflections, respectively.
(the vertical axis of the deflection of Figure \ref{fig: bridge-deck-deflection-part2}
has the units of inches).
The bridge deflections are due to dead loads alone, and since
the bridge geometry and section properties are symmetric,
we expect that the deflections will also exhibit symmetry. They do.

\clearpage
\begin{figure}[th]
\epsfxsize= 4.5truein
\centerline{\epsfbox{my-chapter6-fig15.ps}}
\vspace{0.20 in}
\caption{Highway Bridge : Contour Plot of Bridge Deck Deflections}
\label{fig: bridge-deck-deflection-part1}
\end{figure}

\begin{figure}[h]
\epsfxsize= 4.5truein
\centerline{\epsfbox{my-chapter6-fig16.ps}}
\vspace{0.20 in}
\caption{Three-Dimensional Mesh of Bridge Deck Deflections}
\label{fig: bridge-deck-deflection-part2}
\end{figure}

\clearpage
\vspace{0.15 in}\noindent
{\bf Moving Point Load Input File :}
In Part 2 of our analysis, the latter sections of the input file are
extended so that response evelopes are computed for a point load
moving along the bridge.

\begin{figure}[h]
\epsfxsize= 6.0truein
\centerline{\epsfbox{my-chapter6-fig19.ps}}
\vspace{0.20 in}
\caption{Plan and Front Elevation of Two-Span Continuous Highway Bridge with Moving Live Load}
\label{fig: bridge-moving-load-part1}
\end{figure}

\vspace{0.15 in}\noindent
Figure \ref{fig: bridge-moving-load-part1} shows plan and elevation views
of the bridge and moving point load. A cross sectional view of
the bridge and the moving point load is
shown in Figure \ref{fig: bridge-moving-load-part2}.
Abbreviated details of the input file are as follows:

\begin{figure}[ht]
\epsfxsize= 6.0truein
\centerline{\epsfbox{my-chapter6-fig20.ps}}
\vspace{0.20 in}
\caption{Cross Section of Two-Span Continuous Highway Bridge with Moving Live Load}
\label{fig: bridge-moving-load-part2}
\end{figure}

\vskip 0.1truein
\begin{footnotesize}
\vspace{0.10 in}
\noindent
{\rule{2.3 in}{0.035 in} START OF INPUT FILE \rule{2.3 in}{0.035 in} }
\begin{verbatim}
/* 
 *  =====================================================================
 *  Analysis of Two Span Continuous Composite Beam (100-100ft), analysis
 *  assuming neutral axis of concrete deck coincide with top girder flange
 *  Moving load along one girder
 * 
 *  Written By: Wane-Jang Lin                              December, 1994
 *  =====================================================================
 */ 

/* [a] : Setup problem specific parameters */

   NDimension         = 3;
   NDofPerNode        = 6;
   MaxNodesPerElement = 4;

   StartMesh();

   print "*** GENERATE GRID OF NODES FOR FE MODEL \n\n";

         ..... details are the same as in static analysis datafile ....

   print "*** ATTACH ELEMENTS TO GRID OF NODES \n\n";

         ..... details are the same as in static analysis datafile ....

   print "*** DEFINE ELEMENT, SECTION AND MATERIAL PROPERTIES \n\n";

         ..... details are the same as in static analysis datafile ....

   print "*** SET UP BOUNDARY CONDITIONS \n\n";

         ..... details are the same as in static analysis datafile ....

/* [b] : Compile and Print Finite Element Mesh */

   EndMesh();
   PrintMesh();

/* [c] : Compute Stiffness Matrix */

   SetUnitsType("US");
   stiff = Stiff();

/* [d] : Compute Influence Lines for Moving Live Loads */

   print "\n*** STATIC ANALYSIS PROBLEM \n\n";

   Fx = 0 lbf;    Fy = 0 lbf;    Fz = -1000.0 kips;
   Mx = 0 lbf*in; My = 0 lbf*in; Mz = 0 lbf*in;

   nodeno1 = 97;
   nodeno2 = 103;
   step   = div_L*2+1;
   print "node no 1 =",nodeno1,"\n";
   print "node no 2 =",nodeno2,"\n";
   print "Fz        =",Fz,"\n";
   print "step      =",step,"\n";

   influ_line1 = Zero([ step , 1 ]);  /* array for influence line 1 */
   influ_line2 = Zero([ step , 1 ]);  /* array for influence line 2 */

   load_node = 2;
   lu = Decompose (stiff);
   for( i=1 ; i<=step ; i=i+1 ) {
        NodeLoad( load_node, [Fx,Fy,Fz,Mx,My,Mz] );

        eload = ExternalLoad();
        displ = LUDecomposition( stiff, eload );

        node_displ_1 = GetDispl( [nodeno1], displ );
        node_displ_2 = GetDispl( [nodeno2], displ );

        influ_line1[i][1] = node_displ_1[1][3];
        influ_line2[i][1] = node_displ_2[1][3];

        NodeLoad( load_node, [-Fx,-Fy,-Fz,-Mx,-My,-Mz] );
        load_node = load_node + nodes_per_section;
   }

   PrintMatrix(influ_line1);
   PrintMatrix(influ_line2);

   quit;
\end{verbatim}
\rule{6.25 in}{0.035 in}
\end{footnotesize}

\vspace{0.25 in}\noindent
Figure \ref{fig: bridge-deck-influence-part1} shows the influence line of
vertical displacement in the middle of one span (i.e. at node no 97)
for the first girder subjected to 1000 kips moving live load.
Similarly, Figure \ref{fig: bridge-deck-influence-part2} shows
the influence line of displacement in the middle of one span (i.e. node no 2 = 103)
due to the 1000 kips moving live load.

\vspace{0.15 in}
\noindent\hspace{0.50 in}
The moving load analysis is one situation where a family of
linear equations is solved with multiple right-hand sides.
With this in mind, notice how we have called the function
{\tt Decompose()} once to decompose {\tt stiff} into
a product of upper and lower triangular matrices,
and then called {\tt LUDecomposition()} to compute the
forward and backward substitution for each analysis.
This strategy of equation solving reduces the overall solution
time by approximately 70\%.

\clearpage
\begin{figure}[th]
\epsfxsize= 4.5truein
\centerline{\epsfbox{my-chapter6-fig17.ps}}
\vspace{0.20 in}
\caption{Influence line of displacement in the middle of one span
for the first girder subjected to 1000 kips moving live load}
\label{fig: bridge-deck-influence-part1}
\end{figure}

\begin{figure}[h]
\epsfxsize= 4.5truein
\centerline{\epsfbox{my-chapter6-fig18.ps}}
\vspace{0.20 in}
\caption{Influence line of displacement in the middle of one span
for the second girder}
\label{fig: bridge-deck-influence-part2}
\end{figure}

\clearpage
\section{Time-History Analyses}

\subsection{Modal Analysis of Five Story Steel Frame}

\vspace{0.15 in}\noindent
{\bf Description of Problems :}
In this section we compute the linear time-history response of
the steel frame structure described in Section 6.1.1. to
dead and live gravity loads, plus a 1979 El Centro Earthquake ground motion
scaled to moderate intensity.

\begin{figure}[h]
\epsfxsize= 6.0truein
\centerline{\epsfbox{my-chapter6-fig30.ps}}
\caption{Elevation View and Simplified Model of Moment Resistant Frame
subject to Scaled 1979 El Centro Earthquake}
\label{fig: elevation-and-model}
\end{figure}

\vspace{0.15 in}\noindent
The left-hand side of Figure \ref{fig: elevation-and-model}
is a simplified elevation view of the frame also
shown in Figure \ref{fig: five-story-building-elevation-view}.
Unlike the analysis conducted in Section 6.1.1.,
we will assume that all nodal rotations and vertical deflections of the frame are zero,
and that the horizontal deflections across each floor may be lumped into
a single degree of freedom.
Together these assumptions imply that each floor will
move as a rigid body in the horizontal direction, and that the global
frame behavior may be described with only five degrees of freedom.
The ight-hand side of Figure \ref{fig: elevation-and-model} shows the
simplified model that will be used in the earthquake time-history
analysis -- the shaded filled boxes represent the lumped masses at
each floor level, and the global degrees of freedom begin at
the first floor level and increase to the roof.

\begin{figure}[ht]
\epsfxsize= 4.8truein
\centerline{\epsfbox{my-chapter6-fig5.ps}}
\caption{Ground Acceleration for 1979 El Centro Earthquake}
\label{fig: elcentro-1979}
\end{figure}

\vspace{0.15 in}
Figure \ref{fig: elcentro-1979} shows details of a 10 segment
accelerograph extracted from the 1979 El Centro Earthquake ground motion.
The peak ground acceleration is 86.63 cm/sec/sec. The ground motion
will be scaled to that the peak ground acceleration is 0.15 g.

\vspace{0.15 in}\noindent
{\bf Input File :} The following input file is based on
the input file presented in Section 6.1.1,
with appropriate extensions for the earthquake analysis,
and generation of time-history response quantities for
post-analysis plotting.

\vskip 0.1truein
\begin{footnotesize}
\vspace{0.10 in}
\noindent
{\rule{2.3 in}{0.035 in} START OF INPUT FILE \rule{2.3 in}{0.035 in} }
\begin{verbatim}
/*
 *  ============================================================
 *  Modal Analysis of Five Story Steel Moment Frame subject to
 *  El Centro ground motion.
 *  
 *  Written By: Mark Austin                           July, 1995  
 *  ============================================================
 */

/* [a] : Setup problem specific parameters */

   NDimension         = 2;
   NDofPerNode        = 3;
   MaxNodesPerElement = 2;

   StartMesh();

/* [b] : Generate two-dimensional grid of nodes */

   ..... same as for five story building ....

/* [c] : Attach column elements to nodes */

   ..... same as for five story building ....

/* [d] : Attach beam elements to nodes */

   ..... same as for five story building ....

/* [e] : Define section and material properties */

   ElementAttr("mrf_element") { type     = "FRAME_2D";
                                section  = "mrfsection";
                                material = "mrfmaterial";
                              }

   SectionAttr("mrfsection") { Izz       = 1541.9 in^4;
                               Iyy       =  486.3 in^4;
                               depth     =   12.0 in;
                               width     =   12.0 in;
                               area      =   47.4 in^2;
                             }

   MaterialAttr("mrfmaterial") { density = 7850 kg/m^3;
                                 poisson = 0.27;
                                 yield   = 275000 MPa;
                                 E       = 200000 MPa;
                               }

/* [f] : Apply full-fixity to columns at foundation level */

   for(nodeno = 1; nodeno <= 4; nodeno = nodeno + 1) {
       FixNode( nodeno, [ 1, 1, 1 ]);
   }

   for(nodeno = 5; nodeno <= 24; nodeno = nodeno + 1) {
       FixNode( nodeno, [ 0, 1, 1 ]);
   }

   LinkNode([  5,  6,  7,  8 ], [ 1, 0, 0] );
   LinkNode([  9, 10, 11, 12 ], [ 1, 0, 0] );
   LinkNode([ 13, 14, 15, 16 ], [ 1, 0, 0] );
   LinkNode([ 17, 18, 19, 20 ], [ 1, 0, 0] );
   LinkNode([ 21, 22, 23, 24 ], [ 1, 0, 0] );

/* [g] : Compile and Print Finite Element Mesh */

   EndMesh();
   PrintMesh();

/* [h] : Compute "stiffness" matrix; manually assemble "mass" matrices */

   stiff  = Stiff();
   PrintMatrix(stiff);

   mass = Zero([5,5]);
   mass = ColumnUnits ( mass , [N/m] );
   mass = RowUnits    ( mass , [sec^2] );

/* [i] : Manually assemble "mass" matrix : assume (dead + live) loads */

   dead_load       = 80 lbf/ft^2;
   floor_live_load = 40 lbf/ft^2;
   roof_live_load  = 20 lbf/ft^2;
   frame_spacing   = 20 ft;
   floor_length    = 55 ft;

   tributary_area = frame_spacing * floor_length;
   for(i = 1; i <= 5; i = i + 1) {
     if(i <= 4) then {
        floor_load = tributary_area * (dead_load + floor_live_load);
     } else {
        floor_load = tributary_area * (dead_load + roof_live_load);
     }

     mass [i][i] = mass[i][i] + floor_load / (32.2 ft/sec^2);
   }

   PrintMatrix(mass);

/* [j] : Compute and print eigenvalues and eigenvectors */

   no_eigen = 2;
   eigen       = Eigen(stiff, mass, [ no_eigen ]);
   eigenvalue  = Eigenvalue(eigen);
   eigenvector = Eigenvector(eigen);

   for(i = 1; i <= no_eigen; i = i + 1) {
       print "Mode", i ," : w^2 = ", eigenvalue[i][1];
       print " : T = ", 2*PI/sqrt(eigenvalue[i][1]) ,"\n";
   }

   PrintMatrix(eigenvector);

/* [k] : Generalized mass and stiffness matrices */

   EigenTrans = Trans(eigenvector);
   Mstar   = EigenTrans*mass*eigenvector;
   Kstar   = EigenTrans*stiff*eigenvector;

   PrintMatrix( Mstar );
   PrintMatrix( Kstar );

/* [m] : Setup Rayleigh Damping for Frame Structure */ 

   rdamping = 0.05;
   W1 = sqrt ( eigenvalue[1][1]);
   W2 = sqrt ( eigenvalue[2][1]);

   A0 = 2*rdamping*W1*W2/(W1 + W2);
   A1 = 2*rdamping/(W1 + W2);

   print "A0 = ", A0, " A1 = ", A1, "\n";

   Cstar = A0*Mstar + A1*Kstar;
   PrintMatrix( Cstar );

/* [n] : Define earthquake loadings... */ 

   Elcentro = ColumnUnits( [
     14.56;    13.77;     6.13;     3.73;     1.32;    -6.81;   -16.22;   -22.41;

     ...... details of earthquake removed .....

    -10.05;   -12.35;   -15.72;     0.00  ], [cm/sec/sec] );

   ground_motion_scale_factor = 0.15*981.0/86.63;

   PrintMatrix( Elcentro );

/* [o] : Initialize system displacement, velocity, and load vectors */

   displ  = ColumnUnits( Matrix([5,1]), [m]    );
   vel    = ColumnUnits( Matrix([5,1]), [m/sec]);
   eload  = ColumnUnits( Matrix([5,1]), [kN]);
   r      = One([5,1]);

/* [p] : Initialize modal displacement, velocity, and acc'n vectors */

   Mdispl  = ColumnUnits( Matrix([ no_eigen,1 ]), [m]    );
   Mvel    = ColumnUnits( Matrix([ no_eigen,1 ]), [m/sec]);
   Maccel  = ColumnUnits( Matrix([ no_eigen,1 ]), [m/sec/sec]);

/* 
 * [q] : Allocate Matrix to store five response parameters --
 *       Col 1 = time (sec);
 *       Col 2 = 1st mode displacement (cm);
 *       Col 3 = 2nd mode displacement (cm);
 *       Col 4 = 1st + 2nd mode displacement (cm);
 *       Col 5 = Total energy (Joules)
 */ 

   dt     = 0.02 sec;
   nsteps = 600;
   beta   = 0.25;
   gamma  = 0.5;

   response = ColumnUnits( Matrix([nsteps+1,5]), [sec], [1]);
   response = ColumnUnits( response,  [cm], [2]);
   response = ColumnUnits( response,  [cm], [3]);
   response = ColumnUnits( response,  [cm], [4]);
   response = ColumnUnits( response, [Jou], [5]);

/* [r] : Compute (and compute LU decomposition) effective mass */

   MASS  = Mstar + rdamping*dt*Cstar + Kstar*beta*dt*dt;
   lu    = Decompose(Copy(MASS));

/* [s] : Mode-Displacement Solution for Response of Undamped MDOF System  */

   MassTemp = -mass*r;
   for(i = 1; i <= nsteps; i = i + 1) {
       print "*** Start Step ",i,"\n";

       if(i == 2) {
          SetUnitsOff;
       }

    /* [s.1] : Update external load */

       if(i <= 500) then {
          eload = MassTemp*Elcentro[i][1]*ground_motion_scale_factor;
       } else {
          eload = MassTemp*(0)*ground_motion_scale_factor;
       }

       Pstar = EigenTrans*eload;

       R = Pstar - Kstar*(Mdispl + Mvel*dt + Maccel*(dt*dt/2.0)*(1-2*beta)) -
           Cstar*(Mvel + Maccel*dt*(1-gamma));

    /* [s.2] : Compute new acceleration, velocity and displacement  */

       Maccel_new = Substitution(lu,R); 
       Mvel_new   = Mvel   + dt*(Maccel*(1.0-gamma) + gamma*Maccel_new);
       Mdispl_new = Mdispl + dt*Mvel + ((1 - 2*beta)*Maccel +
                    2*beta*Maccel_new)*dt*dt/2;

    /* [s.3] : Update and print new response */

       Maccel = Maccel_new;
       Mvel   = Mvel_new;
       Mdispl = Mdispl_new;

    /* [s.4] : Combine Modes */

       displ = eigenvector*Mdispl;
       vel   = eigenvector*Mvel;

    /* [s.5] : Compute Total System Energy */

       a1 = Trans(vel);
       a2 = Trans(displ);
       e1 = a1*mass*vel;
       e2 = a2*stiff*displ;
       energy = 0.5*(e1 + e2);

    /* [s.6] : Save components of time-history response */

       response[i+1][1] = i*dt;                            /* Time                  */
       response[i+1][2] = eigenvector[1][1]*Mdispl[1][1];  /* 1st mode displacement */
       response[i+1][3] = eigenvector[1][2]*Mdispl[2][1];  /* 2nd mode displacement */
       response[i+1][4] = displ[1][1];               /* 1st + 2nd mode displacement */
       response[i+1][5] = energy[1][1];                    /* System Energy         */
   }

/* [t] : Print response matrix and quit */

   SetUnitsOn;
   PrintMatrix(response);
   quit;
\end{verbatim}
\rule{6.25 in}{0.035 in}
\end{footnotesize}

\vspace{0.15 in}\noindent
Points to note are:

\vspace{0.10 in}
\begin{description}
\item{[1]}
Notice that we have used the finite element routines to
compute the stiffness matrix, but have manually assembled the mass matrix.
It is assumed that when the earthquake occurs, the frame loads will
equal the dead loading plus the full live loading.
The tributary area equals the frame spacing (20 ft)
times the frame width (55 ft).

\item{[2]}
The global mass and stiffness matrices are $(5 \times 5)$ matrices.
The time-history analysis is simplified by computing the Newmark
integration on the first and second modes of the frame alone.
The generalized mass and stiffness matrices are as described in Section 4.3.

\item{[3]}
The damping matrix is taken as a linear combination
of the generalized mass and stiffness matrices, namely:

\begin{equation}
{\bf C^*} = a_o \cdot {\bf M^*} + a_1 \cdot {\bf K^*}.
\label{eq: damping-part1}
\end{equation}

Let $w_m$ and $\xi_m$ be the circular natural frequency
and ratio of critical damping in the mode $m$,
and $w_n$ and $\xi_n$ be the circular natural frequency
and ratio of critical damping in the mode $n$.
It can be shown that the coefficients $a_1$ and $a_2$ are given by:

\begin{equation}
\left[
\begin{array}{r}
a_1 \\ a_2 \\
\end{array} \right] =
{{2 \cdot w_m \cdot w_n} \over {w_n^2 - w_m^2}} \cdot
\left[
\begin{array}{rr}
    {w_n}  &    -w_m   \\
{-1/{w_n}} & {1/{w_m}} \\
\end{array}
\right] \cdot \left[
\begin{array}{r}
\xi_m \\ \xi_n \\
\end{array}
\right]
\label{eq: damping-part2}
\end{equation}

For this analysis we will let $m ~=~ 1$ and $n ~=~ 2$,
and assume that the critical ratio of damping is the same in both modes.
The input file variable {\tt rdamping} represents the ratio
of critical damping in the first and second modes. Analyses will
be conducted for {\tt rdamping} {\tt =} {\tt 0.0},
and {\tt rdamping} {\tt =} {\tt 0.05}.

\item{[4]}
The time-history analysis is based on the Newmark equations
given in Section 4.2, and the modal analysis equations specified
in Section 4.3. At each step the analysis we compute
solutions to the equation of equilibrium

\begin{equation}
{\bf \Phi^T} {\bf M} {\bf \Phi} {\ddot Y}(t) +
{\bf \Phi^T} {\bf C} {\bf \Phi} {\dot Y}(t) +
{\bf \Phi^T} {\bf K} {\bf \Phi} {Y}(t) =
{\bf \Phi^T} {P}(t) = - {\bf \Phi^T} {\bf M} {\bf r} {\ddot X_g}(t).
\label{eq: damping-part3}
\end{equation}

Here ${\bf \Phi}$, {\bf M}, {\bf C}, and {\bf K} are as previously defined.
{\bf r} is a $(5 \times 1)$ that describes the movement in each of the
frame degrees of freedom due to a unit ground displacement -- in this
case, {\bf r} = $(5 \times 1)$ vector of ones.
${\ddot X_g}$(t) is the ground acceleration at time $t$.

\item{[5]}
The response is computed for 600 steps of time-step 0.02 seconds (i.e. 12 seconds).
The first loop of analysis is conducted with the units
checking turned on -- thereafter, the units are switched off.
Constant acceleration across each time-step is achieved by
setting the Newmark parameters $\beta ~=~ 0.25$ and $\gamma ~=~ 0.5$ --
see Section 4.2 for a detailed discussion on the Newmark method.

\item{[6]}
At the end of each time-step we compute the sum of kinetic
plus potential energy for the frame's first two modes. Theoretical
considerations indicate that when $\beta ~=~ 0.25$ and $\gamma ~=~ 0.5$,
the total energy of the undamped system will be conserved after the
earthquake motion has stopped (i.e. the last 2 seconds of the computed response).

\item{[7]}
We have allocated a 600 by 5 response matrix to stored selected components
of the time-history response. See the input file for a description
of the contents in each column.

\end{description}

\vspace{0.15 in}\noindent
{\bf Abbreviated Output File :} The following output file is a summary
of the output generated for the simulation case {\tt rdamping} = 0.0. 

\vspace{0.20 truein}
\begin{footnotesize}
\noindent
{\rule{1.7 in}{0.035 in} START OF ABBREVIATED OUTPUT FILE \rule{1.7 in}{0.035 in} }
\begin{verbatim}
==========================================
Title : DESCRIPTION OF FINITE ELEMENT MESH                                 
==========================================

=======================
Profile of Problem Size
=======================

   Dimension of Problem        =      2

   Number Nodes                =     24
   Degrees of Freedom per node =      3
   Max No Nodes Per Element    =      2

   Number Elements             =     35
   Number Element Attributes   =      1
   Number Loaded Nodes         =      0
   Number Loaded Elements      =      0

------------------------------------------------------------
Node#      X_coord           Y_coord          Tx    Ty    Rz  
------------------------------------------------------------

    1    0.0000e+00 ft    0.0000e+00 ft    -1    -2    -3 
    2    2.0000e+01 ft    0.0000e+00 ft    -4    -5    -6 
    3    3.5000e+01 ft    0.0000e+00 ft    -7    -8    -9 
    4    5.5000e+01 ft    0.0000e+00 ft   -10   -11   -12 

    ..... details of nodal coordinates removed .....

   24    5.5000e+01 ft    5.0000e+01 ft     5   -51   -52 

------------------------------------------------------------------
Element#     Type         node[1]   node[2]      Element_Attr_Name
------------------------------------------------------------------

       1   FRAME_2D             1         5      mrf_element
       2   FRAME_2D             5         9      mrf_element

       ..... details of frame elements removed .....

      34   FRAME_2D            22        23      mrf_element
      35   FRAME_2D            23        24      mrf_element

------------------------ 
Element Attribute Data :        
------------------------ 

ELEMENT_ATTR No.   1  : name = "mrf_element" 
                      : section = "mrfsection" 
                      : material = "mrfmaterial" 
                      : type = FRAME_2D
                      : gdof [0] =    1 : gdof[1] =    2 : gdof[2] =    3
                      : Young's Modulus =  E =        2.000e+05 MPa
                      : Yielding Stress = fy =        2.750e+05 MPa
                      : Poisson's ratio = nu =        2.700e-01   
                      : Density         =        7.850e+03 kg/m^3
                      : Inertia Izz     =        1.542e+03 in^4
                      : Area            =        4.740e+01 in^2

============= End of Finite Element Mesh Description ==============

SKYLINE MATRIX : "stiff"

row/col                  1             2             3             4             5   
        units          N/m           N/m           N/m           N/m           N/m   
   1            4.35158e+08  -2.17579e+08   0.00000e+00   0.00000e+00   0.00000e+00 
   2           -2.17579e+08   4.35158e+08  -2.17579e+08   0.00000e+00   0.00000e+00 
   3            0.00000e+00  -2.17579e+08   4.35158e+08  -2.17579e+08   0.00000e+00 
   4            0.00000e+00   0.00000e+00  -2.17579e+08   4.35158e+08  -2.17579e+08 
   5            0.00000e+00   0.00000e+00   0.00000e+00  -2.17579e+08   2.17579e+08 

MATRIX : "mass"

row/col                  1            2            3            4            5   
        units          N/m          N/m          N/m          N/m          N/m   
   1    sec^2   5.98259e+04  0.00000e+00  0.00000e+00  0.00000e+00  0.00000e+00
   2    sec^2   0.00000e+00  5.98259e+04  0.00000e+00  0.00000e+00  0.00000e+00
   3    sec^2   0.00000e+00  0.00000e+00  5.98259e+04  0.00000e+00  0.00000e+00
   4    sec^2   0.00000e+00  0.00000e+00  0.00000e+00  5.98259e+04  0.00000e+00
   5    sec^2   0.00000e+00  0.00000e+00  0.00000e+00  0.00000e+00  4.98550e+04

*** SUBSPACE ITERATION CONVERGED IN  9 ITERATIONS 
Mode         1  : w^2 =        313 1/sec^2  : T =     0.3551 sec 
Mode         2  : w^2 =       2648 1/sec^2  : T =     0.1221 sec 

MATRIX : "eigenvector"

row/col                  1            2   
        units                             
   1            2.91629e-01  7.87127e-01
   2            5.58155e-01  1.00000e+00
   3            7.76637e-01  4.84597e-01
   4            9.28270e-01 -3.82551e-01
   5            1.00000e+00 -9.70739e-01

MATRIX : "Mstar"

row/col                  1            2   
        units          N/m          N/m   
   1    sec^2   1.61217e+05  7.27596e-12
   2    sec^2   7.27596e-12  1.66677e+05

MATRIX : "Kstar"

row/col                  1            2   
        units          N/m          N/m   
   1            5.04687e+07  2.04891e-08
   2            0.00000e+00  4.41345e+08
A0 =          0 1/sec  A1 =          0 sec 

MATRIX : "Cstar"

row/col                  1            2   
        units     kg/sec^3     kg/sec^3   
   1    sec^2   0.00000e+00  0.00000e+00
   2    sec^2   0.00000e+00  0.00000e+00

MATRIX : "response"

row/col                  1            2            3            4            5   
        units          sec           cm           cm           cm          Jou   
   1            0.00000e+00  0.00000e+00  0.00000e+00  0.00000e+00  0.00000e+00
   2            2.00000e-02 -5.17642e-04 -3.51330e-04 -8.68972e-04  2.82918e-01

   ...... details of response matrix removed : for details, see the figures.

 600            1.19800e+01 -4.21771e-02 -2.57188e-02 -6.78958e-02  3.88629e+02
 601            1.20000e+01 -7.47478e-02 -1.97584e-02 -9.45062e-02  3.88629e+02
\end{verbatim}
\rule{6.25 in}{0.035 in}
\end{footnotesize}

\vspace{0.25 in}\noindent
Matrices {\tt Mstar} and {\tt Kstar} in the program output correspond
to the generalized mass and stiffness matrices, respectively.
We expect that the computed eigenvectors will be orthogonal with
respect to the mass and stiffness matrices, and hence,
the generalized mass and stiffness matrices will be (nearly) diagonal --
we see that in practice, some round-off errors occur.

\vspace{0.15 in}
\noindent\hspace{0.50 in}
The first and second modes of vibration have natural periods
{\tt 0.3551} sec and {\tt 0.1221} sec, respectively.
When the frame is undamped,
coefficients $a_o$ and $a_1$ evaluate to zero, as expected.

\vspace{0.15 in}
\noindent\hspace{0.50 in}
Figures \ref{fig: bmodal-results-1} to \ref{fig: bmodal-results-4}
summarize the time-history response of the undamped moment resistant frame.
Figure \ref{fig: bmodal-results-1} is the time history response for mode 1,
Figure \ref{fig: bmodal-results-2} the time history response for mode 2,
and Figure \ref{fig: bmodal-results-3} the time history response for
modes 1 and 2 combined. Figure \ref{fig: bmodal-results-4} plots the kinetic
plus potential energy of the frame -- you should notice that once
the ground motion has ceased (at t = 10 seconds),
the Newmark method conserves energy.

\vspace{0.15 in}
\noindent\hspace{0.50 in}
Figures \ref{fig: bmodal-results-5} and \ref{fig: bmodal-results-5}
plot the total time-history response and energy when the frame has 5\% damping.
After the ground motion has ceased, the roof displacement decreases
steadily to zero. You can easily observe that between t = 10 seconds
and t = 12 seconds, the roof oscillates through approximately 5.5 cycles.
This implies an approximate natural frequency of $2/5.5 \approx 0.36$ seconds. 

\clearpage
\begin{figure}[ht]
\epsfxsize= 4.7truein
\centerline{\epsfbox{my-chapter6-fig6.ps}}
\caption{Modal Analysis : First Mode Displacement of Roof (cm) versus Time (sec)}
\label{fig: bmodal-results-1}
\end{figure}

\begin{figure}[h]
\epsfxsize= 4.7truein
\centerline{\epsfbox{my-chapter6-fig7.ps}}
\caption{Modal Analysis : Second Mode Displacement of Roof (cm) versus Time (sec)}
\label{fig: bmodal-results-2}
\end{figure}

\clearpage
\begin{figure}[ht]
\epsfxsize= 4.7truein
\centerline{\epsfbox{my-chapter6-fig8.ps}}
\caption{Modal Analysis : First + Second Mode Displacement of Roof (cm) versus Time (sec)}
\label{fig: bmodal-results-3}
\end{figure}

\begin{figure}[h]
\epsfxsize= 4.7truein
\centerline{\epsfbox{my-chapter6-fig9.ps}}
\caption{Modal Analysis : Total Energy (Joules) versus Time (sec)}
\label{fig: bmodal-results-4}
\end{figure}

\clearpage
\begin{figure}[ht]
\epsfxsize= 4.7truein
\centerline{\epsfbox{my-chapter6-fig28.ps}}
\caption{Modal Analysis at 5\% damping : First + Second Mode Displacement of Roof (cm) versus Time (sec)}
\label{fig: bmodal-results-5}
\end{figure}

\begin{figure}[h]
\epsfxsize= 4.7truein
\centerline{\epsfbox{my-chapter6-fig29.ps}}
\caption{Modal Analysis at 5\% damping : Total Energy (Joules) versus Time (sec)}
\label{fig: bmodal-results-6}
\end{figure}

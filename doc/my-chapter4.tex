
\chapter{Computational Methods for Dynamic Analysis of Structures}

\section{Introduction}

\vspace{0.15 in}
\noindent\hspace{0.50 in}
In this section we demonstrate how ALADDIN may be used to compute the
time-history response of a linear elastic multi-degree of freedom system 
defined by the equation of equilibrium

\begin{equation}
{\bf M} \ddot X(t) + {\bf C} {\dot X}(t) + {\bf K} X(t) = {\bf P}(t).
\label{eq: equl}
\end{equation}

\vspace{0.15 in}\noindent
In equation (\ref{eq: equl}) {\bf M}, {\bf C},
and {\bf K} are $(n \times n)$ mass,
damping, and stiffness matrices, respectively.
${\bf P}(t)$, ${\tt X}(t)$, $\dot{\tt X}{(t)}$, and $\ddot{X}{(t)}$
are $(n \times 1)$ external load, displacement, velocity,
and acceleration vectors at time {\tt t}.

\vspace{0.15 in}
\noindent\hspace{0.50 in}
Two numerical procedures are described and demonstrated in this chapter.
We begin with the method of Newmark Integration to compute the time-history
response of a four story shear structure. 
Then we repeat the analysis using the method of Modal Analysis.
In both cases, the shear structure is acted upon by a saw-toothed 
external force applied to the roof-level degree of freedom.

\section{Method of Newmark Integration}

\vspace{0.15 in}
\noindent\hspace{0.5 in}
Newmark integration methods ~\cite{bathe82} approximate the time
dependent response of linear and nonlinear 2nd-order equations by insisting
that equilibrium be satisfied only at a discrete number of points (or timesteps).
If $(t)$ and $(t+\triangle t)$ are successive timesteps in the integration procedure,
the two equations of equilibrium that must be satisfied are 

\begin{equation}
{\bf M} \ddot{X}(t) + {\bf C} \dot{X}(t) + {\bf K} {X}(t) = {\bf P}(t),
\label{eq: equl-1}
\end{equation}

\vspace{0.01 in}\noindent
and

\begin{equation}
{\bf M} \ddot{X}(t+\triangle t) + {\bf C} \dot{X}(t+\triangle t) +
{\bf K}      {X}(t+\triangle t) = {\bf P}(t+\triangle t).
\label{eq: equl-2}
\end{equation}

\vspace{0.05 in}\noindent
Now let's assume that solutions to equation (\ref{eq: equl-1}) are known,
and (\ref{eq: equl-2}) needs to be solved.
At each timestep there are $3n$ unknowns corresponding to
the displacement, velocity, and acceleration of each component of $X$.
Since we only have $n$ equations, the natural relationship existing between the
acceleration and velocity,

\begin{eqnarray}
{\dot X} (t+\triangle t) &= {\dot X}(t) + \int_{(t)}^{(t+\triangle t)} {\ddot X}(\tau) d \tau,
\end{eqnarray}

\vspace{0.05 in}\noindent
and velocity and displacement

\begin{eqnarray}
{X} (t+\triangle t) &= {X}(t) + \int_{(t)}^{(t+\triangle t)} {\dot  X}(\tau) d \tau,
\end{eqnarray}

\vspace{0.05 in}\noindent
must be enforced to reduce the number of unknowns to $n$.
${\ddot X}(\tau)$ is an unknown function for the acceleration across the timestep.
The Newmark family of integration methods assume that:
(1) acceleration within the timestep behaves in a prescribed manner, and
(2) the integral of acceleration across the timestep may be expressed
as a linear combination of accelerations at the endpoints.
Discrete counterparts to the continuous update in velocity and displacement are:

\begin{eqnarray}
{\dot X}(t+\triangle t) &= &{\dot X}(t) +
{\triangle t} {\left[ (1-\gamma){{\ddot X}(t)} + \gamma {\ddot X}(t+\triangle t) \right]} \\
     {X}(t+\triangle t) &= &{X}(t) + {\triangle t} {\dot X}(t) +
{{{\triangle t}^2} \over 2} {\left[(1-2\beta){{\ddot X}(t)} + 2\beta {\ddot X}(t+\triangle t) \right]}
\end{eqnarray}

\vspace{0.10 in}\noindent
with the parameters $\gamma$ and $\beta$ determining the accuracy and stability
of the method under consideration.
The equations for discrete update in velocity and displacement are
substituted into equation (\ref{eq: equl-2}) and rearranged to give:

\begin{equation}
\hat{\bf M} \ddot{X}(t+\triangle t) = \hat{\bf P}{(t + \triangle t)}
\label{eq: equl3}
\end{equation}

\vspace{0.05 in}\noindent
where

\begin{equation}
\hat{\bf M} = {\bf M} + \gamma {\triangle t} {\bf C} + \beta {\triangle t}^2 {\bf K}
\label{eq: equl4}
\end{equation}

\vspace{0.05 in}\noindent
and

\begin{eqnarray}
\hat{\bf P}{(t + \triangle t)} &= {\bf P}{(t + \triangle t)} - {\bf C} \dot{X}(t) - {\bf K} {X}(t) -
{\triangle t} \left[ {\bf K} \right]  \dot{X}(t) - \nonumber \\
& {\triangle t}
{\left[ (1- \gamma) {\bf C} + {{\triangle t} \over 2} (1 - 2 \beta) {\bf K} \right]} \ddot{\bf X}(t).  
\end{eqnarray}

\vspace{0.15 in}\noindent
It is well known that when $\gamma = 1/2$ and $\beta=1/4$,
acceleration is constant within the timestep $t \in \left[ t, {t+\triangle t} \right]$,
and equal to the average of the endpoint accelerations.
In such cases, approximations to the velocity and displacement
will be linear and parabolic, respectively.
Moreover, this discrete approximation is second order accurate and unconditionally stable.
It conserves energy exactly for the free response vibration of linear undamped SDOF
systems (we will check for conservation of energy in the numerical examples that follow).

\vspace{0.25 in}\noindent
{\bf Numerical Example :}
The method of Newmark Integration is demonstrated by computing the
time-history response of the four story building structure caused by a
time-varying external load applied to the roof level.
Details of the shear building and external loading are shown in
Figures \ref{fig:solution-shear-building} and \ref{fig:applied-force-versus-time}.

\begin{figure} [h]
\epsfxsize=6.0truein
\centerline{\epsfbox{my-chapter3-fig5.ps}}
\caption{Schematic of Shear Building for Newmark Analysis}
\label{fig:solution-shear-building}
\end{figure}

\vspace{0.15 in}\noindent
A simplified model of the building is obtained by
assuming that all of the building mass is lumped at the floor levels,
that the floor beams are rigid, and that the columns are axially rigid.
Together these assumptions generate model that is commonly
known as a shear-type building model, where displacements at
each floor level may be described by one degree-of-freedom alone.
Only four degrees of freedom are needed to
describe total displacements of the structure.

\vspace{0.15 in}
\noindent\hspace{0.5 in}
Details of the mass and stiffness matrices are shown on the right-hand
side of Figure \ref{fig:solution-shear-building}. From a physical point
of view, element {\tt (i,j)} of the stiffness matrix corresponds to the
nodal force that must be applied to degree of freedom {\tt j} in order
to produce a unit displacement at degree of freedom {\tt i}, and zero
displacements at all other degrees of freedom.

\vspace{0.15 in}
\noindent\hspace{0.5 in}
Figure \ref{fig:solution-eigen-problem} summarizes the
mode shape, circular frequencies and natural period for each of
the modes in the shear building.
The shear building has a fundamental period of 0.5789 seconds.

\vspace{0.10 in}
\begin{figure} [th]
\epsfxsize=6.0truein
\centerline{\epsfbox{my-chapter3-fig6.ps}}
\caption{Mode Shapes and Natural Periods for Shear Building}
\label{fig:solution-eigen-problem}
\end{figure}

\vspace{0.15 in}\noindent
The dynamic behavior of the shear building is generated by the
external force, shown in Figure \ref{fig:applied-force-versus-time},
applied to the roof-level degree of freedom.

\begin{figure} [h]
\epsfxsize=5.0truein
\centerline{\epsfbox{my-chapter3-fig2.ps}}
\caption{Externally Applied Force (kN) versus Time (sec)}
\label{fig:applied-force-versus-time}
\end{figure}

\vspace{0.15 in}\noindent
Notice that we have selected the time-scale of the applied
force so that it has a period close to the first
natural period of the structure (i.e. 0.5789 seconds
versus 0.6 seconds period for the applied load).

\vspace{0.15 in}\noindent
{\bf Input File :} A seven-part input file is needed to define the
mass and stiffness matrices, external loading, and solution
procedure via the method of Newmark Integration.
The step-by-step details of our Newmark Algorithm are:

\begin{description}
\item{[1]}
Form the stiffness matrix {\bf K},
the mass matrix {\bf M}, and the damping matrix {\bf C}.
Compute the effective mass matrix $\hat{\bf M}$.
\item{[2]}
Initialize the displacement ${X}(0)$ and velocity $\dot{X}(0)$ at time $0$.
Backsubstitute ${X}(0)$ and $\dot{X}(0)$ into equation (\ref{eq: equl-1}),
and solve for $\ddot{X}(0)$.
\item{[3]}
Select an integration time step $\triangle t$,
and Newmark parameters $\gamma$ and $\beta$.
\item{[4]}
Enter Main Loop of Newmark Integration.
\item{[5]}
Compute the effective load vector $\hat{\bf P}{(t+\triangle t)}$.
\item{[6]}
Solve equation (\ref{eq: equl3}) for acceleration $\ddot{\bf X}{(t + \triangle t)}$.
\item{[7]}
Compute $\dot{\bf X}{(t + \triangle t)}$ and ${\bf X}{(t + \triangle t)}$
by backsubstituting $\ddot{\bf X}{(t + \triangle t)}$ into the
equations for discrete update in velocity and displacement.
\item{[8]}
Go to Step [4].
\end{description}

\vspace{0.15 in}\noindent
Structural damping in the shear building is ignored.

\vspace{0.15 in}\noindent
{\bf Input File :} The four story shear building, external loading,
and newmark algorithm are defined in a seven-part input file.

\begin{footnotesize}
\vspace{0.20 in}
\noindent
{\rule{2.3 in}{0.035 in} START OF INPUT FILE \rule{2.3 in}{0.035 in} }
\begin{verbatim}
/* [a] : Parameters for Problem Size/Newmark Integration */

   dt     = 0.03 sec;
   nsteps =  200;
   gamma  = 0.50;
   beta   = 0.25;

/* [b] : Form Mass, stiffness and load matrices */

   mass = ColumnUnits( 1500*[ 1, 0, 0, 0;
                              0, 2, 0, 0;
                              0, 0, 2, 0;
                              0, 0, 0, 3], [kg] );

   stiff = ColumnUnits( 800*[ 1, -1,  0,  0;
                             -1,  3, -2,  0;
                              0, -2,  5, -3;
                              0,  0, -3,  7], [kN/m] );

   PrintMatrix(mass, stiff);

/* 
 * [c] : Generate and print external saw-tooth external loading matrix. First and
 *       second columns contain time (sec), and external force (kN), respectively.
 */ 

   myload = ColumnUnits( Matrix([21,2]), [sec], [1]);
   myload = ColumnUnits( myload,          [kN], [2]);

   for(i = 1; i <= 6; i = i + 1) {
       myload[i][1] = (i-1)*dt;
       myload[i][2] = (2*i-2)*(1 kN);
   }

   for(i = 7; i <= 16; i = i + 1) {
       myload[i][1] = (i-1)*dt;
       myload[i][2] = (22-2*i)*(1 kN);
   }

   for(i = 17; i <= 21; i = i + 1) {
       myload[i][1] = (i-1)*dt;
       myload[i][2] = (2*i-42)*(1 kN);
   }

   PrintMatrix(myload);

/* [d] : Initialize working displacement, velocity, and acc'n vectors */

   displ  = ColumnUnits( Matrix([4,1]), [m]    );
   vel    = ColumnUnits( Matrix([4,1]), [m/sec]);
   accel  = ColumnUnits( Matrix([4,1]), [m/sec/sec]);
   eload  = ColumnUnits( Matrix([4,1]), [kN]);

/* 
 * [e] : Allocate Matrix to store three response parameters -- Col 1 = time (sec);
 *       Col 2 = roof displacement (m); Col 3 = Total energy (N.m)
 */ 

   response = ColumnUnits( Matrix([nsteps+1,3]), [sec], [1]);
   response = ColumnUnits( response,  [cm], [2]);
   response = ColumnUnits( response, [Jou], [3]);

/* [f] : Compute (and compute LU decomposition) effective mass */

   MASS  = mass + stiff*beta*dt*dt;
   lu    = Decompose(Copy(MASS));

   for(i = 1; i <= nsteps; i = i + 1) {

    /* [f.1] : Update external load */

       if((i+1) <= 21) then {
          eload[1][1] = myload[i+1][2];
       } else {
          eload[1][1] = 0.0 kN;
       } 

       R = eload - stiff*(displ + vel*dt + accel*(dt*dt/2.0)*(1-2*beta));

    /* [f.2] : Compute new acceleration, velocity and displacement  */

       accel_new = Substitution(lu,R); 
       vel_new   = vel   + dt*(accel*(1.0-gamma) + gamma*accel_new);
       displ_new = displ + dt*vel + ((1 - 2*beta)*accel + 2*beta*accel_new)*dt*dt/2;

    /* [f.3] : Update and print new response */

       accel = accel_new;
       vel   = vel_new;
       displ = displ_new;

    /* [f.4] : Compute "kinetic + potential energy" */

       e1 = Trans(vel)*mass*vel;
       e2 = Trans(displ)*stiff*displ;
       energy = 0.5*(e1 + e2);

    /* [f.5] : Save components of time-history response */

       response[i+1][1] = i*dt;
       response[i+1][2] =  displ[1][1];
       response[i+1][3] = energy[1][1];
   }

/* [g] : Print response matrix and quit */

   PrintMatrix(response);
   quit;
\end{verbatim}
\rule{6.25 in}{0.035 in}
\end{footnotesize}

\vspace{0.15 in}\noindent
Points to note are:

\vspace{0.10 in}
\begin{description}
\item{[1]}
In Part [a] of the input file, the variables {\tt nstep} and {\tt dt}
are initialized for 200 timesteps of 0.03 seconds.
The Newmark parameters $\gamma$
and $\beta$ (i.e. input file variables {\tt gamma} and {\tt beta}) are
set to 0.5 and 0.25, respectively.

\item{[2]}
Details of the external load versus time are generated
for 0.6 seconds, and stored in an array {\tt "myload"}.
Column one of {\tt "myload"} contains time (seconds),
and column two, the externally applied force (kN).
The contents of {\tt "myload"} are then transfered to the external
load vector within the main loop of the Newmark Integration.

\item{[3]}
Rather than print the results of the integration at the end
of each timestep, we allocate memory for {\tt "response"},
a $(201 \times 3)$ matrix, and store relevant details of
the computed response at the end of each iteration of the Newmark method.
The first, second, and third columns of {\tt "response"}
contain the time (sec), displacement of the roof (cm),
and total energy of the system (Joules).

\item{[4]}
Theoretical considerations indicate -- see note above -- that after the
external loading has finished, the sum of kinetic energy plus potential
energy will be constant in the undamped shear structure.
The total energy (Joules) is

\begin{equation}
\hbox{Total Energy} ~=~ 
{1 \over 2}
\left[
{\dot X}(t)^T {\bf M} {\dot X}(t) + {X}(t)^T {\bf K} {X}(t)
\right].
\label{eq:total-energy}
\end{equation}

\item{[5]}
Matrix {\tt "response"} is printed after the completion of the Newmark integration.
This strategy of implementation has several benefits. First, the response 
quantities are easier to interpret if they are bundled together in one array.
The second benefit occurs when program output is redirected to a file.
Upon completion of the analysis, it is an easy process to edit the output file,
and feed {\tt "response"} into MATLAB. This is how we created
Figures \ref{fig: shear-displacement-versus-time} to \ref{fig: modal-results-4}.

\end{description}

\vspace{0.15 in}\noindent
{\bf Abbreviated Output File :}
The output file contains summaries of the mass and stiffness matrices for the shear building,
and abbreviated details of the external loading, {\tt "myload"},
and the response matrix {\tt "response"}.

\vspace{0.15 truein}
\begin{footnotesize}
\noindent
{\rule{1.7 in}{0.035 in} START OF ABBREVIATED OUTPUT FILE \rule{1.7 in}{0.035 in} }
\begin{verbatim}
MATRIX : "mass"

row/col          1            2            3            4          
      units           kg           kg           kg           kg   
   1            1.50000e+03  0.00000e+00  0.00000e+00  0.00000e+00
   2            0.00000e+00  3.00000e+03  0.00000e+00  0.00000e+00
   3            0.00000e+00  0.00000e+00  3.00000e+03  0.00000e+00
   4            0.00000e+00  0.00000e+00  0.00000e+00  4.50000e+03

MATRIX : "stiff"

row/col          1            2            3            4          
      units          N/m          N/m          N/m          N/m   
   1            8.00000e+05 -8.00000e+05  0.00000e+00  0.00000e+00
   2           -8.00000e+05  2.40000e+06 -1.60000e+06  0.00000e+00
   3            0.00000e+00 -1.60000e+06  4.00000e+06 -2.40000e+06
   4            0.00000e+00  0.00000e+00 -2.40000e+06  5.60000e+06

MATRIX : "myload"

row/col          1            2          
      units          sec           kN   
   1            0.00000e+00  0.00000e+00
   2            3.00000e-02  2.00000e+00
   3            6.00000e-02  4.00000e+00

   .... details of "myload" matrix deleted ....

  19            5.40000e-01 -4.00000e+00
  20            5.70000e-01 -2.00000e+00
  21            6.00000e-01  0.00000e+00

MATRIX : "response"

row/col          1            2            3          
      units          sec           cm          Jou   
   1            0.00000e+00  0.00000e+00  0.00000e+00
   2            3.00000e-02  2.69339e-02  2.69339e-01
   3            6.00000e-02  1.51106e-01  3.99451e+00

  .... details of "response" matrix deleted ....

 199            5.94000e+00 -3.04363e+00  6.62976e+02
 200            5.97000e+00 -1.71841e+00  6.62976e+02
 201            6.00000e+00 -2.08155e-02  6.62976e+02
\end{verbatim}
\rule{6.25 in}{0.035 in}
\end{footnotesize}

\vspace{0.15 in}\noindent
Points to note are:

\vspace{0.10 in}
\begin{description}
\item{[1]}
Figures \ref{fig: shear-displacement-versus-time} and \ref{fig: shear-energy-versus-time}
are time-history plots of the roof level displacement,
and total system energy versus time.
Given that the duration of external loading is only 0.6
seconds (see Figure \ref{fig:applied-force-versus-time}),
and that we have deliberately ignored structural damping, 
it is reassuring to see that total energy is conserved exactly
beyond time = 0.6 seconds. You should also observe that the natural
period of vibration in Figure \ref{fig: shear-displacement-versus-time}
closely approximates 0.6 seconds,
the lowest natural period of vibration in the structure.
\end{description}

\clearpage
\begin{figure}[ht]
\epsfxsize= 4.7truein
\centerline{\epsfbox{my-chapter4-fig1.ps}}
\caption{Newmark Integration : Lateral Displacement of Roof (cm) versus Time (sec)}
\label{fig: shear-displacement-versus-time}
\end{figure}

\begin{figure}[h]
\epsfxsize= 4.7truein
\centerline{\epsfbox{my-chapter4-fig2.ps}}
\caption{Newmark Integration : Total Energy (Joules) versus Time (sec)}
\label{fig: shear-energy-versus-time}
\end{figure}

\clearpage
\section{Method of Modal Analysis}

\vspace{0.15 in}
\noindent\hspace{0.50 in}
Let ${\bf \Phi}$ be a $(n \times p)$ nonsingular matrix ($p \le n$),
and Y(t) be a $(p \times 1)$ matrix of time-varying generalized displacements.
The objective of the method of modal analysis is to
find a transformation

\begin{equation}
X(t) = {\bf \Phi} Y(t)
\label{eq:modal-1}
\end{equation}

\vspace{0.15 in}\noindent
that will simplify the direct integration of equations (\ref{eq: equl}).
The modal equations are obtained by substituting
equations (\ref{eq:modal-1}) into (\ref{eq: equl}),
and then premultiplying (\ref{eq: equl}) by $\Phi^T$.
The result is:

\begin{equation}
{\bf \Phi^T} {\bf M} {\bf \Phi} {\ddot Y}(t) +
{\bf \Phi^T} {\bf C} {\bf \Phi} {\dot Y}(t) +
{\bf \Phi^T} {\bf K} {\bf \Phi} {Y}(t) = {\bf \Phi^T} {P}(t).
\label{eq:modal-2}
\end{equation}

\vspace{0.15 in}\noindent
The notation in (\ref{eq:modal-2}) may be
simplified by defining the generalized mass matrix as
${\bf M^*} = {\bf \Phi^T} {\bf M} {\bf \Phi}$,
the generalized damping matrix as
${\bf C^*} = {\bf \Phi^T} {\bf C} {\bf \Phi}$,
the generalized stiffness matrix as
${\bf K^*} = {\bf \Phi^T} {\bf K} {\bf \Phi}$,
and the generalized load matrix as
${\bf P^*}(t) = {\bf \Phi^T} {\bf P}(t)$.
Substituting these definitions into equation (\ref{eq:modal-2}) gives

\begin{equation}
{\bf M^*} {\ddot Y}(t) + {\bf C^*} {\dot Y}(t) +
{\bf K^*} Y(t) = {\bf P^*}(t).
\label{eq:modal-3}
\end{equation}

\vspace{0.15 in}\noindent
The transformation ${\bf \Phi}$ is deemed to be effective if
the bandwidth of matrices in (\ref{eq:modal-3}) is
much smaller than in equations (\ref{eq: equl}). From a
theoretical viewpoint, there may be many transformation matrices
${\bf \Phi}$ which will achieve this objective -- a judicious choice of
transformation matrix will work much better than
many other transformation matrices, however.

\vspace{0.15 in}
\noindent\hspace{0.50 in}
To see how the method of modal analysis works in practice,
consider the free vibration response of an undamped system

\begin{equation}
{\bf M} {\ddot X}(t) + {\bf K} X(t) = 0,
\label{eq:modal-4}
\end{equation}

\vspace{0.15 in}\noindent
where {\bf M} and {\bf K} are $(n \times n)$ mass and stiffness matrices.
We postulate that the time-history response of (\ref{eq:modal-4})
may be approximated by a linear sum of $p$ harmonic solutions

\begin{equation}
X(t) = \sum_{i = 1}^{p} \phi_i \cdot y_i(t) =
\sum_{i = 1}^{p} \phi_i \cdot A_i \cdot \hbox{sin} (w_i t + \beta_i) = {\bf \Phi} \cdot Y(t),
\label{eq:modal-5}
\end{equation}

\vspace{0.15 in}\noindent
where $y_i (t)$ is the ith component of $Y(t)$,
and $\phi_i$ is the ith column of ${\bf \Phi}$.
The amplitude and phase angle for the ith mode are
given by $A_i$ and $\beta_i$, respectively --
both quantities may be determined from
the initial conditions of the motion.
We solve the symmetric eigenvalue problem

\begin{equation}
{\bf K} \Phi = {\bf M} \Phi \Lambda
\label{eq:modal-6}
\end{equation}

\vspace{0.15 in}\noindent
for ${\bf \Phi}$ and
$\Lambda$ is a $(p \times p)$ diagonal matrix of eigenvalues

\begin{equation}
\Lambda = diag(\lambda_1, \lambda_2, \cdots, \lambda_p) = diag( w_1^2, w_2^2, \cdots, w_p^2 ).
\label{eq:modal-7}
\end{equation}

\vspace{0.15 in}\noindent
It is well known that the eigenvectors of problem (\ref{eq:modal-5})
will be orthogonal to both the mass and stiffness matrices.
This means that the generalized mass and stiffness matrices will
have zero terms except for diagonal terms.
The generalized mass matrix takes the form

\begin{equation}
{\bf M^*} = \left[
\begin{array}{rrrr}
{m_1}^* &       0 &   \ldots    &     0   \\
      0 &  {m_2}^*&   \ldots    &     0   \\
 \vdots &   \vdots&   \ddots    & \vdots  \\
      0 &        0&   \ldots    & {m_p}^* \\
\end{array}
\right],
\label{eq:uncoupled-1}
\end{equation}

\vspace{0.05 in}\noindent
and the generalized stiffness looks like

\begin{equation}
{\bf K^*} = \left[
\begin{array}{rrrr}
          {w_1}^2 {m_1}^*&0&\ldots&0     \\
          0& {w_2}^2 {m_2}^*&\ldots&0    \\
          \vdots&\vdots&\ddots&\vdots    \\
          0&0&\ldots&{w_n}^2 {m_p}^*     \\
\end{array}
\right].
\label{eq:uncoupled-2}
\end{equation}

\vspace{0.15 in}\noindent
If the damping matrix, {\bf C}, is a linear combination of
the mass and stiffness matrices, then
the generalized damping matrix ${\bf C^*}$ will also be diagonal.
A format that is very convenient for computation is

\begin{equation}
{\bf C^*} = \left[
\begin{array}{rrrr}
2 \xi_1 w_1 {m_1}^* &                   0 &   \ldots    &     0      \\
                  0 &  2 \xi_2 w_2 {m_2}^*&   \ldots    &     0      \\
             \vdots &               \vdots&   \ddots    & \vdots     \\
                   0&                    0&   \ldots    &2 \xi_n w_n {m_p}^* \\
\end{array}
\right],
\label{eq:uncoupled-3}
\end{equation}

\vspace{0.15 in}\noindent
where $\xi_i$ is the ratio of critical damping for the ith mode of vibration.

\vspace{0.15 in}
\noindent\hspace{0.5 in}
For the undamped vibration of a linear multi-degree of freedom system,
the eigenvalue/vector transformation is ideal because it
reduces the bandwidth of ${\bf M^*}$, ${\bf C^*}$, and ${\bf K^*}$ to 1.
In other words, the eigenvalue/vectors transform
equation (\ref{eq: equl}) from $n$ coupled equations
into $p$ ($p \le n$) uncoupled single degree-of-freedom systems.
The required computation is simplified because the total time-history
response may now be evaluated in two (relatively simple) steps:

\begin{description}
\item{[1]}
Computation of the time-history responses for each
of the $p$ single degree-of-freedom systems, followed by
\vspace{0.05 in}
\item{[2]}
Combination of the SDOF's responses into
the time-history response of the complete structure.
\end{description}

\vspace{0.15 in}\noindent
A number of computational methods are available to compute
the time variation of displacements in each
of the single degree of freedom systems. From a theoretical
viewpoint, it can be shown that the total
solution (or general solution) for a damped system is given by

\begin{equation}
y_i (t) = B_i (t) \sin( w_{id} t) + C_i (t) \cos( w_{id} t) 
\label{eq:uncoupled-5}
\end{equation}

\vspace{0.15 in}\noindent
where $w_{id} = w_i \sqrt{1 - \xi_i^2}$ is the damped circular frequency of
vibration for the ith mode.
The time-variation in coefficients $B_i(t)$ and $C_i(t)$ is given by

\begin{eqnarray}
B_i (t) &= & e^{-\xi w_i t}
\left[ {{{\dot y}(0) + \xi w_i y(0)}\over w_d} +
{1 \over m_i^* w_{id}} \int_0^{t} P(\tau)
e^{\xi_i w_i \tau} \cos(w_{id} \tau) d\tau \right] \\
\hbox{and}\qquad
C_i (t) &= & e^{-\xi_i w_i t} \left[ y(0) - {1 \over m_i^* w_{id}}
\int_0^{t} P(\tau) e^{\xi_i w_i \tau} \sin(w_{id} \tau) d\tau \right].
\end{eqnarray}

\vspace{0.15 in}\noindent
where $y_i(0)$ and $\dot{y}_i(0)$ are the
initial displacement and velocity for the ith mode.

\vspace{0.15 in}
\noindent\hspace{0.5 in}
If the details of $P(\tau)$ are simple enough,
then analytic solutions to (\ref{eq:uncoupled-5}) may be possible.
For most practical problems (e.g. earthquake ground motions),
however, numerical solutions to $B_i(t)$ and $C_i(t)$ must be relied upon.
A second approach, and the one that will be followed here,
is to use Newmark Integration to compute the
displacements for each of the individual modes.

\vspace{0.20 in}\noindent
{\bf Numerical Problem :}
We demonstrate the method of modal analysis by repeating the
time-history computation defined in the previous section.
Details of the shear building and external loading are shown in
Figures \ref{fig:solution-shear-building} and \ref{fig:applied-force-versus-time}.

\vspace{0.15 in}\noindent
{\bf Input File :} The input file for modal analysis is a little
more involved than the input file needed for Newmark Integration.
This is due in part, to need to transform the equations of equilibrium to a
simplier coordinate frame before the main loops of modal analysis begin.

\vspace{0.15 in}
\noindent\hspace{0.5 in}
In the input file that follows, $(2 \times 2)$ generalized mass and
stiffness matrices are generated by first computing the $(4 \times 2)$
transformation matrix ${\bf \Phi}$ corresponding to the
first two eigenvectors in the shear building.
The elements of the generalized mass and stiffness
matrices are zero, except for diagonal terms.
In principle, each of the decoupled equations may be solved
as single degree-of-freedom systems; in practice, however,
it is computationally simplier to use the method of Newmark
integration to solve both sets of decoupled equations together.

\vspace{0.20 in}
\begin{footnotesize}
\noindent
{\rule{2.3 in}{0.035 in} START OF INPUT FILE \rule{2.3 in}{0.035 in} }
\begin{verbatim}
/* [a] : Parameters for Modal Analysis and embedded Newmark Integration */

   no_eigen = 2;
   dt       = 0.03 sec;
   nsteps   = 200;
   beta     = 0.25;
   gamma    = 0.50;

/* [b] : Form Mass and stiffness matrices */

   mass = ColumnUnits( 1500*[ 1, 0, 0, 0;
                              0, 2, 0, 0;
                              0, 0, 2, 0;
                              0, 0, 0, 3], [kg] );

   stiff = ColumnUnits( 800*[ 1, -1,  0,  0;
                             -1,  3, -2,  0;
                              0, -2,  5, -3;
                              0,  0, -3,  7], [kN/m] );

   PrintMatrix(mass, stiff);

/* [c] : First two eigenvalues, periods, and eigenvectors */

   eigen       = Eigen(stiff, mass, [no_eigen]);
   eigenvalue  = Eigenvalue(eigen);
   eigenvector = Eigenvector(eigen);

   for(i = 1; i <= no_eigen; i = i + 1) {
       print "Mode", i ," : w^2 = ", eigenvalue[i][1];
       print " : T = ", 2*PI/sqrt(eigenvalue[i][1]) ,"\n";
   }

   PrintMatrix(eigenvector);

/* [d] : Generalized mass and stiffness matrices */

   EigenTrans = Trans(eigenvector);
   Mstar   = EigenTrans*mass*eigenvector;
   Kstar   = EigenTrans*stiff*eigenvector;

   PrintMatrix( Mstar );
   PrintMatrix( Kstar );

/* 
 * [e] : Generate and print external saw-tooth external loading matrix. First and
 *       second columns contain time (sec), and external force (kN), respectively.
 */ 

   myload = ColumnUnits( Matrix([21,2]), [sec], [1]);
   myload = ColumnUnits( myload,          [kN], [2]);

   for(i = 1; i <= 6; i = i + 1) {
       myload[i][1] = (i-1)*dt;
       myload[i][2] = (2*i-2)*(1 kN);
   }

   for(i = 7; i <= 16; i = i + 1) {
       myload[i][1] = (i-1)*dt;
       myload[i][2] = (22-2*i)*(1 kN);
   }

   for(i = 17; i <= 21; i = i + 1) {
       myload[i][1] = (i-1)*dt;
       myload[i][2] = (2*i-42)*(1 kN);
   }

   PrintMatrix(myload);

/* [f] : Initialize system displacement, velocity, and load vectors */

   displ  = ColumnUnits( Matrix([4,1]), [m]    );
   vel    = ColumnUnits( Matrix([4,1]), [m/sec]);
   eload  = ColumnUnits( Matrix([4,1]), [kN]);

/* [g] : Initialize modal displacement, velocity, and acc'n vectors */

   Mdispl  = ColumnUnits( Matrix([ no_eigen,1 ]), [m]    );
   Mvel    = ColumnUnits( Matrix([ no_eigen,1 ]), [m/sec]);
   Maccel  = ColumnUnits( Matrix([ no_eigen,1 ]), [m/sec/sec]);

/* 
 * [g] : Allocate Matrix to store five response parameters --
 *       Col 1 = time (sec);
 *       Col 2 = 1st mode displacement (cm);
 *       Col 3 = 2nd mode displacement (cm);
 *       Col 4 = 1st + 2nd mode displacement (cm);
 *       Col 5 = Total energy (Joules)
 */ 

   response = ColumnUnits( Matrix([nsteps+1,5]), [sec], [1]);
   response = ColumnUnits( response,  [cm], [2]);
   response = ColumnUnits( response,  [cm], [3]);
   response = ColumnUnits( response,  [cm], [4]);
   response = ColumnUnits( response, [Jou], [5]);

/* [h] : Compute (and compute LU decomposition) effective mass */

   MASS  = Mstar + Kstar*beta*dt*dt;
   lu    = Decompose(MASS);

/* [i] : Mode-Displacement Solution for Response of Undamped MDOF System  */

   for(i = 1; i <= nsteps; i = i + 1) {

    /* [i.1] : Update external load */

       if((i+1) <= 21) then {
          eload[1][1] = myload[i+1][2];
       } else {
          eload[1][1] = 0.0 kN;
       } 

       Pstar = EigenTrans*eload;
       R = Pstar - Kstar*(Mdispl + Mvel*dt + Maccel*(dt*dt/2.0)*(1-2*beta));

    /* [i.2] : Compute new acceleration, velocity and displacement  */

       Maccel_new = Substitution(lu,R); 
       Mvel_new   = Mvel   + dt*(Maccel*(1.0-gamma) + gamma*Maccel_new);
       Mdispl_new = Mdispl + dt*Mvel + ((1 - 2*beta)*Maccel + 2*beta*Maccel_new)*dt*dt/2;

    /* [i.3] : Update and print new response */

       Maccel = Maccel_new;
       Mvel   = Mvel_new;
       Mdispl = Mdispl_new;

    /* [i.4] : Combine Modes */

       displ = eigenvector*Mdispl;
       vel   = eigenvector*Mvel;

    /* [i.5] : Compute Total System Energy */

       e1 = Trans(vel)*mass*vel;
       e2 = Trans(displ)*stiff*displ;
       energy = 0.5*(e1 + e2);

    /* [i.6] : Save components of time-history response */

       response[i+1][1] = i*dt;                            /* Time                  */
       response[i+1][2] = eigenvector[1][1]*Mdispl[1][1];  /* 1st mode displacement */
       response[i+1][3] = eigenvector[1][2]*Mdispl[2][1];  /* 2nd mode displacement */
       response[i+1][4] = displ[1][1];               /* 1st + 2nd mode displacement */
       response[i+1][5] = energy[1][1];                    /* System Energy         */
   }

/* [j] : Print response matrix and quit */

   PrintMatrix(response);
   quit;
\end{verbatim}
\rule{6.25 in}{0.035 in}
\end{footnotesize}

\vspace{0.15 in}\noindent
Points to note are:

\vspace{0.10 in}
\begin{description}
\item{[1]}
In Part [a] of the input file, the variables {\tt nstep} and {\tt dt}
are initialized for 200 timesteps of 0.03 seconds.
The Newmark parameters $\gamma$
and $\beta$ (i.e. input file variables {\tt gamma} and {\tt beta}) are
set to 0.5 and 0.25, respectively.
\item{[2]}
Overall behavior of the system is represented
by $(4 \times 4)$ global mass and stiffness matrices. 
A $(4 \times 2)$ transformation matrix, $\Phi$, is computed by
solving equation (\ref{eq:modal-6}) for the first two eigenvectors.
As a result, we expect that the generalized mass and
stiffness will be $(2 \times 2)$ diagonal matrices.
\item{[3]}
The main loop of our modal analysis employs Newmark integration to
compute the time-history response for the two decoupled equations.
\end{description}

\vspace{0.15 in}\noindent
{\bf Abbreviated Output File :}
The abbreviated output file contains only the essential details
of output generated by the modal analysis.
These details include the first two mode shapes of the shear building,
the generalized mass and stiffness matrices,
and heavily edited versions of matrices {\tt "myload"} and {\tt "response"}.

\vspace{0.20 truein}
\begin{footnotesize}
\noindent
{\rule{1.7 in}{0.035 in} START OF ABBREVIATED OUTPUT FILE \rule{1.7 in}{0.035 in} }
\begin{verbatim}
MATRIX : "mass"

   .... see Newmark output for details of "mass" ....

MATRIX : "stiff"

   .... see Newmark output for details of "stiff" ....

*** SUBSPACE ITERATION CONVERGED IN 11 ITERATIONS 

Mode    1.0000e+00  : w^2 =    117.8 rad.sec^-2.0 : T =   0.5789 sec
Mode    2.0000e+00  : w^2 =    586.5 rad.sec^-2.0 : T =   0.2595 sec

MATRIX : "eigenvector"

row/col          1            2          
      units                       
   1            1.00000e+00  1.00000e+00
   2            7.79103e-01 -1.00304e-01
   3            4.96553e-01 -5.39405e-01
   4            2.35062e-01 -4.36791e-01

MATRIX : "Mstar"

row/col          1            2          
      units           kg           kg   
   1            4.30934e+03  5.68434e-14
   2            1.70530e-13  3.26160e+03

MATRIX : "Kstar"

row/col          1            2          
      units          N/m          N/m   
   1            5.07691e+05 -1.96451e-10
   2           -2.91038e-10  1.91282e+06

MATRIX : "myload"

row/col          1            2          
      units          sec           kN   
   1            0.00000e+00  0.00000e+00
   2            3.00000e-02  2.00000e+00

   ...... details of external load removed ....

  20            5.70000e-01 -2.00000e+00
  21            6.00000e-01  0.00000e+00

MATRIX : "response"

row/col          1            2            3            4            5          
      units          sec           cm           cm           cm          Jou   
   1            0.00000e+00  0.00000e+00  0.00000e+00  0.00000e+00  0.00000e+00
   2            3.00000e-02  1.01728e-02  1.21886e-02  2.23614e-02  2.23614e-01

   ...... details of response matrix removed ....

 200            5.97000e+00 -1.26509e+00 -3.65074e-01 -1.63016e+00  6.59154e+02
 201            6.00000e+00  3.48910e-01 -3.36893e-01  1.20176e-02  6.59154e+02
\end{verbatim}
\rule{6.25 in}{0.035 in}
\end{footnotesize}

\vspace{0.15 in}\noindent
Points to note are:

\vspace{0.10 in}
\begin{description}
\item{[1]}
Figures \ref{fig: modal-results-1} and \ref{fig: modal-results-2} 
show the time-history response for the first and second modes, respectively.
Notice that the amplitude of vibration for the first mode is an order of
magnitude larger than for the second mode. You should also observe that
after the external load finishes at time = 0.6 seconds,
the amplitude of vibration is constant within each mode,
with the natural periods of vibration closely matching the eigenvalues/periods
shown in Figure \ref{fig:solution-eigen-problem}.
\item{[2]}
The combined first + second modal response versus
time is shown in Figure \ref{fig: modal-results-3}.
It time-displacement curve is virtually identical to
that computed with the method of Newmark integration.
\item{[3]}
The time variation in total energy versus time
is shown in Figure \ref{fig: modal-results-4}.
One benefit of embedding Newmark inside the modal equations is that
energy is conserved for each if the modes. For the integration
beyond time = 0.6 seconds, the total energy is conserved at 659 Joules.
This quantity compares to 662 Joules for the complete Newmark computation.
\end{description}

\clearpage
\begin{figure}[ht]
\epsfxsize= 4.7truein
\centerline{\epsfbox{my-chapter4-fig3.ps}}
\caption{Modal Analysis : First Mode Displacement of Roof (cm) versus Time (sec)}
\label{fig: modal-results-1}
\end{figure}

\begin{figure}[h]
\epsfxsize= 4.7truein
\centerline{\epsfbox{my-chapter4-fig4.ps}}
\caption{Modal Analysis : Second Mode Displacement of Roof (cm) versus Time (sec)}
\label{fig: modal-results-2}
\end{figure}

\clearpage
\begin{figure}[ht]
\epsfxsize= 4.7truein
\centerline{\epsfbox{my-chapter4-fig5.ps}}
\caption{Modal Analysis : First + Second Mode Displacement of Roof (cm) versus Time (sec)}
\label{fig: modal-results-3}
\end{figure}

\begin{figure}[h]
\epsfxsize= 4.7truein
\centerline{\epsfbox{my-chapter4-fig6.ps}}
\caption{Modal Analysis : Total Energy (Joules) versus Time (sec)}
\label{fig: modal-results-4}
\end{figure}

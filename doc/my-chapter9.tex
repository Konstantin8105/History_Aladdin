
\chapter{Conclusions and Future Work}

\section{Conclusions}

\vspace{0.15 in}
\noindent\hspace{0.5 in}
In the general area of computing and engineering computations,
the 1990's are shaping up to be the decade of global
networking, multimedia, and hand-held computers.
There is now a general trend towards to development of
problem-solving environments that will
provide all of the computational needs for a target class of problems,
including discrete-event and continuous modeling,
possibly at multiple levels of abstraction.
These environments will be flexible enough so that
engineers can incorporate new solution methods in their computations,
and bring computational procedures from once
disparate disciplines together in a natural way.
While the merging of optimization with finite elements,
finite element analysis with graphics,
and computers with communications are unions that are already evident,
many other possibilities exist.
For instance, now that many companies are engaging in 
business practices that span the global marketplace,
it seems only a matter of time before finite element computations will be combined
with design rule checking (or requirements driven development) over the Internet.
These environments will employ modern computer technologies
such as computer graphics, distributed and parallel processors,
and provide support for rapid prototyping ~\cite{gallopoulos94,tesler91}.

\vspace{0.15 in}
\noindent\hspace{0.5 in}
With this vision in mind, our long-term goal for ALADDIN is to provide engineers with
a computational problem-solving environment that is
significantly better than is currently available.
Version 1.0 of ALADDIN is simply a first step in this direction.
The emphasis in development has so far been on the basic system specification,
and its application to matrix and finite element problems.
We have tried to determine what data types, control structures,
and functions will allow engineers to solve a wide variety of problems, 
without the language becoming too large and complicated,
and without extensibility of the system being compromised.
Our strategy for achieving the latter objective is described in Chapter 1.
The test applications have been taken from traditional Civil Engineering,
and have focussed on the design and analysis of
highway bridges and earthquake-resistant buildings.

\vspace{0.15 in}
\noindent\hspace{0.5 in}
ALADDIN (Version 1.0) is approximately 36,000 lines of C,
and getting the program to this point has taken
approximately three years of part-time work.
We have written nearly all of the source code from scratch -- along the way,
a new data structure for the matrices with units was designed and software implemented.
Many of the matrix algorithms that ALADDIN uses are documented
in the forthcoming text of Austin and Mazzoni ~\cite{austin95}.
The language design must be followed by entensive testing of the software
with matrix and finite element test problems that are complicated
enough to push the limits of the software (Note : In our experience,
many coding errors will not show up in trivial engineering applications).
As the work progressed, we occasionally identified small extensions to
the language that would allow ALADDIN to attack a new problem area.
When the language for ALADDIN was first designed
we did not have design rule checking in mind, for example.
Yet once the basic matrix and function language specification was in place, 
and the finite element analyses worked, only a couple of extra functions
were needed for the implementation of basic design rule checking.

\vspace{0.15 in}
\noindent\hspace{0.5 in}
Together these observations lead us to believe that
our goal of keeping the ALADDIN language small and extensible has been achieved.
This work lays the foundation for many avenues of future work.

\section{Future Work}

\vspace{0.15 in}
\noindent\hspace{0.5 in}
Figure \ref{fig: aladdin-future-development} is a schematic
of the current components of work, plus modules of work
that are needed for the next extension of ALADDIN.
The future work should focus on:

\begin{figure}[ht]
\epsfxsize=6.0truein
\centerline{\epsfbox{my-chapter9-fig1.ps}}
\caption{Future Development for ALADDIN}
\label{fig: aladdin-future-development}
\end{figure}

\begin{description}
\item{[1]}
{\tt Nonlinear Algorithms :}
The scope of applications in this report has been restricted 
to time-dependent analysis of linear structural systems.
Future work should extend this work to the time-history analysis of systems
having material and geometric nonlinearities.
A first step in this direction is reported in Chen and Austin ~\cite{chen95}.

\item{[2]}
{\tt User-defined Functions :}
Version 1.0 of ALADDIN does not permit the use of
user-defined functions and procedures in the input file.
When we first began working on ALADDIN,
our feeling was that the input file language should be
relatively simple -- otherwise, why not simply code the whole problem in C ?
An easy way of controling language complexity
is to disallow user-defined functions in the input file,
and this is what we have done for ALADDIN Version 1.0.
Now that the matrix and finite element libraries are working,
this early decision needs to be revised.
We would like to combine the finite element
analysis with discrete simulated annealing optimization algorithms,
along the lines proposed by Kirkpatrick ~\cite{kirkpatrick83}.
The latter could be succinctly implemented if user-defined functions were available. 
The current version of ALADDIN assumes that all user-defined
variables are global.  When user-defined functions are added to ALADDIN,
notions of scope should also be added to variables (this
could be implemented via an array of Symbol Tables).

\item{[3]}
{\tt Graphical Library :}
An obvious limitation of ALADDIN (Version 1.0) is that it
lacks a graphical user interface. Providing a suitable interface
is easier said than done. If ALADDIN simply preported to conduct
finite element analysis, then it would be a relatively easy
process to code-up an interactive pre- and post-finite
element processor in X windows (or a suitable PC graphics package).

The problem with this approach is that it only serves one
half of ALADDIN's capabilities -- in addition to
graphically describing the finite element mesh and modeling
assumptions, a graphical interface should also provide a graphical
counterpart to the solution algorithm specifications (e.g. Newmark Integration).
We will explore the use of visual programming languages
to see if they are suitable for this purpose.

\item{[4]}
{\tt Optimization Library :}
We would like to extend the language specification so that engineers
can describe design objectives, design constraints, and design parameters,
for general engineering optimization problems in a compact manner.
The issue of ``compactness'' is particularly relevant for optimization
problems that involve finite element analysis because they often
contain hundreds, and sometimes thousands, of constraints.

Further work is needed to determine how groups of similar constraints
can be bundled into groups, and how ALADDIN's control structures
can be exploited to write down these constraints in a compact manner.
Once the issue of constraints is sorted out, the pathway for
describing the design parameters and objectives should (hopefully) become clearer.

\item{[5]}
{\tt Test Problem Suite :} New suites of test problem files will be
needed as the details of items [2] and [4] are filled in (please
let us know if you find an error in one of the test files!).

\item{[6]}
{\tt WWW Home Page :} We are currently developing a
World Wide Web (WWW) Home Page for ALADDIN. The ALADDIN home page
will provide basic information on ALADDIN,
sample matrix and finite element test problems,
ALADDIN source code, and copies of this report.
Our near-term plans are to experiment with the dynamic documents
capabilities of WWW, and create home pages where WWW visitors
can interact with an ALADDIN script running at the University of Maryland.

\end{description}

\vspace{0.15 in}\noindent
Work items [1] to [5] are tightly coupled.
Understanding their interaction is an active area of research.

\vspace{0.25 in}\noindent
{\bf Acknowledgements}

\vspace{0.15 in}
\noindent\hspace{0.50 in}
This work has been supported in part by NSF Grant ECD-8803012,
by NSF grants to the Institute for Systems Research at the University of Maryland,
and by a series of fellowship grants to Wane-Jang Lin from the Federal Highway Administration.

\vspace{0.15 in}
\noindent\hspace{0.50 in}
The early stages of this project would not have been possible without
valuable discussions with Dr. Bunu Alibe and Bert Davy,
faculty members in the Department of Civil Engineering at Morgan State University. 

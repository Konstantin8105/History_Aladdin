\chapter {Finite Element Analysis Language}

\section{Introduction}

\vspace{0.15 in}
\noindent\hspace{0.5 in}
In this chapter we describe ALADDIN's built-in functions for
the solution of finite element problems.
Built-in function are provided for:
(a) the generation of finite element meshes,
(b) the definition of external loads,
(c) the specification of boundary conditions,
(d) the specification of section and material properties,
and (e) linking finite element degrees of freedom.

\vspace{0.15 in}
\noindent\hspace{0.5 in}
With the finite element mesh in place, we can then compute
solutions to a particular linear/nonlinear finite elment problem
using the algorithms described in the previous chapters.
To assist the engineer with basic finite element computations,
built-in functions are also provided for the assembly of
global stiffness matrices, global mass matrices, and external load vectors.
An engineer may query the finite element database
for information on the finite element mesh,
section and material properties,
and on the computed displacements and stresses.
Together, these facilities provide an engineer with the means to
write ALADDIN input files that will compare the performance
of a structure against families of design rules.

\section{Structure of Finite Element Input Files}
\label{section-structure-of-input-files}

\vspace{0.15 in}
\noindent\hspace{0.5 in}
The following diagram is a schematic of the main
components in an ALADDIN input file suitable
for the specification and solution of a finite element problem.

\vspace{0.25 in}
\begin{footnotesize}
\noindent
{\rule{2.3 in}{0.035 in} START OF INPUT FILE \rule{2.3 in}{0.035 in} }
\begin{verbatim}
/*
 *  ========================================================
 *  A description of the finite element problem goes here...
 *  ========================================================
 */

  ... Part [1] : Problem specification parameters.

StartMesh();

  ... Part [2] : Generate finite element mesh. Specify section and material
                 properties, external loads, and boundary conditions.

EndMesh();

  ... Part [3] : Describe solution procedure for finite element problem.

  ... Part [4] : If applicable, check performance of structure against
                 design rules.

  ... Part [5] : If applicable, generate arrays of output that are suitable
                 for plotting with MATLAB.

quit;
\end{verbatim}
\rule{6.25 in}{0.035 in}
\end{footnotesize}

\vspace{0.25 in}\noindent
Every ALADDIN input file should begin with a
description of the file's purpose, and who wrote it.

\vspace{0.15 in}
\noindent\hspace{0.5 in}
The problem specification parameters allow an
engineer to state whether a finite element
problem will be two- or three- dimensional,
the maximum number of degree of freedom per node,
and the maximum number of nodes per element.
With this information in place,
{\tt StartMesh()} allocates the working memory for the
finite element data structures.

\vspace{0.15 in}
\noindent\hspace{0.5 in}
ALADDIN statements are written for the finite element mesh generation,
for the section and material properties, the specification of
external loads, and for the boundary conditions in Part [2] of the input file.
Often the overall size of a finite element model can be
reduced by linking degrees of freedom -- functions
to link nodal degrees of freedom are also specified in Part [2].
{\tt Endmesh()} loads the information provided in Part [2]
into the finite element database. 

\vspace{0.15 in}
\noindent\hspace{0.5 in}
Part [3] usually begins with the assembly of the
global stiffness matrix, an external vectors,
and if applicable, assembly of a global mass matrix.
The algorithms presented in the previous chapters (e.g.
newmark integration; modal analysis; optimization procedures)
can be inserted here to solve a specific linear/nonlinear
static/dynamic finite element problem.

\vspace{0.15 in}
\noindent\hspace{0.5 in}
In Parts [4] and [5] ALADDIN statements
are written for design rule checking,
and for the output of arrays that are suitable for plotting with MATLAB.
ALADDIN input file should be
terminated with the command {\tt quit}.

\clearpage
\section{Problem Specification Parameters}

\vspace{0.15 in}\noindent
ALADDIN's specification parameters are:

\begin{itemize}
\item{}
{\tt NDimension} is short for ``number of dimensions.''
For two- and three-dimensional finite element analyses,
{\tt NDimension} equals 2 and 3, respectively.
\item{}
{\tt NDofPerNode} is short for Maximum number of degrees of
freedom per nodal coordinate.
When {\tt NDimension} equals 2, {\tt NDofPerNode} will take the value 3.
And when {\tt NDimension} equals 3, {\tt NDofPerNode} will equal 6.
\item{}
{\tt MaxNodesPerElement} corresponds to the
maximum number of nodes per finite element.
\item{}
{\tt InPlaneIntegPts} corresponds to the number of
in-plane integration points for shell finite elements.
\item{}
{\tt ThicknessIntegPts} number of layers of integration
points through thickness direction of shell finite elements.
\end{itemize}

\vspace{0.10 in}\noindent
These parameter settings are used in the allocation of
memory for the finite element mesh.

\vspace{0.15 in}\noindent
{\bf Short Example :} In this short example, we initialize
the problem specification parameters for a three-dimensional
analysis of a structure with an eight-node shell finite element.

\vspace{0.15 in}
\begin{footnotesize}
\noindent
{\rule{2.1 in}{0.035 in} ABBREVIATED INPUT FILE \rule{2.1 in}{0.035 in} }
\begin{verbatim}

/* [a] : Define Problem Specific Parameters

   NDimension         = 3;
   NDofPerNode        = 5;
   MaxNodesPerElement = 8;   /* .... etc ..... */
\end{verbatim}
\rule{6.25 in}{0.035 in}
\end{footnotesize}

\section{Adding Nodes and Finite Elements}

\vspace{0.15 in}
\noindent\hspace{0.5 in}
Two functions, {\tt AddNode()} and {\tt AddElement()}, are employed
for the generation of finite element nodal coordinates,
and the attachment of finite elements to the nodes.

\begin{itemize}
\item{}
{\tt AddNode( nodeno} {\tt , coord\_vector);} :
Here {\tt nodeno} is the node number in the finite element mesh.
For two-dimensional finite element problems,
{\tt coord\_vector} is a $(1 \times 2)$ matrix
containing the [x, y] nodal coordinates.
When the finite element problem is three-dimensional,
{\tt coord\_vector} is a $(1 \times 3)$ matrix
containing the [x, y, z] nodal coordinates.

\item{}
{\tt AddElmt( elmtno} {\tt , connect\_vector, "name\_of\_elmt\_attr");} :
Here {\tt elmtno} is the element number in the finite element mesh.
{\tt connect\_vector} is a $(1 \times n)$ matrix
containing a list of $n$ nodes to which the finite element will be attached.
{\tt "name\_of\_elmt\_attr"} is the attribute name representing the
finite element's material and section properties --
for details, see Section \ref{ref:material-and-section}.
\end{itemize}

\vspace{0.15 in}\noindent
{\bf Short Example :} Figure \ref{fig:fe-cantilever-mesh} shows
a two-dimensional coordinate system, and a line of six finite element
nodes connected by 2-node beam finite elements.
The nodes are located at y-coordinate = 1 $m$,
and are spaced along the x-axis at 1 $m$ centers,
beginning at x = 1 $m$ and finishing at x = 6 $m$.

\begin{figure} [ht]
\vspace{0.10 in}
\epsfxsize=5.5truein
\centerline{\epsfbox{my-chapter5-fig6.ps}}
\caption{Line of Nodal Coordinates and Beam Finite Elements}
\label{fig:fe-cantilever-mesh}
\end{figure}

\vspace{0.15 in}
\noindent\hspace{0.5 in}
ALADDIN's looping constructs are ideally suited for 
specification of the finite element nodes in a compact manner,
and for the attachment of the two-node finite elements.

\vspace{0.15 in}
\begin{footnotesize}
\noindent
{\rule{2.1 in}{0.035 in} ABBREVIATED INPUT FILE \rule{2.1 in}{0.035 in} }
\begin{verbatim}
   print "*** Generate grid of nodes for finite element model \n\n";

   nodeno = 0;
   x = 1 m; y = 1 m;
   while(x <= 6 m) {
       nodeno = nodeno + 1;
       AddNode(nodeno, [x, y]);
       x = x + 1 m;
   }

   print "*** Attach finite elements to nodes \n\n";

   elmtno = 0; nodeno = 0;
   while(elmtno < 5) {
       elmtno = elmtno + 1; nodeno = nodeno + 1;
       AddElmt( elmtno, [nodeno, nodeno + 1], "name_of_elmt_attr");
   }
\end{verbatim}
\rule{6.25 in}{0.035 in}
\end{footnotesize}

\vspace{0.15 in}\noindent
In the first half of the script, 6 nodal coordinates are added to ALADDIN's database.
Notice how, in the second block of code,
we have used the notation {\tt [nodeno, nodeno + 1]} to
generate $(1 \times 2)$ matrices containing the
node numbers that {\tt elmtno} will be attached to.

\section{Material and Section Properties}
\label{ref:material-and-section}

\vspace{0.15 in}
\noindent\hspace{0.5 in}
The attributes of element, section, and material types are
specified with three functions,
{\tt ElementAttr()}, {\tt SectionAttr()} and {\tt MaterialAttr()},
followed by parameters inserted between braces ${\lbrace \cdots \rbrace}$.
The details of each function are:

\begin{itemize}
\item{}
{\tt ElementAttr("name\_of\_element\_attr")} ${\lbrace \cdots \rbrace};$.
Here {\tt "name\_of\_elmt\_attr"} is a character string for the
name of the element attribute used in function calls to {\tt AddElmt($\cdots$)}.
Three character string arguments should be specified between the braces.

\vspace{0.10 in}
{\tt type} is a name for the finite element type (e.g. FRAME\_2D for
two dimensional frame elements). A list of finite element types
may be found in Section ~\ref{section-finite-element-library}.

\vspace{0.10 in}
Character strings for the {\tt section} and {\tt material} attributes
may taken from the section file {\tt section.h}
and the material file {\tt material.h}, or may be user-defined
with {\tt SectionAttr("...")} and {\tt MaterialAttr("...")}.

\item{}
{\tt SectionAttr("name\_of\_section\_attr")} ${\lbrace \cdots \rbrace};$.
Here {\tt "name\_of\_section\_attr"} is a character
string for the name of the section attribute.
Table ~\ref{tab: my-section-properties} contains a list of
section property names, their meaning, and the units associated with each property.

\item{}
{\tt MaterialAttr("name\_of\_material\_attr")} ${\lbrace \cdots \rbrace};$.
Here {\tt "name\_of\_material\_attr"} is a character
string for the name of the material attribute.
Table ~\ref{tab: my-material-properties} contains a list of
material property names, their meaning,
and the units associated with each property.
\end{itemize}

\vspace{0.15 in}\noindent
When the number of (local) degrees of freedom in a finite element node is fewer
than the number of (global) degrees of freedom at the node to which it is attached,
a mapping must be provided from the local to global d.o.f.
The syntax for such a mapping is

\begin{footnotesize}
\begin{verbatim}
       map ldof [A] to gdof [B].
\end{verbatim}
\end{footnotesize}

\vspace{0.15 in}\noindent
where {\tt [A]} and {\tt [B]} are {\tt $(1 \times n)$} matrices.
We will see, for example, that when plate finite
elements ({\tt type} {\tt =} {\tt PLATE}) are employed in a three-dimensional mesh,
a suitable mapping is {\tt map} {\tt ldof} {\tt [1,2,3]} {\tt to} {\tt gdof} {\tt [2,4,6]}.
This specification is provided as a fourth entry to {\tt ElementAttr()}.

\vspace{0.15 in}\noindent
{\bf Short Example :} The following script loads a finite element attribute
called ``floorelmts'' into ALADDIN's database. The ``floorelmts'' attribute
has three components -- the finite element type is set to {\tt FRAME\_2D},
for two-dimensional beam column finite elements. The element's section
and material properties are defined via links to the section
attribute ``floorsection,'' and the material attribute ``floormaterial.''

\vspace{0.15 in}
\begin{footnotesize}
\noindent
{\rule{2.1 in}{0.035 in} ABBREVIATED INPUT FILE \rule{2.1 in}{0.035 in} }
\begin{verbatim}
print "*** Define element, section, and material properties \n\n";

ElementAttr("floorelmts") { type     = "FRAME_2D";
                            section  = "mysection";
                            material = "ELASTIC";
                          }

SectionAttr("floorsection") { Ixy     = 1 m^4;
                              Iyy     = 2 m^4;
                              Ixx     = 3 m^4;
                              Izz     = 0.66666667 m^4;
                              depth   =   2 m;
                              width   =   1.5 m;
                            }

MaterialAttr("floormaterial") {  E    = 1E+7 kN/m^2;
                                 density = 0.1024E-5 kg/m^3;
                                 poisson = 1.0/3.0;
                                 yield   = 36000 psi;
                              }
\end{verbatim}
\rule{6.25 in}{0.035 in}
\end{footnotesize}

\vspace{0.15 in}\noindent
You should notice that the section and material properties
components are supplied, as required by the {\tt FRAME\_2D}
element specifications described in Section \ref{section-finite-element-library}.

\begin{table}[t]
\tablewidth = 6.25truein
\begintable
Name          | \para{Description}                         | Units ($M^{\alpha}.L^{\beta}.T^{\gamma}$) \crthick
area          | \para{ Area of cross section.}             | $L^2$         \cr
depth         | \para{ Depth of section.}                  | $L$           \cr
thickness     | \para{ Thickness of section.}              | $L$           \cr
Ixx           | \para{ Moment of inertia about x-x axis.}  | $L^4$         \cr
Iyy           | \para{ Moment of inertia about y-y axis.}  | $L^4$         \cr
Izz           | \para{ Moment of inertia about z-z axis.}  | $L^4$         \cr
Ixy and Iyx   | \para{ Moment of inertia about x-y axes.}  | $L^4$         \cr
Ixz and Izx   | \para{ Moment of inertia about x-z axes.}  | $L^4$         \cr
Iyz and Izy   | \para{ Moment of inertia about y-z axes.}  | $L^4$         \cr
rT            | \para{ Radius of Gyration.}                | $L$           \cr
tw            | \para{ Width of web.}                      | $L$           \cr
tf            | \para{ Width of flange.}                   | $L$           \cr
J             | \para{ Torsional Constant.}                | $L^4$         \cr
width         | \para{ Width of section.}                  | $L$    
\endtable
\vspace{0.01 in}
\caption{\bf Section Properties}
\label{tab: my-section-properties}
\end{table}

\begin{table}[ht]
\tablewidth = 6.25truein
\begintable
Name          | \para{Description}                         | Units ($M^{\alpha}.L^{\beta}.T^{\gamma}$) \crthick
E                 | \para{Young's modulus of elasticity}  | $M^1 \cdot L^{-1} \cdot T^{-2}$ \cr
Et                | \para{Tangent modulus of elasticity}  | $M^1 \cdot L^{-1} \cdot T^{-2}$ \cr
poisson           | \para{Poisson's ratio.}               | $M^0 \cdot L^{0} \cdot T^{0}$ \cr
density           | \para{Material density.}              | $M^0 \cdot L^{0} \cdot T^{0}$ \cr
yield             | \para{Yield Stress $F_y$.}            | $M^0 \cdot L^{0} \cdot T^{0}$ \cr
ultimate          | \para{Ultimate Stress $F_u$.}         | $M^0 \cdot L^{0} \cdot T^{0}$ \cr
n                 | \para{Strain Hardening Exponent}      | $M^0 \cdot L^{0} \cdot T^{0}$ \cr
alpha             | \para{Parameter for Ramberg-Osgood relationship}    | $M^0 \cdot L^{0} \cdot T^{0}$ \cr
beta              | \para{Parameter for strain hardening -- beta = 0 for kinematic hardening.
                          beta = 1 for isotropic hardening.}            | $M^0 \cdot L^{0} \cdot T^{0}$ \cr
ialph             | \para{Rotation constant -- ialph = 0 implies none.
                          ialph = 1 implies Hughes rotation constant.}  | $M^0 \cdot L^{0} \cdot T^{0}$ \cr
pen               | \para{Penalty constant.}                            | $M^0 \cdot L^{0} \cdot T^{0}$
\endtable
\vspace{0.01 in}
\caption{\bf Material Properties}
\label{tab: my-material-properties}
\end{table}

\vspace{0.15 in}\noindent
Two functions, {\tt GetSection()} and {\tt GetMaterial()},
are provided for the retrieval of section and material properties.
{\tt GetSection([elmtno])} takes as its argument, a $(1 \times 1)$
matrix containing the element no. It returns a
{\tt $16 \times 1$} matrix of section properties:

\begin{footnotesize}
\begin{verbatim}
====================================================================
Matrix Element      Quantity     Description 
====================================================================
        [1][1]           Ixx     Moment of inertia about x-x axis.
        [2][1]           Iyy     Moment of inertia about y-y axis.
        [3][1]           Izz     Moment of inertia about z-z axis.
        [4][1]           Ixz     Product of inertia x-z.
        [5][1]           Ixy     Product of inertia x-y.
        [6][1]           Iyz     Product of inertia y-z.
        [7][1]        weight     Section weight
        [8][1]            bf     Width of flange
        [9][1]            tf     Thickness of flange
       [10][1]         depth     Section depth
       [11][1]          area     Section area
       [12][1]     thickness     Thickness of plate
       [13][1]     tor_const     Torsional Constant J
       [14][1]            rT     Section radius of gyration
       [15][1]         width     Section width
       [16][1]            tw     Thickness of web
\end{verbatim}
\end{footnotesize}

\vspace{0.15 in}\noindent
Similarly, {\tt GetMaterial([elmtno])} takes as its argument, a $(1 \times 1)$
matrix containing the element no. It returns a
{\tt $10 \times 1$} matrix of material properties:

\begin{footnotesize}
\begin{verbatim}
========================================================================
Matrix Element      Quantity     Description 
========================================================================
        [1][1]             E     Young's modulus
        [2][1]             G     Shear modulus
        [3][1]            Fy     Yield stress
        [4][1]            ET     Tangent Young's Modulus
        [5][1]            nu     Poission's ratio
        [6][1]       density     Material density
        [7][1]            Fu     Ultimate stress
        [8][1]    thermal[0]     x-direction coeff. of thermal expansion
        [9][1]    thermal[1]     y-direction coeff. of thermal expansion
       [10][1]    thermal[2]     z-direction coeff. of thermal expansion
\end{verbatim}
\end{footnotesize}

\vspace{0.15 in}\noindent
{\bf Note :} An engineer should be careful to make sure that a particular
material or section property is used in a way that is consistent
with the actual material/section type. For example,
{\tt GetSection()} will return a {\it thickness} component
for all element types, even though the concept does not apply to
two- and three-dimensional beam finite elements.

\section{Boundary Conditions}

\vspace{0.15 in}
\noindent\hspace{0.5 in}
The boundary condition is controled through {\em FixNode()} function.
For each degree of freedom (DOF), an index (1 or 0) is used to
indicate the fixed DOF or free DOF. The syntax for {\em FixNode()} is:

\begin{itemize}
\item{}
{\tt FixNode(nodeno}{\tt , bc\_vector);}
Here {\tt nodeno} is the node number to which the
boundary condition will be applied,
{\tt bc\_vector} = [$bc_1, bc_2, bc_3,\cdots$] is either a $(1 \times 3)$ matrix
or a $( 1 \times 6)$ matrix.
$bc_i$ = 1 or 0, i is the ith degree of freedom (dof).
$bc_i$ = 1, stands for constraint for that dof,
while  $bc_i$ = 0 for no constraint for that dof.
\end{itemize}

\vspace{0.15 in}\noindent
{\bf Short Example :} This script fixes the boundary conditions of a finite
element mesh at nodes 1 through 5.

\vspace{0.15 in}
\begin{footnotesize}
\noindent
{\rule{2.1 in}{0.035 in} ABBREVIATED INPUT FILE \rule{2.1 in}{0.035 in} }
\begin{verbatim}

/* [a] : Apply boundary conditions */

   dx = 1; dy = 1 ; dz = 1;
   rx = 0; ry = 0 ; dz = 0;

   bcond = [ dx, dy, dz, rx, ry, rz ];

   for( iNode = 1; iNode <= 5; iNode = iNode + 1 ) {
        FixNode ( iNode , bcond );
   }
\end{verbatim}
\rule{6.25 in}{0.035 in}
\end{footnotesize}

\vspace{0.15 in}\noindent
We have used the variables
{\tt dx}, {\tt dy}, and {\tt dz} to represent translational displacements
in the x, y, and z directions, respectively, and the variables
{\tt rx}, {\tt ry}, and {\tt rz} for rotational displacement about
the x, y, and z axes. A zero value of the matrix element means that the
degree of freedom remain unrestrained. A non-zero value means that full-fixity
applies to the degree of freedom. Hence, nodes 1 through 5 are fixed in
their translational degrees of freedom, and pinned in the three
rotational degrees of freedom.

\section{External Nodal Loads}

\vspace{0.15 in}\noindent
External nodal loads are specified with the function {\tt NodeLoad(nodeno,} {\tt load\_vector);}.

\begin{itemize}
\item{}
The function {\tt NodeLoad(nodeno,} {\tt load\_vector);} may be used
to apply external loads to node number {\tt nodeno}.
Here {\tt load\_vector} = $ [F_x, F_y, M_{xy}]$ for two-dimensional problems,
and {\tt load\_vector} = $ [F_x, F_y, F_z, M_{xy}, M_{yz}, M_{zx} ]$
for three-dimensional problems.
\end{itemize}

\vspace{0.15 in}\noindent
{\bf Short Example :} The following script adds two translational forces,
and one moment to nodes 1 through 5 in a two-dimensional finite element mesh.

\vspace{0.15 in}
\begin{footnotesize}
\noindent
{\rule{2.1 in}{0.035 in} ABBREVIATED INPUT FILE \rule{2.1 in}{0.035 in} }
\begin{verbatim}
       FxMax = 1000.0 lbf; Fy = -1000.0 lbf; Mz = 0.0 lb*in;

       for( iNode = 1; iNode <= 5; iNode = iNode + 1 ) {
            Fx = (iNode/5)*FxMax;
            NodeLoad( iNode , [ Fx, Fy, Mz ]); 
       }
\end{verbatim}
\rule{6.25 in}{0.035 in}
\end{footnotesize}

\vspace{0.15 in}\noindent
External nodal loads are applied in the x-direction,
beginning at {\tt 200 lbf} at node 1, and increasing
linearly to {\tt 1000 lbf} at node 5. A gravity load,
{\tt Fy = -1000.0 lbf} is applied to each of the nodes 1 through 5.

\section{Stiffness, Mass and External Loading Matrices}

\vspace{0.15 in}
\noindent\hspace{0.5 in}
The next step, after the details of the finite element mesh have been fully specified,
is to calculate the finite element properties, and to assemble them into an equilibrium system.
For structural finite element analyses, this step involves calculation of
the stiffness and mass matrices, and an external load vector.
Three finite element library functions, {\tt Stiff()}, {\tt Mass()}, 
and {\tt ExternalLoad()} for these purposes:

\begin{itemize}
\item{\tt Mass([massflag]);}
where massflag is either {\tt [1]} or {\tt [-1]}. Therefore the Mass() command can be 
used as mass = Mass([1]); mass = Mass([-1]); or more explicitly, as, Lumped = [1]; mass = Mass(Lumped);
Consist = [-1]; mass = Mass(Consist);

\item{\tt Stiff()}
Stiff(); calculate the linear elastic stiffness matrix.

\item{\tt ExternalLoad()}
ExternalLoad(); calculate the external nodal loads.
\end{itemize}

\vspace{0.15 in}\noindent
{\bf Short Example :} In the following script of code, 
{\tt mass} is the global mass matrix,
{\tt stiff} is the global stiffness matrix,
and {\tt eload} is a vector of external nodal loads applied to
the finite element global degrees of freedom.

\vspace{0.15 in}
\begin{footnotesize}
\noindent
{\rule{2.1 in}{0.035 in} ABBREVIATED INPUT FILE \rule{2.1 in}{0.035 in} }
\begin{verbatim}

/* [a] : Form Mass matrix, stiffness matrix, and external load vector */

   mass  = Mass();
   stiff = Stiff();
   eload = ExternalLoad();
\end{verbatim}
\rule{6.25 in}{0.035 in}
\end{footnotesize}

\vspace{0.15 in}\noindent
{\bf Note :} The three functions {\tt Mass()}, {\tt Stiff()}, and {\tt ExternalLoad()}
should be called only after the function call to {\tt EndMesh()} -- for details
on the expected input-file layout, see Section \ref{section-structure-of-input-files}.

\section{Internal Loads}

\vspace{0.15 in}\noindent
In the development of many numerical/finite element algorithms, we need
to know the distribution of externally applied nodal forces that will
produce a given displacement pattern. 

\begin{itemize}
\item{\tt InternalLoad(displacement) :}
Calculate the internal nodal loads for given displacement. Where
displacement is a matrix vector contains nodal displacement.

\item{\tt InternalLoad(displacement, IncrementalDisplacement)}
The two argument version of {\tt InternalLoad()} computes the
internal nodal load increments for a given displacement and displacement increment.
Here {\tt displacement} and {\tt IncrementalDisplacement} are the matrix vectors
for the nodal displacement and incremental displacement.
This is function is designed for nonlinear analysis.
\end{itemize}

\vspace{0.15 in}\noindent
{\bf Short Example :} In Part {\tt [a]} of the following script we:
(a) call {\tt Copy()} to make a copy of the global stiffness matrix;
(b) decompose the stiffness matrix into a product of upper and lower triangular matrices;
(c) compute the external load vector;
(d) solve the system of equations via forward and backward substitution, and finally,
(e) setup a delta displacement vector with the first entry equal to 0.0001,
and all other entries equal to zero.

\vspace{0.15 in}
\begin{footnotesize}
\noindent
{\rule{2.1 in}{0.035 in} ABBREVIATED INPUT FILE \rule{2.1 in}{0.035 in} }
\begin{verbatim}

/* [a] : Compute initial displacement and displacement increment */

   lu    = Decompose(Copy(stiff));
   eload = ExternalLoad();
   displ = Substitution(lu, eload0);

   delta_displ = Zero([size, 1]);
   delta_displ [1][1] = 0.0001;

/* [b] : Internal load for initial displacement plus displacement increment */

   iload = InternalLoad(displ, delta_displ); 
\end{verbatim}
\rule{6.25 in}{0.035 in}
\end{footnotesize}

\vspace{0.15 in}\noindent
In the second half of this script, {\tt InternalLoad()} computes
the internal nodal loads for the displacement
vector {\tt displ} plus {\tt delta\_displ}. 

\section{Retrieving Information from ALADDIN}

\vspace{0.15 in}
\noindent\hspace{0.50 in}
We have written two families of functions that query ALADDIN's
database for information on the finite element mesh,
and behavior of a particular finite element model.
Together, these functions may be used for design rule checking,
and for the preparation of matrix output that is compatible
with input to MATLAB graphics. The latter could be a plot
of the finite element mesh, contours of displacements, stresses, and so forth.

\vspace{0.15 in}\noindent
The functions that will retrieve information on the finite element mesh are:

\begin{itemize}
\item{}
{\tt GetCoord([nodeno])} takes as its input a $(1\times1)$ matrix
containing the node number.
For two-dimensional finite element problems, {\tt GetCoord()} 
returns a $(1x2)$ matrix ( m[1][1], m[1][2]) = (x,y).
A $(1\times3)$ matrix is returned for three-dimensional
elements (m[1][1], m[1][2], m[1][3]) = (x,y,z).

\item{}
{\tt GetDof([nodeno])} is short for ``get degree of freedom.''
Get the matching global degrees of freedom number for a node.
It returns a (1xn) matrix, n equals to the number of degrees of freedom in the node.
For two-dimensional beam-column elements, n = 3,
and (m[1][1], m[1][2], m[1][3]) = ( dof\_dx, dof\_dy, dof\_rz);
FOr the three-dimensional frame element,
n = 6,  and (m[1][1],m[1][2],m[1][3],m[1][4],m[1][5],m[1][6]) =
(dof\_dx,dof\_dy,dof\_dz,dof\_rx,dof\_ry,dof\_rz).
A negative number in the output matrix indicates the associated degree of freedom is fixed.

\item{}
{\tt GetNode([elmtno])} retrieves a matrix $(1 \times n)$ of node numbers connected to 
finite element {\tt elmtno}. {\tt n} equals to the number of nodes in the element.
For example, n = 2 for 2D and 3D beam element,
and n = 4 for the 4-node shell element.
The matrix will be in the same order as you input the nodes connection in AddElmt()
\end{itemize}

\vspace{0.15 in}\noindent
Similarly, the functions for retrieving information on the
behavior of a finite element model are:

\begin{itemize}
\item{}
{\tt GetDispl([nodeno, displ\_m])} is short for ``get displacement,''
and takes as input, a node number and the calculated displacement matrix.
Get the structural nodal displacements.
It returns a (1xn) matrix, nequals to the number of degrees of freedom in the node.
For two-dimensional beam elements, n = 3, and
(m[1][1], m[1][2], m[1][3]) = (displ\_x,displ\_y,rot\_z).
Similarly, for three-dimensional frame elements, n = 6,
and ( m[1][1], m[1][2], m[1][3], m[1][4], m[1][5], m[1][6]) =
(displ\_x, displ\_y, displ\_z, rot\_x, rot\_y, rot\_z).

\item{}
{\tt GetStress([elmtno],displacement\_matrix)} accepts as its input, ana element number
and the calculated displacement matrix.
This function gets the nodal forces for the element.

\vspace{0.15 in}
In ALADDIN Version 1.0, this function only works for 2D and 3D beam elements.
It returns a (mxn) matrix, m equals to the number of nodes in the element,
n equals to the number of degrees of freedom in a node.
For example, (mxn)=(2x3) for 2D beam element,

\begin{footnotesize}
\begin{verbatim}
                   1       2       3
       node 1    Fx1     Fy1     Mz1
       node 2    Fx2     Fy2     Mz2

      (mxn)=(2x6) for 2D beam element,

                   1       2       3       4       5       6
       node 1    Fx1     Fy1     Fz1     Mx1     My1     Mz1
       node 2    Fx2     Fy2     Fz2     Mx2     My2     Mz2
\end{verbatim}
\end{footnotesize}

\vspace{0.15 in}
The order of the nodes 1,2... is defined in the nodes connection matrix in AddElmt().
\end{itemize}

\vspace{0.15 in}\noindent
These functions will be demonstrated in Chapter 6.

\clearpage
\section{Library of Finite Elements}
\label{section-finite-element-library}

\vspace{0.15 in}\noindent
Currently, the following finite elements are available:

\vspace{0.15 in}\noindent
{\bf PLANE\_STRESS/PLANE\_STRESS :} Two dimensional 4-node
plane stress/plane strain element.

\begin{figure} [ht]
\epsfxsize=2.9truein
\centerline{\epsfbox{my-chapter5-fig1.ps}}
\caption{Schematic of Plane Stress - Plane Strain Finite Element}
\label{fig:fe-plane-stress-plane-strain}
\end{figure}

\vspace{0.15 in}\noindent
May be used to model linear elastic materials.  

\vspace{0.15 in}\noindent
{\bf Section Properties and Material Properties :} 
Young's modulus {\tt E}; Poisson's ratio $\nu$;
Density $\rho$ for dynamic analyses;
Element type = {\tt PLANE\_STRESS} (or {\tt PLANE\_STRAIN}).
No section properties are needed.

\vspace{0.15 in}\noindent
{\bf Example :} 

\vspace{0.10 in}
\begin{footnotesize}
\noindent
{\rule{2.1 in}{0.035 in} ABBREVIATED INPUT FILE \rule{2.1 in}{0.035 in} }
\begin{verbatim}
    beam_length  = 8 m;
    beam_width   = 1 m;
    beam_height  = 2 m;

    nodeno = 1; elmtno = 0;

    while(elmtno <= 3) {
       elmtno = elmtno + 1;
       AddElmt( elmtno, [nodeno+3, nodeno+1, nodeno+2, nodeno+4], "name_of_elmt_attr");
       nodeno = i + 2;
    }

    ....... input statements removed .... 

\end{verbatim}
\rule{6.25 in}{0.035 in}
\end{footnotesize}

\clearpage
\vspace{0.15 in}\noindent
{\bf 2D\_FRAME}: Two dimensional 2-node frame (or beam/column) element.

\begin{figure} [ht]
\epsfxsize=3.2truein
\centerline{\epsfbox{my-chapter5-fig2.ps}}
\caption{Schematic of Two Dimensional Frame Element}
\label{fig:fe-2d-frame}
\end{figure}

\vspace{0.15 in}\noindent
Each node has two translational, and one rotational, degree of freedom.
May be used for modeling of linear elastic materials.  

\vspace{0.15 in}\noindent
{\bf Section Properties and Material Properties :} 
The section properties are; Moment of inertia {\tt Izz}
and cross-section {\tt area} -- alternatively,
the cross section are is computed from the
section {\tt width} (or {\tt bf}) times the {\tt depth}.
The material properties are Young's modulus {\tt E},
Poisson's ratio $\nu$, and density $\rho$.
Note -- the density may be omitted when a default I-beam section
is referenced from the AISC sections header file {\tt section.h}.
Element type = {\tt 2D\_FRAME}.

\vspace{0.15 in}\noindent
{\bf Example :} 

\vspace{0.10 in}
\begin{footnotesize}
\noindent
{\rule{2.1 in}{0.035 in} ABBREVIATED INPUT FILE \rule{2.1 in}{0.035 in} }
\begin{verbatim}
   end1 = 4*floorno + bayno;
   end2 = end1 + 1;

   AddElmt( elmtno, [ end1 , end2 ], "mybeam");

   ....... input code removed ........

   ElementAttr("floorbeam") { type     = "FRAME_2D";
                              section  = "mysection2";
                              material = "mymaterial";
                            }

   SectionAttr("floorsection2") { Izz      = 1600.3 in^4;
                                  Iyy      = 66.2   in^4;
                                  depth    = 21.0     in;
                                  width    = 8.25     in;
                                  area     = 21.46  in^2;
                                }

   MaterialAttr("floormaterial") { density = 0.1024E-5 lb/in^3;
                                   poisson = 0.25;
                                   yield   = 36.0   ksi;
                                   E       = 29000  ksi;
                                 }
\end{verbatim}
\rule{6.25 in}{0.035 in}
\end{footnotesize}

\clearpage
\vspace{0.15 in}\noindent
{\bf 3D\_FRAME}: Three dimensional 2-node frame (or beam/column) element.

\begin{figure} [ht]
\epsfxsize=5.5truein
\centerline{\epsfbox{my-chapter5-fig3.ps}}
\caption{Schematic of Two Dimensional Frame Element}
\label{fig:fe-3d-frame}
\end{figure}

\vspace{0.15 in}\noindent
Each node has three translational, and three rotational, degrees of freedom.
May be used for modeling of linear elastic materials.  

\vspace{0.10 in}\noindent
{\bf Section Properties and Material Properties :} 
The section properties include moments of inertia {\tt Izz} and {\tt Iyy},
and the cross-section {\tt area}. The cross-section area may also be
computed from the section {\tt depth} times its {\tt width} (or {\tt bf}).
The torsional constant and radius of gyration are defined
by parameters {\tt J} and {\tt rT}, respectively.
The material properites are Young's modulus {\tt E},
Poisson's ratio $\nu$, and the density $\rho$.
Note -- the density may be omitted when a default I-beam section
is referenced from the AISC sections header file {\tt section.h}.
The element type is {\tt 3D\_FRAME}.

\vspace{0.10 in}\noindent
{\bf Example :} 

\vspace{0.10 in}
\begin{footnotesize}
\noindent
{\rule{2.1 in}{0.035 in} ABBREVIATED INPUT FILE \rule{2.1 in}{0.035 in} }
\begin{verbatim}
   elmtno = 0;
   while(elmtno < 4) {
         elmtno = elmtno + 1;
         AddElmt( elmtno, [elmtno, elmtno + 1], "floorelmt");
   }

   ....... code deleted .... 

   ElementAttr("floorelmt") { type     = "FRAME_3D";
                              section  = "mysection";
                              material = "mymaterial";
                            }

   SectionAttr("floorsection") { Izz     = 13824 in^4;
                                 Iyy     =  3456 in^4;
                                 area    =   288 in^2;
                                 depth   =    24 in;
                                 width   =    12 in;
                               }

   MaterialAttr("floormaterial") { .... same as for FRAME_2D example .... }

\end{verbatim}
\rule{6.25 in}{0.035 in}
\end{footnotesize}

\clearpage
\vspace{0.15 in}\noindent
{\bf GENERAL SHELL}: Three dimensional 4 and 8-node five degree of
freedom general shell elements, linear/nonlinear elastic-plastic materials.  

\begin{figure} [ht]
\epsfxsize=4.5truein
\centerline{\epsfbox{my-chapter5-fig7.ps}}
\caption{Schematic of General Shell Element}
\label{fig:fe-general-shell}
\end{figure}

\vspace{0.15 in}\noindent
The element formulation is described in Chen and Austin ~\cite{chen95}.

\vspace{0.15 in}\noindent
{\bf Section Properties and Material Properties :} 
The section attribute is {\tt thickness} alone.
The material properties are Young's modulus {\tt E},
Poisson's ratio $\nu$ and density $\rho$ (for dynamic analysis).
The following parameters apply for nonlinear analysis --
Yield stress {\tt yield}, stress-strain parameters {\tt n},
{\tt alpha} and {\tt beta}, and
{\tt type} (Ramberg-Osgood or Bi-Linear or ELASTIC\_PLASTIC material).

\vspace{0.15 in}
\noindent\hspace{0.5 in}
The material types are {\tt ELASTIC},
{\tt ELASTIC\_PLASTIC} and {\tt ELASTIC\_PERFECTLY\_PLASTIC}.

\vspace{0.15 in}
\noindent\hspace{0.5 in}
The element types for the four node and eight node
shell elements are {\tt SHELL\_4N} and {\tt SHELL\_8N}.

\vspace{0.15 in}\noindent
{\bf Example :} 

\vspace{0.10 in}
\begin{footnotesize}
\noindent
{\rule{2.1 in}{0.035 in} ABBREVIATED INPUT FILE \rule{2.1 in}{0.035 in} }
\begin{verbatim}
   elmtno = 1;
   node_connec = [ 1, 2, 3, 4, 12, 13, 14, 15];
   AddElmt(elmtno, node_connec, "name_of_elmt_attr");

   ..... input code removed .....

   ElementAttr("name_of_elmt_attr") { type     = "SHELL_8N";
                                      section  = "mysection";
                                      material = "ELASTIC";
                                    }
   MaterialAttr("ELASTIC") { density = 1.0 lb/in^3; 
                             poisson = 0.25;   
                             yield   = 36000;  
                             E       = 3E+7 psi;
                           }

   SectionAttr("mysection") { thickness =   1 in; }
\end{verbatim}
\rule{6.25 in}{0.035 in}
\end{footnotesize}

\vspace{0.35 in}\noindent
{\bf SHELL WITH DRILLING DEGREE OF FREEDOM : }
A four node flat shell element with six degrees of freedom per node.
It may be used for modeling of linear elastic materials.  

\begin{figure} [ht]
\epsfxsize=4.5truein
\centerline{\epsfbox{my-chapter5-fig4.ps}}
\caption{Schematic of Flat Shell Finite Element}
\label{fig:fe-flat-shell}
\end{figure}

\vspace{0.15 in}\noindent
The theoretical formulation of this shell finite element,
and numerical examples of its performance
may be found in the masters thesis of Lanheng Jin ~\cite{jin94}.

\vspace{0.15 in}\noindent
{\bf Section Properties and Material Properties :} 
The section attribute is {\tt thickness} alone.
The material properties are Young's modulus {\tt E},
Poisson's ratio $\nu$, and Density $\rho$ (for dynamic analyses).
The element has two modeling parameters {\tt ialpha} and {\tt pen}.
The element type is {\tt SHELL\_4NQ}.

\vspace{0.15 in}\noindent
{\bf Example :} 

\vspace{0.10 in}
\begin{footnotesize}
\noindent
{\rule{2.1 in}{0.035 in} ABBREVIATED INPUT FILE \rule{2.1 in}{0.035 in} }
\begin{verbatim}
   AddElmt( elmtno+1, [ a+1, b+1, b+2, a+2 ], "bridgegirder" );

   ..... input code removed .....

   ElementAttr("bridgegirder") { type     = "SHELL_4NQ";
                                 section  = "girder_flange";
                                 material = "STEEL3";
                               }

   SectionAttr( "girder_flange" ) { thickness = 1.100 in; }
\end{verbatim}
\rule{6.25 in}{0.035 in}
\end{footnotesize}

\vspace{0.10 in}\noindent
{\tt STEEL3} is a material property predefined in {\tt material.h},
and loaded into the ALADDIN database during the program's startup procedure.

\vspace{0.35 in}\noindent
{\bf PLATE}: Four node discrete kirchoff quadrilateral (DKQ) plate element.
Each node has three degrees of freedom -- two rotations, and a lateral displacement,
as shown in Figure ~\ref{fig:fe-plate}.

\begin{figure} [ht]
\epsfxsize=3.5truein
\centerline{\epsfbox{my-chapter5-fig5.ps}}
\caption{Schematic of Plate Finite Element}
\label{fig:fe-plate}
\end{figure}

\vspace{0.15 in}\noindent
{\bf Section Properties and Material Properties :} 
The section property is plate {\tt thickness}.
The material properties include Young's modulus {\tt E},
and Poisson's ratio $\nu$.
The element type is {\tt DKT\_PLATE}.

\vspace{0.15 in}\noindent
{\bf Example :} 

\vspace{0.10 in}
\begin{footnotesize}
\noindent
{\rule{2.1 in}{0.035 in} ABBREVIATED INPUT FILE \rule{2.1 in}{0.035 in} }
\begin{verbatim}
   connect = [ 1, 2, 3, 4];
   AddElmt( elmtno, connect, "name_of_elmt_attr");

   ...... input statements removed .....

   ElementAttr("name_of_elmt_attr") { type     = "DKT_PLATE";
                                      section  = "mysection";
                                      material = "ELASTIC";
                                      map ldof [1,2,3] to gdof [2, 4, 6];
                                    }

   SectionAttr("mysection") { thickness = 2.0 in; }

   MaterialAttr("ELASTIC")  { density = 150 lb/ft^3;
                              poisson = 0.3;
                              yield   = 36000;
                              E       = 29000 psi;
\end{verbatim}
\rule{6.25 in}{0.035 in}
\end{footnotesize}

\vspace{0.15 in}\noindent
Currently, this element does not have a mass matrix.

\chapter{Data Types : Physical Quantity and Matrix Data Structures}

\section{Introduction}

\vspace{0.15 in}
\noindent\hspace{0.5 in}
Basic engineering quantities such as, length, mass, and force,
are defined by a numerical value (number itself) plus physical units.
The importance of the engineering units is well known.
Unfortunately, physical units are not a standard part of main-stream finite element
software packages -- indeed, most engineering software packages simply
assume that engineering units will be consistent,
and leave the details of checking to engineer (in the writers' experience,
it is not uncommon for engineers to overlook this detail).
While this practice of implementation may be satisfactory for
computation of well established algorithms, it is almost certain
to lead to incorrect results when engineers are working on the
development of new and innovative computations.

\vspace{0.15 in}
\noindent\hspace{0.5 in}
ALADDIN deviates from this trend by supporting
the basic data type physical quantity,
as well as matrices of physical quantities.
Engineers may define units as part of the problem description,
and manipulate these quantities via standard arithmetic and matrix operations.
Any set of valid engineering units can be used.
For example, {\em mm}, {\em cm}, {\em m}, {\em km}, and {\em in},
are all valid units of length.
In this chapter, we discuss the data structures used to
implement physical quantities, and matrices of physical quantities.
Special storage techniques for large symmetric matrices are also mentioned.

\section{Physical Quantities}

\vspace{0.15 in}
\noindent\hspace{0.5 in}
Any physical unit can be decomposed into basic units.
The four basic units needed for engineering analysis are: 
length unit $\cal L$; mass unit $\cal M$; time unit t; and temperature unit $\cal T $.
Any engineering unit can be obtained with the following combinations:

 \[ unit =  k {\cal L}^\alpha {\cal M}^\beta t^\gamma {\cal T}^\delta. \]

\vspace{0.15 in}\noindent
where $\alpha, \beta, \gamma, \delta $ are exponents,
and k is the scale factor.
For units of length and mass,
$[\alpha, \beta, \gamma, \delta ]$ = $[1, 0, 0, 0] $ and
$[\alpha, \beta, \gamma, \delta ]$ = $[0, 1, 0, 0] $, respectively.
Non-dimensional quantities (i.e. numbers, degrees, and radians) are given by the
family of zero exponents $[\alpha, \beta, \gamma, \delta ]$ = $[0, 0, 0, 0]$.
With this basic units in place,
we can define the units data structure as:

\begin{footnotesize}
\begin{verbatim}
        #define  SI     100
        #define  US     200
        #define  SI_US  300

        typedef struct dimensional_exponents {
                char         *units_name; /* units name               */
                double      scale_factor; /* scale/conversion factor  */
                double      length_expnt; /* exponent for length      */
                double        mass_expnt; /* exponent for mass        */
                double        time_expnt; /* exponent for time        */
                double        temp_expnt; /* exponent for temperature */
                int           units_type; /* US or SI units           */
        } DIMENSIONS;
\end{verbatim}
\end{footnotesize}

\vspace{0.15 in}\noindent
The character string stores the unit name,
and scale\_factor, the scale factor with respect to a basic unit.
For example, in US units the basic unit of length is inch.
It follows that 12 will be the scale\_factor for one foot.
The units\_type flag allows for the representation of
various systems of units (e.g. SI, US, and combined SI\_US).
SI\_US units are those shared by both US and SI systems -- the
two most important examples being time and non-dimensional units.
We use variables of data type double to represent exponents,
thereby avoiding mathematical difficulties in manipulation
of quantities. For example, if $ x = y ^{0.5}$,
the unit exponents of x,  $[\alpha_x, \beta_x, \gamma_x, \delta_x ]$ = 
$0.5 \times [\alpha_y, \beta_y, \gamma_y, \delta_y]$.
If the units exponents were defined as integers,
then x would be incorrectly truncated to
$[\alpha_x, \beta_x, \gamma_x, \delta_x ]$ = $[0, 0, 0, 0]$.
With the units data structure in place,
the quantity data structure is simply defined as

\begin{footnotesize}
\begin{verbatim}
        typedef struct engineering_quantity {
                double       value;
                DIMENSIONS  *dimen;
        } QUANTITY;
\end{verbatim}
\end{footnotesize}

\vspace{0.15 in}\noindent
Notice that physical units alone are the special
case of a quantity with numerical value 1.0.
 
\subsection{Relationship between Quantity and Units}

\vspace{0.15 in}
\noindent\hspace{0.5 in}
In engineering applications, there is no need to distinguish
between different systems of units during calculation.
The main use of units is for clarification
of problem input and problem output.
ALADDIN stores a physical quantity as an {\em unscaled} {\em value}
with respect to a set of reference units,
and by default, all quantities are stored internally in the SI units system.
In the SI system, the reference set of units are
meter ``m'' for length, kilogram ``kg'' for mass,
second ``sec'' for time and ``deg\_C'' for temperature.

\vspace{0.15 in}
\noindent\hspace{0.5 in}
Suppose, for example, that we define the quantity, x = 1 km (one kilometer).
Because 1 km = 1000 m, x will be saved internally as value = 1000,
In the data structure {\tt DIMENSIONS},
units\_name will point to the character string ``km''.
The scale\_factor for ``km'' with respect to reference unit ``m'' is 1000.
Now lets see what happens when we print x.
Because the output value = value/scale\_factor = 1.
``print x'' will still give ``1 km''.

\subsection{US and SI Units Conversion}

\vspace{0.15 in}
\noindent\hspace{0.5 in}
Scale factors are needed to convert units in US to SI, and vice versa.
All quantities in US units must be converted into SI
before they can be used in calculations. Fortunately,
for most sets of units, only a single scale factor is
needed to convert a quantity from US into SI and
vice versa (e.g.  1 in = 24.5 mm = 25.4E-3 m).

\vspace{0.15 in}
\noindent\hspace{0.5 in}
The important exception is temperature units.
Conversion of temperature units is complicated by the
nonlinear relationship between systems -- that is
$x^o F \neq x \times 1^o F$, since $ x^o F = (5/9)(x -32) \times 1^o C $.
In other words, the value and unit of a temperature quantity can not be treated separately.
We have to know the value of temperature before we can convert it's units from
one to another unit system. ALADDIN handles this problem by first installing
the ``deg\_F'' (unit for $ ^o F $) with scale\_factor = 1 into symbol table.
When the value (a number) and the unit are combined into a quantity
we convert the deg\_F into deg\_C. When a temperature quantity is printed out,
we need to convert it's units to the prefered units.
A related complication is units for temperature increment.
One degree of temperature increase is different from one degree temperature.
To see how this problem arises, let $y = 1^o F$.
If y is a one degree temperature increase,
then $ y = (\alpha + 1)^oF - \alpha^oF = 5/9 ^oC$,
where $\alpha $ is value of previous temperature.
If, on the other hand, y is at one degree Fahrenheit temperature,
then $ y = (5/9)(1-32) ^oC = -17.22 ^o C$.
Similar difficulties crop up in the computation of temperature gradient --
$ 10 mm/^oF $ can only be considered as a 10 mm extension
for one Fahrenheit degree temperature increase, which is $ 18 mm/^oC $.
This cannot be considered as $ (10 mm) / (1^oF)$, which is $ (-0.5806 mm) / (1^oC)$
(divisions of this kind in real engineering computations are meaningless).
To avoid these confusions, a new set of units are chosen
for temperature increments, they are ''DEG\_F and DEG\_C'' to distinguish from ``deg\_F and deg\_C''.
The conversion between DEG\_F and DEG\_C is DEG\_F = 5/9 DEG\_C.
And for simplicity, we will ignore the case of $ (10 mm) / (1^oF) $.

\begin{table}[p]
\tablewidth = 6.0truein
\begintable
\multispan{2}\tstrut\hfil Functions for Units Operations    \hfil\crthick
Function           | \para{Purpose of Function}                       \cr
SameUnits()        | \para{Check if two units are of same type}       \cr
UnitsMult()        | \para{Multiply two units}                        \cr
UnitsDiv()         | \para{Divide two units}                          \cr
UnitsCopy()        | \para{Make a copy of a unit}                     \cr
UnitsPower()       | \para{compute a unit raised to a power}          \cr
UnitsNegate()      | \para{Negate a unit d = 1/d}                     \cr
UnitsPrint()       | \para{Print name, scalefactor, type and exponents of a unit}   \cr
ZeroUnits()        | \para{Initialize a unit}                                       \cr
DefaultUnits()     | \para{Determine pointer on symbol table for a given unit name} \cr
UnitsSimplify()    | \para{Simplify unit name string with simple character string}  \cr
RadUnitsSimplify() | \para{Convert a nondimensional unit into a radian unit}        \cr
UnitsLength()      | \para{Measure the total length of several units names}         \cr
ConvertTempUnits() | \para{Convert temperature units between SI and US systems}     \cr
UnitsConvert()     | \para{Convert units between US and SI systems --
                           except for temperature units}
\endtable
\vspace{0.01 in}
\caption{\bf Functions for Units Operations}
\label{tab: my-units-functions}
\end{table}

\vspace{0.20 in}\noindent
{\bf Library Functions for Internal Operations on Units :}
ALADDIN also provides functions for internal operations
on units -- example functions include copying units,
obtaining default units, checking consistency of units,
simplifying the units name after various operations,
calculating the length of units name,
and printing of units. For a summary of functions,
see Table~\ref{tab: my-units-functions}.
{\tt Note :} Some mathematical functions such as
log(x), ln(x) or exp(x), cannot have units in their argument lists.
Other mathematical functions, such as sine and cosine,
can only have degree or radian in their arguments.

\clearpage
\section{Matrices}

\vspace{0.15 in}\noindent
ALADDIN's matrix module has following features:

\begin{description}
\item {[1]}
It supports wide range of matrix operations, data types (integer, double and complex),
and storage methods. In this section, only the double and indirect storage methods
are discussed. The skyline matrix for large degree of freedom will be
discussed in next section.
\vspace{0.05 in}
\item {[2]}
It allows users to dynamic allocate and deallocate the memory of matrix.
\vspace{0.05 in}
\item {[3]}
It provides the units for each element of the matrix.
\end{description}

\vspace{0.15 in}\noindent
Our strategy of development is to devise a software framework that will allow
any combination of data type and storages scheme to be implemented.

\begin{figure}[h]
\epsfxsize=6.0truein
\centerline{\epsfbox{my-chapter7-fig1.ps}}
\caption{Layout of Memory in Matrix Data Structure}
\label{fig: my-chapter2-fig1}
\end{figure}

\vspace{0.15 in}\noindent
Figure~\ref{fig: my-chapter2-fig1} shows a high-level layout
of memory for the matrix data structure.
Details of the matrix data structure are as follows:

\begin{footnotesize}
\begin{verbatim}

        /* Data Structures for Matrices of Engineering Quantities */

        typedef struct {
                double dReal, dImaginary;
        } COMPLEX;

        typedef enum {
                INTEGER_ARRAY  = 1,
                DOUBLE_ARRAY   = 2,
                COMPLEX_ARRAY  = 3
        } DATA_TYPE;

        typedef enum {
                SEQUENTIAL = 1,
                INDIRECT   = 2,
                SKYLINE    = 3,
                SPARSE     = 4
        } INTERNAL_REP;

        typedef struct MATRIX {
                char      *cpMatrixName;    /*  *name           */
                int             iNoRows;    /*  no_rows         */
                int          iNoColumns;    /*  no_columns      */
                DIMENSIONS  *spRowUnits;    /*  *row_units_buf  */
                DIMENSIONS  *spColUnits;    /*  *col_units_buf  */
                INTERNAL_REP       eRep;    /*  storage type    */ 
                DATA_TYPE         eType;    /*  data type       */
                union {
                    int           **iaa;
                    double        **daa;   
                    COMPLEX       **caa;
                } uMatrix;
        } MATRIX;
\end{verbatim}
\end{footnotesize}

\vspace{0.15 in}\noindent
Memory is provided for a character string to the matrix name,
two integers for the number of matrix rows and columns,
two one-dimensional arrays of data type {\tt DIMENSIONS},
and integer flags for the basic data type and storage scheme for matrix elements.
The union {\tt uMatrix} contains pointers to matrix
bodies of integers, doubles and {\tt COMPLEX} elements.
For the purpose of this study, however, we will implement and
describe algorithms for data type double plus units
alone (i.e. {\tt DATA\_TYPE} equals {\tt DOUBLE\_ARRAY}).

\begin{figure}[h]
\epsfxsize=5.0truein
\centerline{\epsfbox{Units_Buffer.ps}}
\caption{Matrix with Units Buffers}
\label{fig: Units_Buffer}
\end{figure}

\vspace{0.15 in}
\noindent\hspace{0.5 in}
The units for elements in a matrix are stored
in two one-dimensional arrays of data type DIMENSIONS.
One array stores column units, and the second array row units.
The units for matrix element at row {\bf i} and column {\tt j}
is simply the product of the ${\tt i}^{th}$ element of the
row units buffer and the ${\tt j}^{th}$ element of column units buffer.
Figure~\ref{fig: Units_Buffer} shows, for example, a 3x3 stiffness matrix.
Elements A[1][1] = 10 pa.m = 10 N/m; A[2][3] = 7
$pa.m^2 $ = 7 N, and A[3][3] = 13 $ pa.m^2*m $ = 13 N.m.
This strategy for storing units not only requires much less
memory than complete element-by-element storage of units,
but it reflects the reality that most engineering matrices
are in fact, convenient representations of equations of motion and equlibrium.
The units of individual terms in these equations must be consistent.


\begin{figure}[h]
\vspace{0.10 in}
\epsfxsize=6.0truein
\centerline{\epsfbox{my-chapter7-fig2.ps}}
\caption{Matrix body for Indirect Storage Pattern}
\label{fig: my-chapter2-fig2}
\end{figure}

\vspace{0.15 in}
\noindent\hspace{0.5 in}
Matrices generated through the command language and file input are
stored with an {\tt INDIRECT} storage pattern.
Details of the latter storage pattern are shown
in Figure~\ref{fig: my-chapter2-fig2}.

\subsection{Skyline Matrix Storage}

\vspace{0.15 in}
\noindent\hspace{0.5 in}
The method of skyline storage is suitable for large symmetric matrices,
which are sparsely populated by non-zero
elements along (or close to) the matrix diagonal.
ALADDIN uses skyline storage for matrices
generated during finite element analysis.
To illustrate the general idea of skyline storage,
consider symmetric matrix A given by

\[ A = \left [
\begin{array}{rrrrrr}
 10 &  4 & 0 & 0 &  3 &  0 \\
  4 & 45 & 0 & 1 &  0 &  0 \\
  0 &  0 & 3 & 0 &  0 &  0 \\
  0 &  1 & 0 & 1 &  6 &  3 \\
  3 &  0 & 0 & 6 &  4 & 34 \\
  0 &  0 & 0 & 3 & 34 & 20 
\end{array}
\right] \]

\vspace{0.15 in}\noindent
We first note that indirect storage of A would require a $(6 \times 1)$ array
of pointers, plus six one-dimensional arrays,
each containing 6 elements of data type double.
On an engineering workstation where pointers are 4 bytes, 
312 bytes are needed for the matrix body.
The method of skyline storage reduces memory requirements
by taking advantage of matrix symmetries,
and storing in each column, only those matrix
elements from the matrix diagonal to the uppermost non-zero element.
The ``distance'' between the diagonal and the
last non-zero element is called the ``column height.'' 
Figure~\ref{fig: Matrix_Skyline} shows the layout of
memory for skyline storage of matrix A.
In each column, memory is allocated for the 
memory for the size of ``column height'' plus one,
with the height stored in the first element of the corresponding array.
The second component of the column array stores the matrix diagonal,
and the last element, the upper-most non-zero element in the array.
The column height is needed for computation of matrix operations --
for example, matrix element ({\tt iRowNo}, {\tt iColumnNo}) is
accessed and printed with the formula

\begin{figure}[t]
\epsfxsize=5.0truein
\centerline{\epsfbox{Matrix_Skyline.ps}}
\caption{Method of Skyline Storage for Symmetric Matrix}
\label{fig: Matrix_Skyline}
\end{figure}

\begin{footnotesize}
\begin{verbatim}
    ik = MIN( iRowNo, iColumnNo );
    im = MAX( iRowNo, iColumnNo );

    if((im-ik+1) <= spA->uMatrix.daa[im-1][0])
        printf(" %12.5e ", spA->uMatrix.daa[ im-1 ][ im-ik+1 ]);
    else
        printf(" %12.5e ", 0.0);
\end{verbatim}
\end{footnotesize}

\vspace{0.15 in}\noindent
Here {\tt MIN()} and {\tt MAX()} are macros that return the
minimum and maximum of two input arguments, respectively.

\vspace{0.15 in}\noindent
{\bf Functions in Matrix Library :}
Table~\ref{tab: my-matrix-operations} contains a
summary of matrix functions provided by ALADDIN.
During matrix operations, consistency of units is checked,
as is compatibility of matrix dimensions.
Readers should notice the subtle difference between function pairs
MatrixAdd() and MatrixAddReplace(),
MatrixSub() and MatrixSubReplace(), and
MatrixMult() and MatrixMultReplace().
In the functions MatrixInverseIteration() and MatrixHouseHolder(),
matrices A and B must be symmetric.

\begin{table}[hp]
\tablewidth = 6.0truein
\begintable
\multispan{2}\tstrut\hfil Matrix Function and Purpose       \hfil\crthick
Function           | \para{Purpose of Function}                       \cr
MatrixAlloc()      | \para{Allocate memory for matrix}                \cr
MatrixFree()       | \para{Free memory for matrix}                    \cr
MatrixPrint()      | \para{Print content of matrix}                   \cr
MatrixCopy()       | \para{Make a copy of matrix}                     \cr
MatrixTranspose()  | \para{Compute transpose of matrix}               \cr
MatrixAdd()        | \para{Compute sum of two matrices $[C] = [A] + [B]$} \cr
MatrixSub()        | \para{Compute difference of two matrices $[C] = [A] - [B]$} \cr
MatrixMult()       | \para{Compute product of matrices}                \cr
MatrixNegate()     | \para{Negate matrix [B] = -[A]}                   \cr
MatrixAddReplace() | \para{Replacement sum of two matrices $[A] = [A] + [B]$} \cr
MatrixSubReplace() | \para{Replacement difference of two matrices $[A] = [A] - [B]$} \cr
MatrixMultReplace()| \para{Replacement product of two matrices $[A] = [A].[B]$} \cr
MatrixNegateReplace() | \para{Negate matrix replacement [A] = -[A]}    \cr
MatrixSubstitute() | \para{Substitute small matrix into larger matrix} \cr
MatrixExtract()    | \para{Extract small matrix from larger matrix}    \cr
MatrixSolve()      | \para{Solve linear equations $[A]\{x\} = \{b\}$}  \cr
MatrixInverse()    | \para{Compute matrix inverse $[A]^{-1}$}          \cr
MatrixInverseIteration() | \para{Use inverse iteration to solve symmetric
                                 eigenvalue problem $[A]\{x\} = \lambda [B]\{x\}$
                                 for lowest eigenvalue and eigenvector}\cr
MatrixHouseHolder() | \para{Use Householder transformation and QL Algorithm
                            to solve $[A]\{x\} = \lambda \{x\}$}
\endtable
\vspace{0.01 in}
\caption{\bf Selected Matrix Operations}
\label{tab: my-matrix-operations}
\end{table}

\subsection{Units Buffers for Matrix Multiplication}

\vspace{0.15 in}
\noindent\hspace{0.5 in}
The handling of units in multiplication of two
dimensional matrices needs special attention. 
Let A be a $(p \times q)$ matrix with row units
buffer [ $ a_1, a_2, \cdots, a_p $ ] and column units buffer[ $ b_1, b_2, \cdots, b_r $].
And let B be a $(q \times r)$ matrix with row units buffer
[ $ c_1, c_2, \cdots, c_q $ ] and a column units buffer [$ d_1, d_2, \cdots, d_q $].
The units for elements $(A)_{ik}$ and $(B)_{kj}$
are $ a_i*b_k $ and $ c_k*d_j $, respectively.
Moreover, let C be the product of A and B. From basic linear algebra
we know that $(C)_{ij} = A_{ik}*B_{kj}$, with summation implied on indice {\tt k}.
The units accompanying $ (C)_{ij} $ are $a_i b_k * c_k d_j$ for k = 1, 2, ..., q.
Due to the consistency condition, all of the terms in 
$\sum_{k = 1}^q A_{ik}*B_{kj}$ must have same units.
Put another way, the exponents of units must satisfy the product constraint
$a_i b_1 c_1 d_j = a_i b_2 c_2 d_j = \cdots = a_i b_q c_q d_j $.
The units for $ C_{ij} $ are $ a_i b_1 c_1 d_j $.

\begin{figure}[h]
\epsfxsize=5.5truein
\centerline{\epsfbox{M_Mult.ps}}
\caption{Units Buffer Multiplication of Two Matrices}
\label{fig: M_Mult}
\end{figure}

\vspace{0.15 in}\noindent
The units buffers for matrix C are written as a
row units buffer [ $ a_1c_1, a_2 c_1, \cdots, a_p c_1 $],
and a column buffer is [ $ d_1b_1, d_2 b_1, \cdots, d_r b_1 $].
This arrangement of units exponents is graphically displayed in Figure ~\ref{fig: M_Mult}.
It is important to notice that although the units for C matrix are unique,
the solution for the units buffers is not unique.

\subsection{Units Buffers for Inverse Matrix}

\vspace{0.15 in}
\noindent\hspace{0.5 in}
The units buffers for an inverse matrix may be
obtained as the special case of above results.
Again let A be the square matrix of interest,
with number of rows and columns equal (i.e. p = q).
Let matrix B be the inverse matrix of matrix A. From the previous section
we know that the units for element $ C_{ij} $ are $ a_i b_k c_k d_j $,
and because B is the inverse of matrix A,
the diagonal elements of the matrix C must be dimensionless
(i.e. $a_i b_k c_k d_i = 1$ for $k = 1,2 \cdots p$ and $i = 1,2 \cdots p$).
However, because contents of the units buffers need not be unique, and

\[ c_k d_i = {1 \over { a_i b_k}} \]

\vspace{0.15 in}\noindent
we can arbitrarily select $c_k$ and $d_j$
with the following set as the units buffers: 

\[ c_k = {1 \over {b_k}} \]

\vspace{0.15 in}\noindent
for $ k = 1,2, \cdots p$, and $d_i = {1/{a_i}}$ for $i = 1,2, \cdots p$.
The reader should notice that in this arrangement of units exponents,
the row units buffer of inverse matrix of A
is the inverse of A's column units buffer,
and the column buffer of inverse matrix of A
is the inverse if A's row units buffer.

\vspace{0.15 in}
\noindent\hspace{0.5 in}
Let [M] be a $(n \times n)$ square matrix, $[M]^{-1}$ it's inverse.
In expanded matrix form (i.e. including units buffers),
[M] and $[M]^{-1}$ can be written:

\[ [M] =  \left [
\begin{array}{rrrrr}
units  & b_1     & b_2    & \cdots & b_n    \\
a_1    &  M_{11} & M_{12} & \cdots & M_{1n} \\
a_2    &  M_{21} & M_{22} & \cdots & M_{2n} \\
\cdots &  \cdots & \cdots & \cdots & \cdots \\
a_n    &  M_{31} & M_{32} & \cdots & M_{nn}
\end{array}
\right]
 \]

\[ [M]^{-1} =  \left [
\begin{array}{rrrrr}
units      & {a_1}^{-1} & {a_2}^{-1} & \cdots & {a_n}^{-1} \\
{b_1}^{-1} &  M^*_{11}  & M^*_{12}   & \cdots & M^*_{1n}   \\
{b_2}^{-1} &  M^*_{21}  & M^*_{22}   & \cdots & M^*_{2n}   \\
\cdots     &  \cdots    & \cdots     & \cdots & \cdots     \\
{b_n}^{-1} &  M^*_{31}  & M^*_{32}   & \cdots & M^*_{nn}
\end{array}
\right]
 \]

\vspace{0.15 in}\noindent
The corresponding [IR] and [IL] are :

\[ [IR] = [M]\times [M]^{-1} 
 = \left [  
\begin{array}{rrrrr}
units           & b_1.{a_1}^{-1} & b_1.{a_2}^{-1} & \cdots & b_1.{a_n}^{-1}    \\
a_1.{b_1}^{-1}  &  1             & 0              & \cdots & 0      \\
a_2.{b_1}^{-1}  &  0             & 0              & \cdots & 0      \\
\cdots          &  \cdots        & \cdots         & \cdots & \cdots \\
a_n.{b_1}^{-1}  &  0             & 0              & \cdots & 1
\end{array} 
\right ]  
\]

\[
= \left [ 
\begin{array}{rrrrr}
units  & {a_1}^{-1}     & {a_2}^{-1} & \cdots & {a_n}^{-1}  \\
a_1    &  1             & 0               & \cdots & 0      \\
a_2    &  0             & 0               & \cdots & 0      \\
\cdots &  \cdots        & \cdots          & \cdots & \cdots \\
a_n    &  0             & 0               & \cdots & 1
\end{array}
\right ]
\]

\[ [IL] = [M]^{-1}\times [M] = \left [
\begin{array}{rrrrr}
units       & a_1    & a_2     & \cdots & a_n    \\
{a_1}^{-1}  &  1     & 0       & \cdots & 0      \\
{a_2}^{-1}  &  0     & 0       & \cdots & 0      \\
\cdots      & \cdots & \cdots  & \cdots & \cdots \\
{a_n}^{-1}  &  0     & 0       & \cdots & 1
\end{array}
\right]
\]

\vspace{0.15 in}\noindent
We observe that the units of [IL] and [IR] depend only
on the row units buffer of the matrix [M].

\vspace{0.15 in}\noindent
{\bf Example from Structural Engineering :}
In undergraduate structural analysis classes we learn that
an external force vector \{f\} can be written as the product 
of a stiffness matrix {\tt K} and displacement vector {\tt d}.

\[ [K] \{ d\} = \{f\} \]

\vspace{0.15 in}\noindent
Suppose that we are working in a three dimensional coordinate space (x,y,z).
Without a loss of generality in discussion,
we can limit the total number of degrees of freedom to three;
two translations $u$ and $v$ in directions x and y, respectively,
and a rotational degree of freedom $\theta$ about the z axis,
The translational forces and moment in
vector \{f\} are $ f_1, f_2 $, and $ M $.
In expanded form the abovementioned equation is:

\[ \left [
\begin{array}{rrr}
 k_{11} & k_{12} & k_{13} \\
 k_{21} & k_{22} & k_{23} \\
 k_{31} & k_{32} & k_{33}
\end{array}
\right] \; \times
\left \{
\begin{array}{r}
 u  \\
 v \\
 \theta
\end{array}
\right \} = 
\left \{
\begin{array}{r}
 f_1  \\
 f_2 \\
 M
\end{array}
\right \} 
 \]

\vspace{0.15 in}\noindent
Because $u$ and $v$ are translational displacements,
they will have length units, ``m'' or ``in.''
External forces $f_1$ and $f_2$ will have units newton ``N,'' or pound force ``lbf''.
For the underlying equations of equilibrium to be consistent,
units for $k_{11}, k_{12}, k_{21}$ and $k_{22}$ must be ``N/m'' or ``lbf/in.''
And because $\theta$ has non-dimensional radian units,
$k_{13}$ and $k_{23}$ have units  ``N'' or ``lbf.''
The third equation represents rotational equilibrium,
with its right-hand side having units of moment (i.e ``N.m'' or ``lbf.in'').
The units for $k_{31}$ and $k_{32}$ are ``N'' or ``lbf,''
and units for $k_{33}$, ``N.m'' or ``lbf.in.''
For SI units we write:

\[ [K] =  \left [
\begin{array}{rrrr}
units & N/m & N/m & N \\
 \;   &  k_{11} & k_{12} & k_{13} \\
 \;   &  k_{21} & k_{22} & k_{23} \\
 m    & k_{31} & k_{32} & k_{33}
\end{array}
\right] 
\]

\noindent
and for US units:
\[ [K] =  \left [
\begin{array}{rrrr}
units & lbf/in & lbf/in & lbf \\
 \;   &  k_{11} & k_{12} & k_{13} \\
 \;   &  k_{21} & k_{22} & k_{23} \\
 in   &  k_{31} & k_{32} & k_{33}
\end{array}
\right].
 \]

\vspace{0.15 in}\noindent
The equations of equilibrium
and units buffers for the corresponding compliance matrix are:

\[ \left [
\begin{array}{rrr}
 c_{11} & c_{12} & c_{13} \\
 c_{21} & c_{22} & c_{23} \\
 c_{31} & c_{32} & c_{33}
\end{array}
\right] \; \times
\left \{
\begin{array}{r}
 f_1  \\
 f_2 \\
 M
\end{array}
\right \} =
\left \{
\begin{array}{r}
 u  \\
 v \\
 \theta
\end{array}
\right \}
 \]

\noindent
For SI units,

\[ [C] =  \left [
\begin{array}{rrrr}
units & \;      & \;     &  1/m \\
 m/N  &  c_{11} & c_{12} & c_{13} \\
 m/N  &  c_{21} & c_{22} & c_{23} \\
 1/N  &  c_{31} & c_{32} & c_{33}
\end{array}
\right] 
 \]

\vspace{0.15 in}\noindent
A similar matrix C can be written for US units.
We observe that the units buffers for the compliance matrix,
which is the inverse of the stiffness matrix,
follows the general rule stated at
the conclusion of the previous section.

\vspace{0.15 in}
\noindent\hspace{0.5 in}
Now let's look at the numerical details and units of
the matrix product $[K] \times [C]$ -- for notational convenience,
we will call the identity product [IR]. 
The elements first and second rows of $[IR]$ have
units of [ 1, 1, 1/m], and the third row of $[IR]$, units [m,m,1].
In expanded matrix form, we have:

\[ [IR] = [K]\times [C] = \left [
\begin{array}{rrrr}
units  & \;& \;& 1/m \\
\; & 1 & 0 & 0 \\
\; & 0 & 1 & 0 \\
 m & 0 & 0 & 1
\end{array}
\right]
 \]

\vspace{0.15 in}\noindent
Similarly we have

\[ [IL] = [C]\times [K] = \left [
\begin{array}{rrrr}
units  & \;& \;& m \\
\;  & 1 & 0 & 0 \\
\;  & 0 & 1 & 0 \\
1/m & 0 & 0 & 1
\end{array}
\right]
 \]

\vspace{0.15 in}\noindent
Three points are noted about this result.
Matrices [IL] and [IR] are not equal as most engineers might expect --
instead, [IL] is the transpose of [IR].
Second, [IL] and [IR] are not symmetric with respect
to the units or units buffers.
Finally, these ``identity'' matrices are not dimensionless.
Even though the values of off-diagonal elements are all zero,
some of them have non-dimensional units.

\vspace{0.15 in}
\noindent\hspace{0.5 in}
Some readers may feel uncomfortable knowing that
so-called ``indentity'' matrices are not dimensionless,
and may wonder how could that be.
Actually, when you look carefully into the equations,
the results are not surprising.
For example, let's take the first column of compliance matrix [C]
and label it $\{C_1\} = [ c_{11}, c_{21}, c_{31} ] ^T $.
When [K] $\{C_1\}$ is compared to [K] $\{d\}$,
one quickly realizes that $ [ c_{11}, c_{21}, c_{31} ]$
takes the place of [u, v, $\theta$] (here $c_{11}, c_{21}$ are
the displacement per unit force, and $c_{31}$ is the rotation per unit moment).
Given that the units for \{f\} are $ [force, force, moment]^T $,
then units for the first column of matrix [IR]
will be [force/force, force/force, moment/force] (or [1, 1, m]).
Similar results may be derived for the remaining columns of C,
and in fact, may be generalized.



\chapter{Introduction}

\section{Introduction}

\vspace{0.15 in}
\noindent\hspace{0.5 in}
In the 1970's and early 1980's finite element packages were
often developed with the goal of optimizing numerical considerations alone.
As a result, they offer a restricted set of numerical algorithms to compute
time-history responses, and can be very difficult to maintain and extend.
Although current softwares -- see for example, .... we need to find a list --
have increased their capacity in solving engineering problems, some problems still remain.
For example, units are not part of the program infrastructure.
Indeed, it is the engineer's responsibility to check that units are consistent.
In practical engineering analysis, however, units can be as important
as the numerical quantity itself. Incorporating engineering units into computer programs will
benefit engineering analysis in many ways. 

\vspace{0.15 in}
\noindent\hspace{0.5 in}
Readability in input file is another problem for many finite element softwares.
The format specifications of input files in 
many engineering packages is so cryptic, that it is often impossible to
understand the purpose of the engineering problem without first
carefully reading of the program user manual.
For example, part of input file from the ABAQUS \cite{abaqus92} finite
element package to generate nodes is:

\begin{footnotesize} 
\begin{verbatim}
    *NODE
    1
    7, 10.
    87,10., 10.
\end{verbatim}
\end{footnotesize} 

\noindent\hspace{0.5 in}
An engineer cannot readily understand what the program is doing, 
unless he/she is familiar with the ABAQUS user manual. (Actually ABAQUS generates
finite element nodes 1, 7, and 87 at coordinates (x,y,z) = (0,0,0),
(10, 0, 0), and (10, 10, 0)). The same input could be made more
readable if it were rewritten as:
\begin{footnotesize} 
\begin{verbatim}
    node1   =  1;  coord1   = [  0,  0,  0];
    node7   =  7;  coord7   = [ 10,  0,  0]; 
    node87  = 87;  coord87  = [ 10, 10,  0];

    AddNode( node1, coord1 );
    AddNode( node7, coord7 );
    AddNode(node87, coord87);
\end{verbatim}
\end{footnotesize} 

\noindent or simply as  
\begin{footnotesize} 
\begin{verbatim}
    AddNode( 1, [  0,  0,  0]);
    AddNode( 7, [ 10,  0,  0]);
    AddNode(87, [ 10, 10,  0]);
\end{verbatim}
\end{footnotesize} 
\noindent
as in our input file.
\vspace{0.25 in}

\noindent
{\bf Matrix Operations :}
Engineers usually not limit their minds to numbers and simple arithmetics, matrix
operations are more often used in engineering analysis, especially in finite element analysis.
Unfortunately, only few finite element softwares, if any, provides the capacity 
of performing matrix operations. In this study, we provide a 
input file that allows the analyst to key in the solution layout the same
way he writes down on his notebook including the matrix operation, such as : $[A] = [B]*[C] $. 
\vspace{0.15 in}

\noindent
{\bf Flexibility of Programs : }
The flexibility of programs is also very important for many advanced users of programs.
It allows users to solve his problem in a way that the program is not provided. Many softwares
provides user defined subroutines feature to enhance the flexibility of programs. The much
preferred way is to layout the solution strategy in the input file. This feature is provided
in this our program in this study.
\vspace{0.15 in}

\noindent\hspace{0.5 in}
We propose to deviate from this model by trying to design finite element
software within the framework of a well-defined language or grammar ??
Experience in other domains (i.e. computer science) indicates such
strategies result in modular software that is extensible. 
\vspace{0.15 in}

\noindent\hspace{0.5 in}
The idea is that an engineer would use the language to not only generate/display
finite element meshes, but also control the numerical strategy.
Once the problem description is complete, the engineer would write down the discrete
equations to be solved inside the finite element data file, along with the numerical strategy
(linearization and convergence criteria) that should be followed to solve
a particular linear/nonlinear finite element problem.
Engineers are able to implement these strategies
without having to learn the internal details of the code, or having to recompile the
source code. To test out this idea, we built a stack machine using the
compiler construction tool YACC (Yet Another Compiler Compiler), and designed
a simple finite element macro language. The stack machine reads instructions
from a problem description file. These instructions are for generation of the finite 
element model, and step-by-step solution of finite element problems. 
The macro language supports definition of engineering variables
with engineering units, matrices, looping constructs, and conditional branching,
just like traditional programming languages. The language also supports solid
modeling and automatic generation of finite element meshes from solid models.

\vspace{0.15 in}
\noindent\hspace{0.5 in}
To illustrate above features, few examples are given to show the format of
input files. More examples can be found in following chapters and at the
end of the report.
\vspace{0.15 in}

\noindent
{\bf Example [1] :}
In our first example, we assign the physical
quantity ``2 in'' to a variable named ``x'',
and print ``x'' out in terms of different, but compatible,
units of length. The units are meter, cm, mm, ft.
The input file is

\begin{footnotesize}
\begin{verbatim}
    x = 2 in;
    print x (m);
    print x (cm);
    print x (mm);
    print x (ft);
\end{verbatim}
\end{footnotesize}

\noindent The output is
\begin{footnotesize}
\begin{verbatim}
    0.0508 m    5.08 cm    50.8 mm  0.1667 ft
\end{verbatim}
\end{footnotesize}
\vspace{0.05 in}

\noindent
Similarly, to print a temperature at 33.8 degree of Fahrenheit
in terms of Celsius is

\begin{footnotesize}
\begin{verbatim}
    print 33.8 deg_F (deg_C);
\end{verbatim}
\end{footnotesize}

\noindent
and the result is {\tt 1 deg\_C}.

\vspace{0.2 in}
\noindent
{\bf Example [2]:}
Now let's see how a for-loop construct and basic matrix
operations may be combined to setup and iteratively compute the
solution to family of linear equations [A] \{x \} = \{b\}, namely:

\begin{equation}
 [A] = \left [
\begin{array}{rrr}
 1 & 15 & 15 \\
 15 & 1 & 15 \\
 15 & 15 & 1 
\end{array} 
\right] \; \; \;
\{b\} = \left [ 
\begin{array}{r}
 31  \\
 31 \\
 31 
\end{array} 
\right ]
\label{eq:example2}
\end{equation}

\vspace{0.15 in}
\noindent
Here we adopt the notation [~] for matrices and \{~\} for column vectors,
and observe -- by inspection -- that the analytic solution 
to \ref{eq:example2} is $\{x\} = \{1, 1, 1\}^T$.
Now let vector $\{x\}_k$ be the $k^{th}$ estimate of $\{x\}$.
Because $\{x\}_k$ will not be equal to $\{x\}$ exactly,
we define a residual $\{r\}_k$ as

\begin{equation}
\{r\}_k = \{b\}-[A]\{x\}_k 
\label{eq: eq-2}
\end{equation}

\vspace{0.15 in}
\noindent
and an update step

\begin{equation}
\{x\}_{k+1} = \{x\}_k + \{\Delta x\}_k.
\label{eq: eq-3}
\end{equation}

\vspace{0.15 in}
\noindent
for each iteration.
During each iteration of our numerical algorithm,
the objective is to compute the update $\{\Delta x\}_k$ in such
as way that $\{r\}_{k+1} = \{b\} - [A]\{x\}_{k+1}= 0$.
Satisfying $\{r\}_{k+1} = 0$ is equivalent to solving

\begin{equation}
\{r\}_k = [A]\{\Delta x\}_k.
\label{eq: eq-1}
\end{equation}

\vspace{0.15 in}\noindent
at each iterate.
Details of the numerical algorithm are as follows:

\vspace{0.15 in}
\begin{description}
\item{[1]}
\indent{Initialization: fill $ [A] $ matrix, and set $ \{x\}_1 = (0,0,1)^T $,
and error = 1.0.}
\vskip 0.05truein
\item{[2]}
\indent{For k = 1, calculate the residual vector $\{r\}_k $ according eq.
\ref{eq: eq-2},}
\vskip 0.05truein
\item{[3]}
\indent{Solve for $ \{\Delta x\}_k $ according to equation:
$([L][U]) \{\Delta x\}_k = \{r\}_k $. In solving this problem, matrix $ [A] $ is
decomposed into $ [L][U] $ by Cholesky decomposition method, where $[L]$ and
$[U]$ are lower and upper triangle matrices. }
\vskip 0.05truein
\item{[4]}
\indent{Update $\{x\}_{k+1} = \{x\}_k + \{\Delta x\}_k$.}

\vskip 0.05truein
\item{[5]}
\indent{Estimate average error by calculating the Euclidean norm of
$\{\Delta x\}_k $, and also printing the iteration number k and the error.}
\vskip 0.05truein
\item{[6]}
\indent{Compare error with 1E-7.If error $\leq$ 1E-7, k = k + 1,
go to step [2].
If error $ \geq $ 1E-7 stop iteration and print out the results.}
\end{description}
\vspace{0.2 in}

\noindent
A built-in linear equation solver is used to perform
the $[L][U]$ decomposition, and to solve the equations.
We set the smallest error to be 1E-7, and the iteration will be terminated if the error
is less than 1E-7. An initial value of error is set to be larger than 1.E-7 to start
the iteration. Here this value is set to be 1.0.
Details of the input file are as follows:\\
\vskip 0.1truein
\begin{footnotesize}
\vspace{0.10 in}

\noindent
{\rule{2.3 in}{0.035 in} START OF INPUT FILE \rule{2.3 in}{0.035 in} }
\begin{verbatim}
    print "*** INPUT MATRIX [A], VECTOR {b} and TRIAL VECTOR {x} \n\n";

    A = [1, 15, 15; 15, 1, 15; 15, 15, 1];
    b = [31; 31; 31];
    x = [0; 0; 1];
    norm = 1.0;

    print "*** STARTING TO SOLVE [A]{x} = {b} \n";

    PrintMatrix(A,b,x);

    print "*** START ITERATION  \n\n";

    for( k = 1; norm > 1E-7; k = k + 1) {
         r       = b - A*x;
         delta_x = Solve(A, r);   
         x       = x + delta_x;
         norm    = L2Norm(delta_x);
         print " iteration no = ", k, "\n";
         print " norm         = ", norm, "\n";
    }
    print "\n*** OUTPUT SOLUTION  \n";
    PrintMatrix(x);
\end{verbatim}
{\rule{2.3 in}{0.035 in} END OF INPUT FILE \rule{2.3 in}{0.035 in} }
\end{footnotesize}

\noindent
where, {\tt Solve()} is a function for solving a system of linear equation,
and {\tt PrintMatrix()} is the function for print one or more number of matrices.
The function {\tt L2Norm()} computes the Euclidean norm of a vector.
{\tt print} is a keyword for printing a quantity or comment enclosed
within double quotes -- "~" {\tt $ \backslash n $} is the keyword for starting a newline.
The outputs are :
\vskip 0.1truein
\begin{footnotesize}
\vspace{0.10 in}
\noindent
{\rule{2.3 in}{0.035 in} START OF INPUT FILE \rule{2.3 in}{0.035 in} }
\begin{verbatim}
*** INPUT MATRIX [A], VECTOR {b} and TRIAL VECTOR {x} 

*** STARTING TO SOLVE [A]{x} = {b} 

MATRIX : "A"

row/col          1            2            3          
      units                                 
   1            1.00000e+00  1.50000e+01  1.50000e+01
   2            1.50000e+01  1.00000e+00  1.50000e+01
   3            1.50000e+01  1.50000e+01  1.00000e+00

MATRIX : "b"

row/col          1          
      units             
   1            3.10000e+01
   2            3.10000e+01
   3            3.10000e+01

MATRIX : "x"

row/col          1          
      units             
   1            0.00000e+00
   2            0.00000e+00
   3            1.00000e+00

*** START ITERATION  

 iteration no =     1.0000e+00 
 norm         =     1.4142e+00 
 iteration no =     2.0000e+00 
 norm         =     0.0000e+00 

*** OUTPUT SOLUTION  

MATRIX : "x"

row/col          1          
      units             
   1            1.00000e+00
   2            1.00000e+00
   3            1.00000e+00
\end{verbatim}
{\rule{2.4 in}{0.03 in} END OF INPUT FILE \rule{2.4 in}{0.03 in} }
\end{footnotesize}

\vspace{0.15 in}\noindent
The numerical procedure converges to the specified
order of accuracy in only two iterations.
Because equations ~\ref{eq:example2} are linear,
we could compute and print the solution by simply typing

\begin{verbatim}
    x = Solve( A, b);
    PrintMatrix(x);
\end{verbatim}

\vspace{0.15 in}
\noindent
{\tt Solve()} is a built-in linear equation solver,
and {\tt PrintMatrix()} is a function for printing one or more matrices.

\section{Objectives}
The primary objectives of this research project are (1) to creat a
computational environment that supports the engineering calculation, finite element analysis,
and engineering optimizations; (2) to develop a finite element analysis package under such
computational environment. This report is divided in two parts. The part I
describes the computational environment, and part II, the finite element package. The user manual and
example input files are located in Part IV.

\vspace{0.15 in}\noindent
The objective of part I is to develop a low-level program language
for engineering analysis. The language should be flexible enough to deal
with most of engineering problems, including finite element analysis.
That requires the language to be able to perform the following tasks : 

\begin{description} 
  \item {(a)}
   {\bf Control-flow tasks :} The language should have the ability to perform looping
         constructs and conditional branching.
  \item {(b)}
   {\bf Mathematics tasks :} The language should have the ability to perform the basic
        mathematics operations, numerical algorithm operations, matrix operations,
        and to solve algebraic equations, eigenvalue equations.
  \item {(c)}
   {\bf Engineering tasks :}The language should also have the ability to perform basic
        engineering analysis, such as dynamic and finite element analysis with engineering
        units.
\end{description} 

\noindent
The basic design principles of this stack machine and its application
in finite element analysis is given in part I.



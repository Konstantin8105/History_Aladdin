\chapter{dummy}
\section{Introduction}
The concept of base isolation is to provide the flexibility and
energy dissipation capacity by
a specially designed isolation system. The isolation system
is placed between the superstructure and
its foundation. Conceptually, base isolation reduces ground motion
transmitted to the superstructure above the
isolator, reducing the response of a typical structure
and the corresponding loading. Thus base-isolated
structures require lighter structural members than
nonisolated structures. In addition, non-structural
components (utilities, partitions, suspended ceiling,
equipments etc.) are also less likely to be damaged
in a base-isolated structure~\cite{ferritto}.

The systems used for seismic protection of buildings
and bridges are mainly elastometric
or sliding isolation systems. Flexibility in elasometric isolation
systems is provided by elastometric bearings
(laminated rubber bearings reinforced with steel plates).
Energy dissipation capacity is provided
by the led plug within the rubber unit,
as in lead-damping elastometric bearings or by steel dampers.


Sliding bearings(Teflon-slider sliding on a stainless steel
plate) in sliding isolation systems support and
decouple the superstructure from the ground. Sliding
bearing dissipate energy by means of frictions. Restoring
force or recentering capacity is provided by helical springs
or by springs in the form of rubber cylinders
\cite{nagarajaiah}.

%\section{Element for modelling base-isolator}

The existing programs such as NPAD, specially developed for
base-isolation structures~\cite{way-jeng}
or ANSR, a general purpose finite element program~\cite{mondkar},
have elastic-plastic nonlinear elements that
can be used to model elasometric isolators.
Element developed in~\cite{nagarajaiah} has the capacity to handle both
elstometric and/or sliding isolation systems.

In general, the elastometric isolator can be represented by
material models with bilinear or power-law behavior. And
sliding systems can be represented by models with rigid-plastic behavior.
In this study a elasto-plastic model has been used to model the isolator.





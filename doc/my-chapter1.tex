
\chapter{Introduction to ALADDIN}

\section{Problem Statement}

\vspace{0.15 in}
\noindent\hspace{0.5 in}
This report describes the development and capabilities of ALADDIN (Version 1.0),
an interactive computational toolkit for the matrix
and finite element analysis of engineering systems.
The current target application area for ALADDIN is
design and analysis of traditional Civil Engineering structures,
such as highway bridges and earthquake-resistant buildings.
With literally hundreds of engineering analysis and
optimization computer programs having been written in the past 10-20 years (see
references ~\cite{abaqus92,balling83,mondkar75,nye87,nye82,nye86,wu86,zienkiewitz89}
for some examples), a reader might rightfully
ask {\it who needs to write another engineering analysis package ?}

\vspace{0.15 in}
\noindent\hspace{0.5 in}
We respond to this challenge, and motivate the short- and long-term goals of
this work by first noting that during the past two decades,
computers have been providing approximately 25\% more power per dollar per year.
Advances in computer hardware and software have allowed for
the exploration of many new ideas, and have been a key catalyst in
what has led to the maturing of computing as a discipline.
In the 1970's computers were viewed primarly as
machines for research engineers and scientists -- compared to today's standards,
computer memory was very expensive, and central processing units were slow.
Early versions of structural analysis and finite element computer programs,
such as ABAQUS \cite{abaqus92}, ANSR ~\cite {mondkar75},
and FEAP ~\cite{zienkiewitz89} were written in the FORTRAN computer language,
and were developed with the goal of optimizing
numerical and/or instructional considerations alone.
These programs offered a restricted, but well implemented,
set of numerical procedures for static structural analyses,
and linear/nonlinear time-history response calculations.
And even though these early computer programs were not particularly easy to use,
practising engineers gradually adopted them because they allowed for
the analysis of new structural systems in a ways that were previously intractable.

\vspace{0.15 in}
\noindent\hspace{0.5 in}
During the past twenty years, the use of computers in engineering
has matured to the point where importance is now placed on ease of use,
and a wide-array of services being made available
to the engineering profession as a whole.
Computer programs written for engineering computations
are expected to be fast and accurate, flexible, reliable, and of course, easy to use.
Whereas an engineer in the 1970's might have been satisfied by a
computer program that provided numerical solutions to a very specific engineering problem,
the same engineer today might require the engineering analysis,
plus computational support for design code checking, optimization,
interactive computer graphics, network connectivity, and so forth.
Many of the latter features are not a bottleneck for getting the job done.
Rather, features such as interactive computer graphics simply make the
job of describing a problem and interpreting results easier --
the pathway from ease-of-use to productivity gains is well defined.
It is also well worth noting that computers once viewed as a tool for computation alone,
are now seen as an indispensable tool for computation and communications.
In fact, the merging of computation and communications is making fundamental
changes to the way an engineer conducts his/her day-to-day business activities.
Consider, for example, an engineer who has access to a high speed
personal computer with multimedia interfaces and global network connectivity,
and who happens to be part of a geographically dispersed development team.
The team members can use the Internet/E-mail for day-to-day communications,
to conduct engineering analyses at remote sites,
and to share design/analysis results among the team members.
Clear communication of engineering information among the team members
may be of paramount importance in determining the smooth development of a project.

\vspace{0.15 in}
\noindent\hspace{0.50 in}
The difficulty in following-up on the abovementioned hardware advances with
appropriate software developments is clearly reflected in
the economic costs of project development.
In the early 1970's software consumed approximately 25\% of total costs,
and hardware 75\% of total costs for development of data intensive systems.
Nowadays, development and maintenance of software
typically consumes more than 80\% of the total project costs.
This change in economics is the combined result of falling hardware costs,
and increased software development budgets needed to
implement systems that are much more complex than they used to be.
Whereas one or two programmers might have written a complete program twenty years ago,
today's problems are so complex that teams of programmers are needed
to understand a problem and fill-in the details of required development.

\vspace{0.15 in}
\noindent\hspace{0.50 in}
When a computer program has a poorly designed architecture,
its integration with another package can be very difficult,
with the result often falling short of users' expectations.
Let's suppose, for example, that we wanted to interface the
finite element package FEAP ~\cite{zienkiewitz89} with the interactive
optimization-based design environment called DELIGHT ~\cite{balling83,nye87,wu86}.
Since FEAP was not written with interfaces to external environments
as a design criterion, a programmer(s)
faced with this task would first have to figure out
how FEAP and DELIGHT work (not an easy task),
and then devise a mapping from DELIGHT's external
interface routines to FEAP's subroutines.
In the first writer's opinion, such a mapping is likely to exist,
but only after several subroutines have been added to FEAP.
The programmer(s) would need the computer skills and tenacity to stick-with the
lengthy period of code development that would ensue.
And what about the result ?
In our experience ~\cite{austin87a,austin87b,balling83},
the integrated DELIGHT-FEAP tool would most likely do a
very good job of solving a narrow range of problems.
Extending DELIGHT-FEAP's capabilities to a wider range of
applications could be a greater technical challenge than the orginal integration.
These obstacles, coupled with the marginal benefits, have led us to conclude that
such an ad-hoc approach to tool integration is not worthwhile.
A smarter move would be to focus our long-term research efforts
on the development of methods that will lead to
the integration of engineering systems in a natural way.

\section{ALADDIN Components}

\vspace{0.15 in}
\noindent\hspace{0.50 in}
Project ALADDIN is motivated by the lack of structure (i.e. organization) in
finite element and optimization-based computer programs needed for efficient tool integration.
The goals of this work are to better understand the structure these
packages should take -- we will investigate the problem by 
designing and implementing a system specification for how a
matrix and finite element system ought to work.
The system specification will include:

\begin{description}
\item{[1]}
{\tt A Model :} The model will include data structures
for the information to be stored, and a stack machine for the execution
of the matrix and finite element algorithms.

\item{[2]}
{\tt A Language :} The language will be a means for describing the
matrices and finite element mesh -- it will
act as the user interface to the underlying model.

In traditional approaches to problem solving,
engineers write the details of a problem and its solution procedure on paper.
They use physical units to add clarity to the problem description,
and may specify step-by-step details for a numerical solution to the problem.
We would like the ALADDIN language to be textually descriptive,
and strike a balance between simplicity and extensibility.
It should use a small number of data types and control structures,
incorporate physical units, and yet, be descriptive enough so
that the {\it pencil and paper} and
ALADDIN {\it problem description} files are almost the same.

\item{[3]}
{\tt Defined Steps and Ordering of the Steps :} The steps will define the
transformations (e.g. nearly all engineering processes will require iteration and branching)
that can be carried out on system components.

\item{[4]}
{\tt Guidance for Applying the Specification :}
Guidance includes factors such as descriptive problem description
files and documentation.

\end{description}

\vspace{0.15 in}\noindent
Our research direction is inspired in part by the systems
integration methods developed for the European ESPRIT Project ~\cite{kronlof93},
and by the success of C.
Although the C programming language has only 32 keywords,
and a handful of control structures, its judicious combination with
external libraries has resulted in the language
being used in a very wide range of applications.

\vspace{0.15 in}
\noindent\hspace{0.5 in}
Figure \ref{fig: aladdin-development} is a
schematic of the ALADDIN (Version 1.0) architecture,
and shows its three main parts: (1) the kernel;
(2) libraries of matrix and finite element functions, and (3) the input file(s).

\begin{figure}[th]
\epsfxsize=5.5truein
\centerline{\epsfbox{my-chapter1-fig1.ps}}
\caption{High Level Components in ALADDIN (Version 1.0)}
\label{fig: aladdin-development}
\end{figure}

\vspace{0.15 in}
\noindent\hspace{0.5 in}
Specific engineering problems are defined in ALADDIN problem
description files, and solved using
components of ALADDIN that are part interpreter-based, and part compiled C code.
It is important to keep in mind that as the speed of CPU processors increases,
the time needed to prepare a problem description increases
relative to the total time needed to work through an engineering analysis.
Hence, clarity of an input file's contents is of paramount importance.
In the design of the ALADDIN language we attempt to achieve these goals with:
(1) liberal use of comment statements (as with the C programming language,
comments are inserted between {\tt /* .... */}),
(2) consistent use of function names and function arguments,
(3) use of physical units in the problem description, and
(4) consistent use of variables, matrices,
and structures to control the flow of program logic.

\vspace{0.15 in}
\noindent\hspace{0.5 in}
ALADDIN problem descriptions and their solution algorithms
are a composition of three elements:
(1) data, (2) control, and (3) functions ~\cite{salter76}:

\vspace{0.15 in}\noindent
{\bf [1] Data :} 
ALADDIN supports three data types, ``character string'' for variable names,
physical quantities, and matrices of physical quantities for engineering data.
For example, the script of code

\begin{footnotesize} 
\begin{verbatim}
       xCoord   = 2 m;
       xVelociy = 2 m / sec;
\end{verbatim}
\end{footnotesize} 

\vspace{0.15 in}\noindent
defines two physical quantities, {\tt xCoord} as 2 meters,
and {\tt xVelocity} as 2 meters per second.
Floating point numbers are stored with double precision accuracy, and are
viewed as physical quantities without units.
There are no integer data types in ALADDIN.

\vspace{0.15 in}\noindent
{\bf [2] Control :} Control is the basic mechanisms in a programming
language for the specification of looping constructs and conditional branching.

\begin{figure}[ht]
\vspace{0.15 in}
\epsfxsize=5.0truein
\centerline{\epsfbox{my-chapter1-fig3.ps}}
\caption{Branching and Looping Constructs in ALADDIN}
\label{fig: aladdin-branch-and-loop}
\end{figure}

\vspace{0.15 in}\noindent
In Chapter 2 we will see that ALADDIN supports the
{\tt while} and {\tt for} looping constructs,
and the {\tt if} and {\tt if-then-else} branching constructs.
The data and control components of the ALADDIN language are
implemented as a finite-state stack machine model, which follows in the
spirit of work presented by Kernighan and Pike ~\cite{kernighan-pike}.
ALADDIN's stack machine reads blocks of command statements from either a problem
description file, or the keyboard,
and converts them into an array of machine instructions.
The stack machine then executes the statements.

\vspace{0.15 in}\noindent
{\bf [3] Functions :} 
The functional components of ALADDIN provide hierarchy to the
solution of our matrix and finite element processes,
and are located in libraries of compiled C code,
as shown on the right-hand side of Figure \ref{fig: aladdin-development}.
Version 1.0 of ALADDIN has functional support for basic matrix operations,
that includes numerical solutions to linear equations, 
and the symmetric eigenvalue problem.
A library of finite element functions is provided.

\begin{figure}[ht]
\vspace{0.10 in}
\epsfxsize=4.5truein
\centerline{\epsfbox{my-chapter1-fig2.ps}}
\caption{Schematic of Functions in ALADDIN}
\label{fig: aladdin-functions}
\end{figure}

\vspace{0.15 in}\noindent
Figure ~\ref{fig: aladdin-functions} shows the general
components of a function call, it's input argument list,
and the function return type.
A key objective in the language design is to devise families of
library functions that will make ALADDIN's problem
solving ability as wide as possible.

\vspace{0.15 in}
\noindent\hspace{0.5 in}
Version 1.0 of ALADDIN does not support user-defined
functions in the input-file (or keyboard) command language.

\vspace{0.15 in}
\noindent\hspace{0.50 in}
The strategy we have followed in ALADDIN's development is to keep
the number and type of arguments employed in library function calls small.
Whenever possible, the function's return type and arguments 
should be of the same data type, thereby allowing the output
from one or more functions to act as the input to following function call.
More precisely, we would like to write input code that takes the form

\begin{footnotesize} 
\begin{verbatim}
   returntype1 = Function1();                            /* <= application area 1 */
   returntype2 = Function2();                            /* <= application area 1 */

   returntype3 = Function3( returntype1, returntype2 );  /* <= application area 2 */

or even 

   returntype3 = Function3( Function1(), Function2() );
\end{verbatim}
\end{footnotesize} 

\vspace{0.15 in}\noindent
If {\tt Function1()} and {\tt Function2()} belong to application area 1 (e.g. matrix analysis),
and {\tt Function3()} belongs to application area 2 (e.g. finite element analysis; optimization),
then this language structure allows
application areas 1 and 2 to be combined in a natural way.
In Chapters 2 to 6, we will see that most of the built-in functions
accept one or two matrix arguments, and return one matrix argument.
Occasionally, we will see built-in function that have more than
two function arguments, and return a quantity instead of a matrix.
Collections of quantities are returned from functions by
bundling them into a single matrix -- the individual quantities
are then extracted as elements of the function return type.

\begin{table}[th]
\vspace{0.10 in}
\tablewidth = 5.5truein
\begintable
Factor                   | Interpreted Code | Compiled Code  \crthick
User Control             | High             | Very Low       \cr
Required User Knowledge  | High             | Medium         \cr
Speed of Execution       | Slow             | Fast           
\endtable
\vspace{0.01 in}
\caption{Trade-Offs -- Interpreted Code Versus Compiled Code}
\label{tab: tradeoffs-in-design}
\end{table}

\vspace{0.15 in}\noindent
Table \ref{tab: tradeoffs-in-design} contains a summary of trade-offs
between the use of interpreted code versus compiled C code.
A key advantage of interpreters is flexibility -- problem parameters and
algorithms may be modified during the problem solving process,
and without having to recompile source code.
This feature reduces the time needed to work
through an interation of the problem solving process.
The well known trade-off is speed of execution.
Interpreted code executes much slower -- approximately ten times slower --
than compiled C code.
Consequently, we expect that as ALADDIN evolves, new algorithms
will be developed in tested as interpreted code,
and once working, will be converted into libraries of compiled C code having
an interface that fits-in with the remaining library functions.


\section{Scope of this Report}

\vspace{0.15 in}
\noindent\hspace{0.50 in}
This report is divided into four parts.
In Part II we will see that the ALADDIN language supports
variable arithmetic with physical units, matrix operations with units,
and looping and branching control structures,
where decisions are made with quantities having physical units.
Chapters 3 and 4 demonstrate use of the ALADDIN language via
a suite of problems -- we compute the roots of nonlinear equations,
demonstrate the Han-Powell method of optimization,
and solve several problems from structural
engineering and structural dynamics. 

\vspace{0.15 in}
\noindent\hspace{0.50 in}
Part III of this report describes the built-in functions for the
generation of finite element meshes, external loads,
and finite element modeling assumptions. We demonstrate these
features by solving a variety two- and three-dimensional finite element problems.

\vspace{0.15 in}
\noindent\hspace{0.50 in}
We have written Part IV for ALADDIN developers,
who may need to understand the data structures and algorithms
used to read and store matrices, to create and store finite element meshes,
and to construct the stack machine.
Readers of this section are assumed to have a
good working knowledge of the C programming language.

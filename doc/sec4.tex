\thesischapter{4}{Numerical Examples}

Performance of the flat shell finite elements is evaluated by working
through three numerical examples. They are:

(a) square plate simply supported on four edges,

(b) cantilever I-shape cross section beam, and

(c) folded plate simply supported on two opposite sides.

\section{Square Plate Simply Supported on Four Edges}

Consider the square plate simply supported on four edges, as shown in Fig
4.1. Two load cases are considered; (a) a uniform loading over the
entire plate, and (b) a concentrated point load at the center of the
plate. For each load case, computed displacements are compared to
analytical displacements.

\subsection{Uniform Loading over the Entire Plate}

Consider a square plate simply supported on all four edges subjected to a
uniforming loading, shown in Fig. 4.1. 
\begin{figure}[htb]
\vspace{3.7in}
\special{psfile="figs.d/fig401.ps" hscale=80 vscale=80 hoffset=-10 voffset=-320}
\label {fig401}
\caption{{\em Meshes of square plate simply supported on 4 edges.}}
\vskip 0.2truein
\end{figure}
The exact transverse displacement 
at the center, $w_{c}^{*}$, from the plate theory ~\cite{peter,
timoshenko} can be expressed as
\bea
w_{c}^{*} = 0.00406 \frac{q_{0}a^{4}}{D}          \label {eq401}
\eea
where $q_{0}$ is the uniforming loading, $a$ is the length of edge of the
square plate and
\bea
D = \frac{Eh^{3}}{12(1-\mu^{2})}               \label {eq402}
\eea
Substituting the values of $E$, $\mu$, $h$, $q_{0}$ and $a$  of this
example into equations (\ref{eq401}) and (\ref{eq402}) gives \  
\[D = 1.1446886 \times 10^{5}\]
and 
\[w_{c}^{*} = 1.064045 \times 10^{-1} (in).\]

Because the plate geometry is symmetric about $x$-axis and $y$-axis, only
one quarter of the plate is taken for numerical computation.
Regular meshes on the plate quarter with $N=2$ and $4$ (See Fig. 4.1)
are considered. Numerical results are evaluated by comparing the 
transverse displacements of the plate center, $w_{c}$, to the exact
theoretical solution. A summary of results is provided in Table \ref{tab1}. 

\begin{table}[htb]
\vskip 0.4truein
\begintable
\multispan{3}\cr
                  Meshes N               |    $2$    |     $4$       \cr
Displacements $w_{c}$ ($\times 10^{-1}$) | $1.06027$ |  $1.06405$    \cr
                   Errors                | $0.355\%$ | $0.000489\%$ 
\endtable
\vskip 0.1truein
\caption{{\em The transverse displacements at the center of the
square plate simply supported on 4 edges under uniform load over the
entire plate with different meshes and the comparations with the exact
theoretical solution.}}
\label{tab1}
\vskip 0.1truein
\end{table}

\subsection{Concentrated Loading at the Center}

Similarly, consider the square plate subjected to a concentrated 
loading, $P=30000 bl$ at the center. The theoretical exact transverse 
displacement at the center, $w_{c}^{*}$, from the plate theory 
~\cite{peter, timoshenko}, can be expressed as
\bea
w_{c}^{*} = 0.0115999 \frac{Pa^{2}}{D}          \label {eq404}
\eea
where $P$ is the concentrated loading at the center, $a$ is the length 
of edge of the square plate, and $D$ is as described in equation 
(\ref{eq402}). Substituting the values of $D$, $P$, and $a$ into equation
(\ref{eq404}) gives
\bea
w_{c}^{*} = 3.0401019 \times 10^{-1} (in)
\eea
Once again, numerical displacements are computed for only a quarter of
the plate. With regular meshes $N=2$, $4$ and $8$ (See Fig. 4.1), the 
transverse displacements at the center $w_{c}$ computed by the elements 
described in this thesis are compared with the exact theoretical solution 
in Table \ref{tab2}.

\begin{table}[htb]
\vskip 0.4truein
\begintable
\multispan{4}\cr
               Meshes N                 |   $2$   |   $4$   |   $8$    \cr
Displacements $w_{c}$ ($\times 10^{-1}$)|$3.32666$|$3.12850$|$3.06664$ \cr
                Errors                  |$9.426\%$|$2.908\%$|$0.873\%$
\endtable
\vskip 0.1truein
\caption{\em {The transverse displacements at the center of the
square plate simply supported on 4 edges under concentrated point load
at the center with different meshes and the comparations with the exact
theoretical solution.}}
\label{tab2}
\vskip 0.1truein
\end{table}

\section{Cantilever I-shape Cross Section Beam}

In the second example, displacements are computed for a cantilever beam
having an I-shape cross section. Three load cases are considered. The 
first is displacements due to a concentrated load at the center of the 
free end, as shown in Fig. 4.2. Second, displacements are computed for a
uniform load on the center line of the top face, as shown in Fig. 4.3.
The third is under two level concentrated loads at the flanges of the
free end in opposite directions along y, as shown in Fig. 4.4.

\subsection{Concentrated Load at the Center of the Free End}

Displacements are computed for a cantilever beam with I-shape cross 
section loaded with point load $P$ at the center of the free end shown 
in Fig. 4.2. 
\begin{figure}[htb]
\vspace{3.5in}
\special{psfile="figs.d/fig403.ps" hscale=85 vscale=85 hoffset=-10
voffset=-320}
\label {fig403}
\caption{{\em Cantilever I-beam under a concentrated load at the end}}
\vskip 0.2truein
\end{figure}
The solution of the transverse displacement at the free end, 
$w^{*}$, from the beam bending theory with shear effect is expressed as 
\bea
w^{*}=\frac{PL^{3}}{3EI}+\frac{PL}{A_{w}G}           \label{eq450}
\eea
where the second term represents shear effect. $P$ is the load, and $L$ 
is the length of the beam. $I=33.8802$ is modulus of the area and 
$A_{w}=1.1875$ is area of the web. Suppose that the shear modulus is as 
$E/G=2.5$. Substitute values of $I$, $A_{w}$, $E$, $G$, $P$ and $L$ 
into equation (\ref{eq450}), so that
\[ w^{*}=2.85552 \times 10^{-2} (in)\]
The transverse displacements $w$ at point 1 (see Fig. 4.2) are computed 
by using the elements described in this thesis. Results are tabulated 
in Table \ref{tab3}. Speed of convergence for numerical results is define. The 
convergent rate $\lambda$ 
\[
\lambda_{N_{i}} = \frac{w_{N_{i}}-w_{N_{i-1}}}{w_{N_{i}}},
\]
with $N_{1}=1$, $N_{2}=2$, $N_{3}=4$, $N_{4}=8$ and $N_{5}=16$ describes
the rate of convergence for numerical results. These results are tabulated 
in the same table. We can observe that the displacement computed with $N=8$ 
is already very closed to the result from the beam bending theory.

\begin{table}[htb]
\vskip 0.4truein
\begintable
\multispan{5}\cr
               Meshes N            |    $2$   |    $4$   |    $8$   |   $16$    \cr
Displacement $w$ ($\times 10^{-2}$)|$-2.65646$|$-2.80859$|$-2.85107$|$-2.85482$ \cr
    Convergent rates $\lambda$     |$        $| $5.417\%$| $1.490\%$|$0.1314\%$ 
\endtable
\vskip 0.1truein
\caption{\em {The transverse displacements at the free end of the
I-shape section cantilever beam under concentrated point load at the 
center of the free end with different meshes and their convergent rates.}}
\label{tab3}
\vskip 0.1truein
\end{table}

\subsection{Uniform Load along Center Line of Top Face}

Also, I look at a cantilever beam with I-shape cross section under a 
uniformly distributed line load $q_{0}$ along the center line of the 
top face, as shown in Fig. 4.3. 
\begin{figure}[htb]
\vspace{3.5in}
\special{psfile="figs.d/fig404.ps" hscale=85 vscale=85 hoffset=-10
voffset=-320}
\label {fig404}
\caption{{\em Cantilever I-beam under a uniformly distributed line load 
along the center line of the top face}}
\vskip 0.2truein
\end{figure}
The solution of the transverse displacement at the free end, $w^{*}$, 
from the beam bending theory with shear effect is expressed as 
\bea
w^{*}=\frac{q_{0}L^{4}}{8EI}+\frac{q_{0}L^{2}}{2A_{w}G}      \label{eq460}
\eea
where similarly the second term represents shear effect. $q_{0}$ is 
the unique load, and $L$ is the length of the beam. Substituting 
$q_{0}=20 bl/in$, $L=40"$, $I=33.8802$, $A_{w}=1.1875$ and $E/G=2.5$ into 
equation (\ref{eq460}) produces
\[ w^{*}=2.22585 \times 10^{-2}\]

The transverse displacements $w$ at point 1 (see Fig. 4.3) computed 
by using the elements described in this thesis; convergent rates are 
tabulated in Table \ref{tab4}. 

\begin{table}[htb]
\vskip 0.4truein
\begintable
\multispan{5}\cr
              Meshes  N            |   $2$    |    $4$   |    $8$   |   $16$    \cr
Displacement $w$ ($\times 10^{-2}$)|$-2.19184$|$-2.22248$|$-2.23926$|$-2.24607$ \cr
     Convergent rates $\lambda$    |$        $| $1.379\%$|$0.7494\%$|$0.3032\%$
\endtable
\vskip 0.1truein
\caption{\em {The transverse displacements at the free end of the
I-shape section cantilever beam under uniform load along center line of
the top face with different meshes and their convergent rates.}}
\label{tab4}
\vskip 0.1truein
\end{table}

From the two cases of the I-beam above, we can see that the
displacements from beam bending theory and computed by elements in this
thesis are in very close agreement.

\subsection{Two Level Concentrated Loads at the Flanges of the
Free End in Opposite Directions Along y}

Displacements are computed for a cantilever beam having I-shape cross 
section, subject to two concentrated load $P$ at the flanges of the 
free end in opposite directions along y, as shown in Fig. 4.4. 
\begin{figure}[htb]
\vspace{3.5in}
\special{psfile="figs.d/fig405.ps" hscale=85 vscale=85 hoffset=-10
voffset=-320}
\label {fig405}
\caption{{\em Cantilever I-beam under two level concentrated loads at
the flanges of the free end in opposite directions along y}}
\vskip 0.2truein
\end{figure}
The transverse displacements $w$ at point 1 (see Fig. 4.4) are 
computed by using the elements described in this thesis and their 
convergent rates are tabulated in Table \ref{tab5}.

\begin{table}[htb]
\vskip 0.4truein
\begintable
\multispan{5}\cr
              Meshes N             |   $2$   |   $4$   |   $8$   |   $16$   \cr
Displacement $w$ ($\times 10^{-1}$)|$2.45142$|$2.50528$|$2.51482$|$2.52135$ \cr
     Convergent rates $\lambda$    |$       $|$2.150\%$|$0.379\%$|$0.2590\%$
\endtable
\vskip 0.1truein
\caption{\em {The transverse displacements at point 1 of the
I-shape section cantilever beam under two lever concentrated loads at
the flanges of the free end in opposite directions along y with
different meshes and their convergent rates.}}
\label{tab5}
\vskip 0.1truein
\end{table}

The horizontal displacements along $y$, $v$, at point 1 (see Fig. 4.4)
computed by using the elements described in this thesis and their
convergent rates are tabulated in Table \ref{tab6}.

\begin{table}[htb]
\vskip 0.4truein
\begintable
\multispan{5}\cr
              Meshes N             |    $2$   |    $4$   |    $8$   |   $16$    \cr
Displacement $w$ ($\times 10^{-1}$)|$-1.38753$|$-1.46572$|$-1.48824$|$-1.49583$ \cr
     Convergent rates $\lambda$    |$        $|$5.335\%$ |$1.513\%$ |$0.5074\%$ 
\endtable
\vskip 0.1truein
\caption{\em {The horizontal displacements at point 1 of the
I-shape section cantilever beam under two lever concentrated loads at
the flanges of the free end in opposite directions along y with
different meshes and their convergent rates.}}
\label{tab6}
\vskip 0.1truein
\end{table}

\section{Folded Plate Simply supported on two opposite sides}

As the third example, I consider a folded plate, as shown in Fig. 4.5.
\begin{figure}[htb]
\vspace{3.5in}
\special{psfile="figs.d/fig402.ps" hscale=85 vscale=85 hoffset=-10 voffset=-220}
\label {fig402}
\caption{{\em A folded plate simply supported on two opposite sides.}}
\vskip 0.2truein
\end{figure}
The meshes with $N=1$, $2$ and $4$ are used and the results, the
transverse displacements $w$ at points $1$ and $2$, and their convergent 
rates $\lambda$ are tabulated in Table \ref{tab7} and Table \ref{tab8}, 
respectively.

\begin{table}[htb]
\vskip 0.4truein
\begintable
\multispan{4}\cr
             Meshes N              |    $1$   |    $2$   |    $4$    \cr
Displacement $w$ ($\times 10^{-1}$)|$-1.38009$|$-1.41003$|$-1.42237$ \cr
     Convergent rate $\lambda$     |          | $2.144\%$|$0.8465\%$
\endtable
\vskip 0.1truein
\caption{\em {The transverse displacements at point 1 of the
folded plate simply supported on two opposite sides under uniform
load along the center line with different meshes and their convergent
rates.}}
\label{tab7}
\vskip 0.1truein
\end{table}

\begin{table}[htb]
\vskip 0.4truein
\begintable
\multispan{4}\cr
              Meshes N             |
 $\ \ \ \ \ 1\ \ \ \ \ $|   $2$    |    $4$    \cr
Displacement $w$ ($\times 10^{-1}$)|
                        |$-1.35207$|$-1.36062$ \cr
     Convergent rate $\lambda$     |
                        |          |$0.6284\%$
\endtable
\vskip 0.1truein
\caption{\em {The transverse displacements at point 2 of the
folded plate simply supported on two opposite sides under uniform
load along the center line with different meshes and their convergent
rates.}}
\label{tab8}
\vskip 0.1truein
\end{table}

From the examples of the I-beam and the folded plate, it can be
observed that the solutions rapidly converge, and are stable.
